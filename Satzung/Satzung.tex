\documentclass[fontsize=12pt,paper=a4,pagesize]{scrartcl}
\usepackage{lmodern}
\usepackage[T1]{fontenc}
\usepackage[utf8]{inputenc}
\usepackage[ngerman]{babel}
\renewcommand*{\othersectionlevelsformat}[3]{\S\,#3\autodot\enskip}

\title{Satzung des Hackspace Jena e.V.}
\date{In der Fassung vom 10. Januar 2012}

\begin{document}
\maketitle
\section{Name, Sitz, Geschäftsjahr}

\begin{enumerate}
	\item Der Verein trägt den Namen ``Hackspace Jena''. Der Verein soll in
		das Vereinsregister beim Amtsgericht Jena eingetragen und der Name
		dann um den Zusatz ``e.V.'' ergänzt werden.
	\item Der Verein hat seinen Sitz in Jena. Das Geschäftsjahr entspricht
		dem Kalenderjahr.
\end{enumerate}

\section{Gemeinnützigkeit}

\begin{enumerate}
	\item Der Verein verfolgt ausschließlich und unmittelbar gemeinnützige
		Zwecke im Sinne des Abschnittes ``Steuerbegünstigte Zwecke'' der
		Abgabenordnung.
	\item Der Verein ist selbstlos tätig; er verfolgt nicht in erster Linie
		eigenwirtschaftliche Zwecke.
	\item Mittel des Vereins dürfen nur für die satzungsmäßigen Zwecke
		verwendet werden. Die Mitglieder erhalten keine Zuwendungen aus
		Mitteln des Vereins.
	\item Die Mitglieder erhalten keine Gewinnanteile. Es darf keine Person
		durch Ausgaben, die dem Zweck des Vereins fremd sind, oder durch
		unverhältnismäßig hohe Vergütungen begünstigt werden. Bei
		Ausscheiden oder Auflösung dürfen Vereinsmitglieder keine Anteile
		des Vereinsvermögens erhalten.
\end{enumerate}

\section{Zweck des Vereines und Zweckverwirklichung}
\begin{enumerate}
	\item Grundlegender Zweck des Vereins ist die Förderung
		\begin{itemize}
			\item der Erziehung, der Volks- und der Berufsbildung einschließlich
				  der Studentenhilfe
			\item der Forschung, der Kunst und der Kultur
			\item des Umgangs mit Technologie sowie
			\item der öffentlichen Auseinandersetzung
		\end{itemize}
		auf den Themengebieten
		\begin{itemize}
			\item der Informationstechnologie
			\item der Computersicherheit und
			\item des Datenschutzes
		\end{itemize}

	\item Das Handeln des Vereins ist durch die Gedanken
		\begin{itemize}
			\item der Gleichberechtigung
			\item des internationalen Austauschs und
			\item der Mitwirkung an der pluralistischen, demokratischen
				Gesellschaft
		\end{itemize}
		bestimmt.

	\item Das primäre Mittel zur Verwirklichung des Vereinszwecks sind
		Aufbau und Betrieb einer Begegnungsstätte, die eine räumliche
		Grundlage für Aktivitäten im Sinne dieser Satzung bildet.

	\item Die sekundären Mittel zur Verwirklichung des Vereinszwecks sind
		schwerpunktmäßig:

	\begin{itemize}
		\item Veranstaltung von öffentlichen Vorträgen, Seminaren, Tagungen und
			anderen Informationsveranstaltungen zu den Themengebieten
			des Vereins

		\item gemeinschaftliche, kritische Rezeption von Medieninhalten
			wie Dokumentationen, Vortragsmitschnitte oder Artikel, die der
			Bildung auf den Themengebieten des Vereins dienen

		\item Ausstellung technischer Geräte von historischem oder aktuellem
			Interesse

		\item Durchführung von Projekten zur Förderung, Bildung und Erziehung
			der Jugend in Themenbereichen des Vereins wie etwa
			\begin{itemize}
				\item der angeleiteten Entwicklung von Soft- und Hardwarekomponenten,
				\item dedizierter Bildungsveranstaltungen oder
	    		\item Kooperationen mit Schulen
	    	\end{itemize}
		\item Vernetzung mit lokalen und internationalen Organisationen und
			Gruppen im Themenspektrum des Vereins durch
			\begin{itemize}
				\item Organisation von Austauschfahrten,
				\item gemeinsame Vorträge und Tagungen,
				\item Betrieb und gemeinsame Nutzung von
					Kommunikationsinfrastruktur und
				\item Kooperation mit User-Groups und Nutzerstammtischen
			\end{itemize}

		\item Förderung von Aktivitäten zu interdisziplinären Aspekten der
			Vereinsthemen

		\item Bereitstellung der physischen und elektronischen Infrastruktur zur
			Durchführung von Projekten im Sinne des Satzungszwecks,
			insbesondere die Einrichtung eines Hardwarelabors

		\item Einbindung künstlerischer Arbeiten zum und im Bereich Computer,
			Technik, neue Medien in das Vereinsleben, insbesondere durch
			Ausstellung und Vorführung künstlerischer Arbeiten in den
			Vereinsräumen sowie die Integration kreativer Elemente in
			deren Einrichtung.
	\end{itemize}
\end{enumerate}
\section{Mitgliedschaft}
\begin{enumerate}
	\item Ordentliches Mitglied des Vereins kann jede natürliche Person werden,
		die seine Ziele unterstützt. Fördermitglied kann jede natürliche oder
		juristische Person werden.

	\item Über die Aufnahme von Mitgliedern entscheidet der Vorstand.

	\item Die Beitrittserklärung erfolgt schriftlich gemäß §11 gegenüber
		dem Vorstand. Die Mitgliedschaft beginnt mit der Aushändigung einer
		entsprechenden Bestätigung durch ein Vorstandsmitglied.

	\item Hat der Vorstand die Aufnahme abgelehnt, so kann der
		Mitgliedschaftsbewerber Einspruch zur nächsten Mitgliederversammlung
		einlegen, die daraufhin abschließend über die Aufnahme oder Nichtaufnahme
		entscheidet.

	\item Die Mitgliedschaft endet durch Austrittserklärung, durch
		Ausschluss, durch Tod von natürlichen Personen oder durch Auflösung
		und Erlöschung von nicht natürlichen Personen. Die Beitragspflicht
		für das laufende Geschäftsjahr wird von der Geschäftsordnung geregelt.

	\item Der Austritt wird durch eine gemäß §11 schriftliche Willenserklärung
		gegenüber dem Vorstand erklärt.
\end{enumerate}

\section{Ausschluss eines Mitglieds}
\begin{enumerate}
	\item Ein Mitglied kann durch Beschluss des Vorstandes ausgeschlossen
		werden, wenn es das Ansehen des Vereins schädigt, es den satzungsgemäßen
		Zielen des Vereins entgegenwirkt oder seinen Beitragsverpflichtungen
		nicht nachkommt. Der Vorstand muss dem auszuschließenden Mitglied den
		Beschluss in schriftlicher Form gemäß §11 unter Angabe von Gründen
		mitteilen und ihm auf Verlangen eine Anhörung gewähren.

	\item Gegen den Beschluss des Vorstandes ist die Anrufung der
		Mitgliederversammlung zulässig. Bis zum Beschluss der
		Mitgliederversammlung ruht die Mitgliedschaft.
\end{enumerate}

\section{Rechte und Pflichten der Mitglieder}

\begin{enumerate}
	\item Die ordentlichen Mitglieder sind berechtigt, die Leistungen des
		Vereins entsprechend der vorhandenen Möglichkeiten und in angemessenem
		und verhältnismäßigem Ausmaß in Anspruch zu nehmen.

	\item Die Mitglieder sind verpflichtet, die satzungsgemäßen Zwecke des
		Vereins zu unterstützen und zu fördern.

	\item Der Verein erhebt einen Mitgliedsbeitrag, zu dessen Zahlung die
		Mitglieder verpflichtet sind. Näheres regelt eine Geschäftsordnung,
		die von der Mitgliederversammlung beschlossen wird.

\end{enumerate}

\section{Organe des Vereins}

\begin{enumerate}
	\item Die Organe des Vereins sind:
		\begin{itemize}
			\item Die Mitgliederversammlung
			\item Der Vorstand
		\end{itemize}
\end{enumerate}

\section{Mitgliederversammlung}

\begin{enumerate}
	\item Die Mitgliederversammlung ist das oberste Beschlussorgan des
		Vereins. Ihr obliegen alle Entscheidungen, die nicht durch die
		Satzung oder die Geschäftsordnung einem anderen Organ übertragen
		wurden.

	\item Beschlüsse werden von der Mitgliederversammlung durch öffentliche
		Abstimmung getroffen. Auf Wunsch eines ordentlichen Mitglieds ist geheim
		abzustimmen.

	\item Jedes ordentliche Mitglied hat genau eine Stimme.

	\item Zur Fassung eines Beschlusses ist eine einfache Mehrheit der
		abgegebenen Stimmen notwendig. Ausgenommen sind die in §9 und §10
		geregelten Angelegenheiten. Eine zur Herstellung der
		Beschlussfähigkeit nötige Untergrenze von abgegebenen Stimmen wird
		in der Geschäftsordnung festgelegt.

	\item Eine ordentliche Mitgliederversammlung, bezeichnet als
		Jahreshauptversammlung, wird einmal jährlich einberufen. Ihre
		Tagesordnung umfasst unter anderem den Rechenschaftsbericht des
		Vorstands über die Vereinstätigkeit sowie den Rechenschaftsbericht des
		Schatzmeisters für das vorherige Geschäftsjahr.

	\item Eine außerordentliche Mitgliederversammlung kann jederzeit
		einberufen werden, wenn mindestens 23\% der ordentlichen Mitglieder
		oder der Vorstand dies jeweils schriftlich gemäß §11 unter Angabe
		eines Grunds beantragen. Dem angegebenen Grund müssen die gewünschten
		Tagesordnungspunkte zu entnehmen sein; sie werden auf die Einladung
		übernommen.

	\item Dem Vorstand obliegt zu allen Mitgliederversammlungen die
		Festsetzung eines Termins und die rechtzeitige Einladung aller
		Mitglieder bis spätestens zwei Wochen vor dem von ihm festgesetzten
		Termin. Bei von den Mitgliedern beantragten Mitgliederversammlungen
		darf der Termin nicht mehr als acht Wochen nach dem Eingang des Antrags
		beim Vorstand liegen.

	\item Der Vorstand kann die Einladungen auf schriftlichem Weg gemäß §12
		zustellen, muss jedoch eine Kopie auf dem Postweg zustellen, falls
		das Mitglied den Wunsch dazu schriftlich gemäß §12 angemeldet hat.

	\item In der Einladung werden die Tagesordnungspunkte sowie weitere
		nötige Informationen bekannt gegeben. Die Mitgliederversammlung kann
		per Beschluss die Tagesordnung verändern.

	\item Über die Beschlüsse der Mitgliederversammlung ist ein Protokoll
		anzufertigen, das vom Versammlungsleiter und vom Schriftführer zu
		unterzeichnen ist. Das Protokoll ist innerhalb von 14 Tagen allen
		Mitgliedern zugänglich zu machen und auf der nächsten
		Mitgliederversammlung genehmigen zu lassen.

	\item Der Vorstandsvorsitzende ist Versammlungsleiter der
		Mitgliederversammlung. Die Mitgliederversammlung kann durch
		Beschluss einen anderen Versammlungsleiter oder Schriftführer
		bestimmen.

\end{enumerate}

\section{Vorstand}

\begin{enumerate}
	\item Der Vorstand besteht aus mindestens drei ordentlichen
		Mitgliedern: dem Vorstandsvorsitzenden, dem Schatzmeister und dem
		Schriftführer. Des Weiteren können bis zu drei Beisitzer in den
		Vorstand gewählt werden. Es kann auf Wunsch der
		Mitgliederversammlung auf eine Wahl der Beisitzer verzichtet werden.

	\item Vorstand im Sinne des §26 BGB sind der Vorstandsvorsitzender,
		Schatzmeister sowie der Schriftführer. Diese sind einzeln
		berechtigt, den Verein nach außen zu vertreten. Die Geschäftsordnung
		kann hierfür Einschränkungen festlegen.

	\item Der Schatzmeister überwacht die Haushaltsführung und verwaltet
		das Vermögen des Vereins. Näheres regelt die Geschäftsordnung.

	\item Vorstandsmitglieder können jederzeit von ihrem Amt zurücktreten.

	\item Bei Rücktritt oder andauernder Ausübungsunfähigkeit eines
		Vorstandsmitglieds ist der gesamte Vorstand neu zu wählen. Bis zur
		Wahl eines neuen Vorstands ist der bisherige Vorstand zur
		bestmöglichen Wahrnehmung seiner Aufgaben verpflichtet.

	\item Die Amtsdauer der Vorstandsmitglieder beträgt ein Jahr. Sie werden
		von der Mitgliederversammlung aus den ordentlichen Mitgliedern des
		Vereins gewählt. Es werden nacheinander Vorstandsvorsitzender,
		Schatzmeister und Schriftführer sowie falls gewünscht bis zu drei
		Beisitzer gewählt. Eine Wiederwahl ist beliebig oft zulässig.

	\item Der Vorstand ist Dienstvorgesetzter aller vom Verein angestellten
		Mitarbeiter. Er kann diese Aufgabe einem Vorstandsmitglied übertragen.

	\item Die Vorstandsmitglieder sind grundsätzlich ehrenamtlich tätig.
		Sie haben Anspruch auf Erstattung notwendiger Auslagen, deren Rahmen
		von der Geschäftsordnung festgelegt wird.

	\item Der Vorstand tritt nach Bedarf zusammen. Die Vorstandssitzungen
		werden vom Schriftführer schriftlich gemäß §11 einberufen. Der Vorstand
		ist beschlussfähig, wenn mindestens zwei Drittel der Vorstandsmitglieder
		anwesend sind. Die Beschlüsse der Vorstandssitzung sind schriftlich zu
		protokollieren.

	\item Jedes Vorstandsmitglied hat bei Abstimmungen des Vorstands eine
		Stimme. Bei Abstimmungen ist eine Mehrheit von zwei Dritteln der
		abgegebenen gültigen Stimmen nötig.
\end{enumerate}

\section{Satzungs- und Geschäftsordnungsänderung}

\begin{enumerate}
	\item Über Satzungs- und Geschäftsordnungsänderungen kann in der
		Mitgliederversammlung nur abgestimmt werden, wenn auf diesen
		Tagesordnungspunkt hingewiesen wurde und der Einladung sowohl der
		bisherige als auch der vorgesehene neue Text beigefügt worden war.

	\item Für die Satzungs- oder Geschäftsordnungsänderung ist eine
		Mehrheit von zwei Dritteln in der Mitgliederversammlung erforderlich.

	\item Satzungsänderungen, die von Aufsichts-, Gerichts- oder
		Finanzbehörden aus formalen Gründen verlangt werden, kann der
		Vorstand von sich aus vornehmen. Diese Satzungsänderungen müssen der
		nächsten Mitgliederversammlung mitgeteilt werden.
\end{enumerate}

\section{Auflösung des Vereins und Vermögensbindung}
\begin{enumerate}

	\item Die Auflösung des Vereins muss von der Mitgliederversammlung mit
		einer Mehrheit von drei Vierteln beschlossen werden. Die Abstimmung
		ist nur möglich, wenn auf der Einladung zur Mitgliederversammlung
		als einziger Tagesordnungspunkt die Auflösung des Vereins
		angekündigt wurde.

	\item Bei Auflösung des Vereins, Aufhebung der Körperschaft oder
		Wegfall der gemeinnützigen Zwecke darf das Vermögen der
		Körperschaft nur für steuerbegünstigte Zwecke verwendet werden. Zur
		Erfüllung dieser Voraussetzung wird das Vermögen einer anderen
		steuerbegünstigten Körperschaft oder einer Körperschaft öffentlichen
		Rechts für steuerbegünstigte Zwecke übertragen, die ebenfalls den
		Auftrag zur Bildung und Volksbildung im Umgang mit
		Informationstechnologie wahrnimmt. Näheres kann die Geschäftsordnung
		regeln.

	\item Der Grundsatz der Vermögensbindung ist bei der Fassung von
		Beschlüssen über die künftige Verwendung des Vereinsvermögens
		zwingend zu erfüllen.

	\item Bei Verlust der Anerkennung als gemeinnütziger Verein gelten die
		vorgenannten Absätze analog. Das Vermögen und die Güter des Vereins
		werden entsprechend übertragen.
\end{enumerate}

\section{Schriftform, Abstimmungsfähigkeit}

\begin{enumerate}
	\item Schriftliche Erklärungen im Sinne dieser Satzung können auch
		elektronische Dokumente sein. Die Geschäftsordnung bestimmt
		Anforderungen, Zustellwege und Zuordnung derartiger Dokumente.
\end{enumerate}

\end{document}

\documentclass[fontsize=12pt,paper=a4,pagesize]{scrartcl}
\usepackage{lmodern}
\usepackage[T1]{fontenc}
\usepackage[utf8]{inputenc}
\usepackage[ngerman]{babel}
\usepackage{eurosym}
\title{Geschäftsordnung des \\ Hackspace Jena e.V.}
\date{Fassung vom 29. Januar 2012}
\renewcommand*{\othersectionlevelsformat}[3]{\S\,#3\autodot\enskip}
\begin{document}

\maketitle{}

\section{Mitgliedsbeiträge}

\begin{enumerate}
    \item Der Verein erhebt gemäß §5 seiner Satzung Mitgliedsbeiträge wie
        folgt:
        \begin{itemize}
            \item 8\euro{}/Monat für ermäßigte Mitgliedschaft
            \item 16\euro{}/Monat für normale Mitgliedschaft
            \item 32\euro{}/Monat, 42\euro{}/Monat, 64\euro{}/Monat,
                  128\euro{}/Monat oder 256\euro{}/Monat für
                  Fördermitglieder
        \end{itemize}
        Der Mitgliedsbeitrag wird für mindestens einen Monat im Voraus
        entrichtet. Einmal gezahlte Mitgliedsbeiträge werden nicht
        zurückerstattet.

    \item Voraussetzung für die ermäßigte Mitgliedschaft ist die Vorlage
        eines gültigen Schüler-, Studenten-, Renten- oder
        Behindertenausweises, einer Arbeitslosigkeitsbescheinigung oder
        eines vergleichbaren Nachweises gegenüber dem Vorstand. Der
        entsprechende Nachweis ist jährlich neu zu erbringen.

    \item Jedem Mitglied steht es frei, den Verein durch einen höheren
        Mitgliedsbeitrag stärker finanziell zu unterstützen. Damit sind
        keinerlei Privilegien oder Stimmvorteile verbunden.

    \item Änderungen bezüglich Mitgliedsbeitrag oder Mitgliedsart sind dem
        Vorstand schriftlich mitzuteilen und gelten mit sofortiger Wirkung.
\end{enumerate}

\section{Beschlussfähigkeit der Mitgliederversammlung}

\begin{enumerate}
    \item Die Untergrenze für die Beschlussfähigkeit der
        Mitgliederversammlung gemäß Satzung §7 beträgt 23\% der ordentlichen
        Mitglieder.
\end{enumerate}

\section{Einschränkungen der Verfügungsberechtigung des Vorstands}
\begin{enumerate}
    \item Einzelne Vorstandsmitglieder sind bei Rechtsgeschäften bis zu
        einem Betrag von 400\euro{} verfügungsberechtigt. Über einen Betrag
        von bis zu 2000\euro{} können zwei Vorstandsmitglieder gemeinsam
        verfügen. Bei höheren Beträgen ist ein Beschluss durch die
        Mitgliederversammlung nötig.
\end{enumerate}

\section{Grundsätze der Vermögensverwaltung des Vereins}

\begin{enumerate}
    \item Die Summe der Ausgaben eines Jahres darf das liquide
            Vereinsvermögen nicht übersteigen.
\end{enumerate}

\section{Aufgaben des Schatzmeisters}

\begin{enumerate}
    \item Der Schatzmeister hat auf eine sparsame und wirtschaftliche
        Haushaltsführung hinzuwirken.

    \item Der Schatzmeister legt nach Eintragung des Vereins in das
        Vereinsregister ein Konto auf den Namen des Vereins an und
        verwaltet dort das Vereinsvermögen.

    \item Für Abhebungen vom Vereinskonto ist die Unterschrift von zwei
        Vorstandsmitgliedern nötig.

    \item Der Schatzmeister informiert die Vereinsmitglieder
        mindestensvierteljährlich sowie innerhalb von sechs Wochen
        nach größeren Veranstaltungen, bei denen der Verein als
        Veranstalter oder Mitveranstalter auftritt, über den
        Kassenstand. Einnahmen und Ausgaben über 100\euro{} sind dabei
        einzeln aufzulisten.

    \item Als Vorstandsmitglied hat der Schatzmeister die Einbringung der
        Mitgliedsbeiträge und anderer Einnahmen zu organisieren. Dabei
        genießt er die volle Unterstützung des Vorstands.

    \item Für laufende Einnahmen und Ausgaben führt der Schatzmeister eine
        Bargeldkasse. Überschüssige Bargeldsummen werden von ihm regelmäßig
        auf dem Vereinskonto abgelegt.

    \item Für Bareingänge stellt der Schatzmeister eine formgerechte
        Quittung in doppelter Ausfertigung aus, davon eine für den Einzahler.

    \item Der Schatzmeister legt ein geeignetes Vermögensregister an, das
        nach den Regeln der einfachen Buchführung zu führen ist und aus
        folgenden Teilen besteht:
        \begin{itemize}
            \item Kassenbuch für die Bargeldkasse
            \item Hauptbuch für das Vereinskonto
            \item Inventarliste für Vermögensgegenstände
        \end{itemize}

    \item Jede einzelne Ausgabe muss belegt werden. Jeder Beleg muss von
        dem Vereinsmitglied, das die Ausgabe getätigt hat, umgehend beim
        Schatzmeister eingereicht werden.

    \item Sollten Güter zugunsten des Vereins eingehen, sind diese im
        Vermögensregister einzutragen. Nach Genehmigung durch den Vorstand hat
        der Schatzmeister ein Aufbewahrungsprotokoll anzufertigen, ein Exemplar
        für den Besorger, eins zur Dokumentation beim Schatzmeister.

    \item Der Schatzmeister führt die Liste der Vereinsmitglieder.
        Periodisch werden von ihm die sich ergebenden Veränderungen durch
        Zugänge und Abgänge den Vereinsmitgliedern mitgeteilt.

    \item Für den Jahresabschluss oder bei Wechsel des Schatzmeisters ist
        durch ihn eine Bilanz zu erstellen.

\end{enumerate}

\section{Erstattung der Auslagen des Vorstands}

\begin{enumerate}
    \item Auslagen des Vorstandes zur Verfolgung der Vereinszwecke werden
        in voller Höhe erstattet. Auf Beschluss der Mitgliederversammlung
        muss der Vorstand in einer Stellungnahme Zweck- und
        Verhältnismäßigkeit der Ausgaben nachweisen.
\end{enumerate}

\section{Elektronische Schriftform}

\begin{enumerate}
    \item Elektronische Dokumente im Sinne von §12 der Satzung sind mit
        PGP/GPG oder mit S/MIME signierte E-Mails. Jedes Mitglied kann beim
        Vorstand einen öffentlichen Schlüssel bzw. sein Zertifikat
        hinterlegen, dessen Signatur die jeweiligen E-Mails tragen müssen.
        Das Mitglied hat bei Kompromittierung des Schlüssels für
        Benachrichtigung des Vorstands zu sorgen.
\end{enumerate}


\section{Sicherheitsbeauftragter}

\begin{enumerate}
    \item Der Vorstand ernennt einen Sicherheitsbeauftragten. Seine
        Aufgaben umfassen insbesondere die Aufklärung und Information der
        Mitglieder zu Sicherheits- und Schutzmaßnahmen, gesetzlichen
        Regelungen und notwendigen Verhaltensweisen zur Vermeidung von
        Unfällen. Weiterhin überprüft er die Einhaltung dieser Regelungen in
        den Räumen des Vereins.
\end{enumerate}

\end{document}

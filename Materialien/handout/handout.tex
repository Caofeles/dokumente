\documentclass[a4paper,12pt,twoside]{scrartcl}
\usepackage[ngerman]{babel}
\usepackage[T1]{fontenc}
\usepackage[utf8]{inputenc}
\usepackage[pdftex,
            pdftitle={Handout: Der Hackspace Jena e.V.},
            pdfsubject={Der Hackspace Jena e.V.},
            pdfkeywords={Krautspace, Hackerspace, Hackspace Jena e.V., Veransataltungen}]{hyperref}
%\usepackage{newtxtext,newtxmath}
\usepackage[top=2.5cm, bottom=3cm, left=2.5cm, right=3cm]{geometry}
%\usepackage{array}
%\usepackage{listings}
%\usepackage{html}
%\usepackage{url} % Kollidiert evtl. mit hyperref
%\usepackage{fancyhdr}
%\usepackage{graphicx}
%\usepackage[usenames,dvipsnames,svgnames,table]{xcolor}
%\usepackage{amsmath}
%\usepackage{amssymb}
%\usepackage{tikz}
%\usetikzlibrary{trees,arrows,decorations,matrix}
% Zeilenabstand zwischen den Items verringern:
\newcommand{\itemlinespace}{\setlength{\itemsep}{1pt}}
\begin{document}
\thispagestyle{empty}
\section*{Der Hackspace Jena e.V.}
\begin{itemize}
\itemlinespace{}
\item https://www.krautspace.de
\item gegründet am 29. Februar 2012 von 20 Jenaer Hackern im Jentower
\item dort ein kleiner Raum von Towerbyte zur Verfügung gestellt
\item Bezug der Vereinsräume in der Krautgasse im August 2012, Name "Krautspace"
\item gemeinnütziger Verein: Offen für jeden, Mitgliedschaft nicht erforderlich, gegenseitige Hilfe und Hilfe zur Selbsthilfe, Bildungsauftrag
\item wir suchen immer nach neuen Mitstreitern und Sponsoren
\end{itemize}
\subsection*{Was ist Hacken? Was sind Hackerspaces}
\begin{itemize}
\itemlinespace{}
\item kreativer Umgang mit Technik, Freude am Basteln, Zusammenhänge verstehen
\item anders als die Hacker, von denen man in den Mainstream-Medien hört
\item keine illegalen Aktivitäten wie oft in Medien verbreitet wird
\item Ein Hackerspace ist ein offener Raum zum Hacken
\end{itemize}
\subsection*{Der Krautspace}
\begin{itemize}
\itemlinespace{}
\item zwei Büroräume in der Krautgasse
\item Elektroniklabor/ Werkstatt (Oszilloskop, Labornetzteil, Lötstationen, 3D-Drucker, Cutplotter)
\item Raum und Geräte für Vorträge und Workshops
\item Bar mit Getränken (mehrere Sorten Mate) und Snacks für's leibliche Wohl
\item Bibliothek
\item kleine Küche
\end{itemize}
\subsection*{Veranstaltungen}
\begin{itemize}
\itemlinespace{}
\item großes Spektrum, nicht nur technisch
\item Technikveranstaltungen: Elektronikgruppe, Linux User Group, Freifunk Jena, Cryptoparties
\item Andere Veranstaltungen: offene Runde, Kochen (Foodhacking), Brettspielrunde,  Furrys
\end{itemize}
\end{document}

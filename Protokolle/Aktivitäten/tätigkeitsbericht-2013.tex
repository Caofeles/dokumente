\documentclass[10pt, a4paper]{scrartcl}
\usepackage[ngerman]{babel}
\usepackage[utf8]{inputenc}
\begin{document}

\section*{Geschäfts- und Tätigkeitsbericht 2013}
Der Verein hat seine Räume weiter ausgerüstet,
es wurde ein Projektor für Vorträge angeschafft und
das Elektroniklabor z.\,B. durch ein Labornetzteil weiter eingerichtet.
Großer Beliebtheit erfreute sich der eingerichtete freie Internetanschluss.

Eine Vorstellung des Vereins und seiner Aktivitäten erfolgte zu Beginn des Wintersemester
zum einen bei der Stadtführung für die Informatik-Erstsemester als auch mittels eines eigenen Standes 
auf dem Markt der Möglichkeiten an der Friedrich-Schiller Universität.
In der Studentenzeitung Akrützel wurde nach einem Interview ein Artikel über den Verein veröffentlicht.

Die regelmäßigen Veranstaltungen konnten in diesem Jahr weiter ausgebaut und ergänzt werden.

\subsection*{Regelmäßige Vereinsaktivitäten}
Folgende Aktivitäten fanden bis auf wenige Ausnahmen wöchentlich statt:
\begin{itemize}
	\item Elektronikrunde
		\begin{itemize}
		    \item Treffen zum Arbeiten an Elektronik- und Technikprojekten
		    \item gegenseitiger Austausch und Hilfestellung
		\end{itemize}
	\item Offene Dienstagsrunde
		\begin{itemize}
			\item Austausch über Computersicherheit, Datenschutz und Informatik
			\item oft spontane Kurzvorträge über aktuelle Forschungsinhalte (Informatik, Physik)
		\end{itemize}
	\item Chaostreff
		\begin{itemize}
			\item Diskussion über Ereignisse aus dem Umfeld des Chaos Computer Clubs
		\end{itemize}
	\item Plenum
		\begin{itemize}
			\item Diskussion aktueller Vereinsthemen
			\item Erarbeitung von Entscheidungshilfen für den Vorstand
		\end{itemize}
	\item Stammtisches der Linux-User-Group Jena
		\begin{itemize}
			\item Erfahrungsaustausch, gegenseitigen Hilfe bei Problemen zum Thema freie Software und insbesondere GNU/Linux
		\end{itemize}
	\item Freifunktreffen
		\begin{itemize}
		    \item Treffen der Jenaer Freifunkgemeinschaft zur Besprechung der aktuellen Entwicklung (Software, Netzausbau)
		    \item Weitertragen des Freifunkkonzepts an Interessierte
		\end{itemize}
	\item Lockpicking
		\begin{itemize}
		    \item Erlernen des Sports Schlösser ohne den passenden Schlüssel aufzusperren
		    \item kritische Auseinandersetzung mit den Sicherungsmechanismen historischer und moderner Schlösser
		\end{itemize}
	\item Gemeinsames Kochen
		\begin{itemize}
			\item Unkompliziertes Zusammentreffen zu einem weniger technischen Thema
			\item Vermittlung von kulinarischen Fähigkeiten und Kreativität in der Küche
		\end{itemize}
	\item Spieleabend
		\begin{itemize}
			\item Spielen von Brett- und Kartenspielen mit unterschiedlichen Spielkonzepten zu einem bestimmten vorher festgelegten Thema
			\item zum Teil Vorstellung selbst entwickelter Spiele
		\end{itemize}
\end{itemize}

\subsection*{Vorträge \& Workshops}
Eine Auswahl von Vorträgen und Workshops:
\begin{itemize}
	\item Workshop: Ikea-GRÖNÖ-Kassetten-Lampen-Hacken (Lampenschirme aus alten Musikkassetten basteln)
	\item Zweite Cryptoparty
	\item Workshop: Linux-Kernel-Hacking
	\item Filmvorführung "`TPB AFK"' und anschließende Diskussion
	\item Workshop: Free your Android Workshop
	\item Vortrag: Linux Blockdeviceveredelung -- Einführung in LVM, DM, MD und deren Kombination
	\item Vortrag: Zeichnen für Hacker
	\item Krautspace Wandertag zur Sternwarte am Forst
	\item Workshop: Compilerbau
	\item Vortrag: Wie baue ich mein eigenes ARM-basierendes Platinenlayout
	\item Lange Nacht der Wissenschaften Jena
		\begin{itemize}
		    \item offizielle Teilnahme des Vereins an der Langen Nacht der Wissenschaften in Jena
		    \item Durchführung von Lötworkshops bei denen LED-Männchen, Alarmanlagen und Licht-Theremine gelötet wurden
		    \item Demonstration einer nicht-newtonschen Flüssigkeit
		    \item Vortrag zu Anonymität im Internet
		    \item Demonstration exotischer Rechnerarchitekturen
		\end{itemize}
\end{itemize}

\subsection*{LPI-Lerngruppe}
Veranstaltungsreihe zur gemeinsamen Vorbereitung zur Prüfungen für das LPI Level 1 Zertifikat des LPIC. 

\end{document}

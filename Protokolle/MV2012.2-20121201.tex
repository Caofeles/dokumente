\documentclass{scrartcl}
\usepackage[utf8]{inputenc}
\usepackage[T1]{fontenc}
\usepackage[ngerman]{babel}
\title{Protokoll der Mitgliederversammlung 2012.2}
\author{Hackspace Jena e.\,V.}
\date{01.\,12.\,2012}
\begin{document}
\maketitle{}


\begin{itemize}
\item Anwesende: Zu Beginn 11 Vereinsmitglieder, ein Gast
\item Protokollant: Tim Schumacher
\item Versammlungsleitung\footnote{Wechselnd auf Grund von spontaner
Kandidatur für Ämter}: Felix Kästner, Adrian Pauli, Frank Lanitz
\item Ort: Krautspace, Krautgasse 26 in Jena
\item Zeit: 01.12.2012 von 15:15 bis 18:15
\end{itemize}

\section{Begrüßung}
Kassenwart Frank Lanitz und Felix Kästner begrüßen die Anwesenden. Frank
Lanitz stellt fest, dass zu Beginn der Versammlung 11 ordentliche
Mitglieder und 1 Gast anwesend sind. Die Versammlung ist beschlussfähig,
da mindestens 7 Mitglieder anwesend sein müssen.

\section{Wahl des Versammlungsleiters}
Felix Kästner wird in einer offenen Wahl durch Handzeichen mit 10
Stimmen und einer Gegenstimme zum Versammlungsleiter gewählt.

\section{Wahl des Protokollführers}
Einstimmig wird Tim Schumacher in einer offenen Wahl durch Handzeichen
zum Protokollführer gewählt.

\section{Genehmigung der Tagesordnung}

Es werden zwei Punkte zur Tagesordnung hinzugefügt:
\begin{enumerate}
	\item Die Abstimmung des Protokolls der Gründungsversammlung und
	\item der Ausblick auf den Haushalt 2013, welcher innerhalb des
Berichtes des Vorstandes abgehandelt wird
\end{enumerate}
Die Tagesordnung wird einstimming durch offene Wahl mit Handzeichen
angenommen.

Frank berichtet, dass die Einladung für die Versammlung 14 Tage im
Voraus -- am 17.11.2012 gegen 10:30 -- an alle Mitlgieder per Email
sowie als Kopie an die öffentliche Mailingliste versendet wurde. Es gab
eine unzustellbare Mail, für die Frank die Einladung fernmündlich mit
Verweis auf die öffentliche Maillingliste ausgesprochen hat. Das
betreffende Mitglied war mit der Information zufrieden und bei der
Versammlung anwesend. Die Korrektheit der Einladung wurde von der
Versammlung einstimmig durch Handzeichen bestätigt.

\section{Bestimmung der Kassenprüfer}

Die Kassenprüfer Felix Kästner und Martin Neß werden für die Periode von
Januar bis November 2012 durch Handzeichen einstimmig als Kassenprüfer
angenommen.

\section{Rechenschaftsbericht des Vorstandes inkl. Schatzmeister}
Der Bericht des Vorstandes befindet sich im Anhang zu diesem Protokoll.
Auf der Versammlung wird dieser vorgestellt und auf Fragen dazu
eingegangen.

Rückfragen und Diskussion gibt es vor allem zur Thematik der Türöffnung
und der vorausgegangen Installation des elektronischen Schlosses.
Die Thematik wird auf das nächste Plenum vertagt.
Eine Haftungsübertragung auf das ausführende Mitglied wurde von der
Mehrheit per Meinungsbild abgelehnt.

\section{Abstimmung Protokoll Gründungsversammlung}
Das Protokoll wird in einer offenen Wahl durch Handzeichen mit 9 Stimmen
und 2 Enthaltungen angenommen.

\section{Bericht der Kassenprüfer}
Martin Neß und Felix Kästner stellen das Ergebnis der Kassenprüfung vor.
Die Kassenprüfer können die Abrechnung nachvollziehen. Mit einer
Ausnahme waren alle Belege für die vergangene Periode vollständig. Eine
zum Zeitpunkt der Prüfung fehlende Rechnung bzgl. des Hostings des
Webauftrittes wurde durch den Kassenwart nachgereicht.

Die Kassenprüfung war erfolgreich.

In einer offenen Abstimmung durch Handzeichen wurde die Kassenprüfung
mit 10 Stimmen und einer Enthaltung angenommen.

\section{Entlastung des Vorstandes}
In einer offenen Wahl wird der Vorstand mit 10 Stimmen und einer
Enthaltung entlastet.

Um 17:20 trifft ein weiteres stimmberechtigtes Mitglied zur
Veranstaltung ein. Damit sind nun 12 ordentliche Mitglieder und 1 Gast
anwesend.

\section{Wahl des Vorstandes und der Kassenprüfer}

Ein Meinungsbild ergab, dass keine Beisitzer gewählt werden sollen. Es
werden also nur Vorsitzender, Schriftführer und Kassenwart gewählt.

\subsection{Vorsitzender}
Zur Wahl stellen sich Konrad Schöbel (amtierender Vorstand) und Felix
Kästner. Da Felix Kandidat für die Wahl ist, gibt er die
Versammlungsleitung an Adrian Pauli ab. Das Plenum stimmt dem Wechsel
einstimmig zu.

Es wird in geheimer Personenwahl abgestimmt.

\begin{table}[h!]
    \centering
    \begin{tabular}{c|c|c|c}
        \textbf{Konrad} & \textbf{Felix} & \textbf{Enthaltungen} &
\textbf{Ungültig} \\ \hline
        5 & 3 & 4 & 0 \\
    \end{tabular}
\end{table}

Konrad nimmt die Wahl zum Vorsitzenden an.

\subsection{Schriftführer}
Zur Wahl stellen sich Jan Huwald (amtierender Schriftführer) und Felix
Kästner. Es wird in geheimer Personenwahl abgestimmt.

\subsubsection*{1.\,Wahlgang}
\begin{table}[h!]
    \centering
    \begin{tabular}{c|c|c|c}
        \textbf{Jan} & \textbf{Felix} & \textbf{Enthaltungen} &
\textbf{Ungültig} \\ \hline
        5 & 5 & 2 & 0 \\
    \end{tabular}
\end{table}

Damit konnte keiner der Bewerber die Mehrheit der Stimmen auf sich
vereinen, so dass ein zweiter Wahlgang notwendig wurde.

\subsubsection*{2.\,Wahlgang}
    \begin{table}[h!]
    \centering
    \begin{tabular}{c|c|c|c}
        \textbf{Jan} & \textbf{Felix} & \textbf{Enthaltungen} &
\textbf{Ungültig} \\ \hline
        5 & 6 & 1 & 0 \\
    \end{tabular}
\end{table}

Felix Kästner hat die Mehrheit der Stimmen auf sich vereint. Er nimmt
die Wahl an.

\subsection{Kassenwart}

Zur Wahl stellt sich Frank Lanitz. Es wird offen durch Handzeichen gewählt.

\begin{table}[h!]
    \centering
    \begin{tabular}{c|c|c}
        \textbf{Ja} & \textbf{Nein} & \textbf{Enthaltungen} \\ \hline
        10 & 0 & 2
    \end{tabular}
\end{table}

Frank Lanitz nimmt die Wahl an.

\subsection{Kassenprüfer}

Martin Neß und Adrian Pauli kandidieren für den Posten der 
Kassenprüfer. Frank Lanitz übernimmt die Versammlungsleitung von 
Adrian Pauli (einstimmige Zustimmung der Versammlung). Es wird 
öffentlich und im Block durch Handzeichen abgestimmt.

\begin{table}[h!]
    \centering
    \begin{tabular}{c|c|c}
        \textbf{Ja}& \textbf{Nein} & \textbf{Enthaltungen} \\ \hline
        7 & 0 & 5
    \end{tabular}
\end{table}

Der Vorstand ist damit komplett. Die Stammdaten befinden sich im Anhang unter \ref{sec:neuer_vorstand}.
Ein Mitglied verlässt die Veranstaltung. Damit sind noch 11 Mitglieder anwesend.
%   - Es muss geprüft werden, ob die Kassenprüfer wiedergewählt werden
% dürfen?

\section{Satzungsänderungsanträge}
Frank Lanitz stellt die beiden vorliegenden Satzungsänderungsanträge
vor.

\subsection{Änderung des Vereinszweck in Vorbereitung der
Gemeinnützigkeit (§3 der aktuellen Satzung)}

Es gibt keine weiteren Fragen zum Änderungsantrag. Es wird in offener
Wahl durch Handzeichen abgestimmt:

\begin{table}[h!]
    \centering
    \begin{tabular}{c|c|c}
        \textbf{Ja} & \textbf{Nein} & \textbf{Enthaltung} \\ \hline
        10 & 0 & 1
    \end{tabular}
\end{table}

Die Satzungsänderung wurde angenommen. §3 lautet nun wie im Anhang unter
\ref{sec:new_para_3} abgedruckt.

\subsection{Änderung §1(1) auf Grund der erfolgreichen Eintragung ins
Vereinsregister}

Es wird festgestellt, dass die Änderung eine rein formale Angelegenheit
ist. Keine Rückfragen zum Antrag. Es wird in offener Wahl durch
Handzeichen abgestimmt:

Die Satzungsänderung wurde angenommen. §1 lautet nun wie im Anhang unter
\ref{sec:new_para_1} abgedruckt.

\begin{table}[h!]
    \centering
    \begin{tabular}{c|c|c}
        \textbf{Ja} & \textbf{Nein} & \textbf{Enthaltung} \\ \hline
        12 & 0 & 0
    \end{tabular}
\end{table}

\section{Allgemeines}

\subsection{Zutritt zum Raum}
Die Versammlung spricht sich dafür aus, dass jedes Mitglied auch
weiterhin Zutritt zu den Vereinsräumen haben kann, so lang kein
triftiger Grund dem entgegen steht.
Gäste erhalten Zutritt im Sinne der Satzung und in Begleitung eines
Mitgliedes.
Eine Ausnahme bilden »Freunde des Vereins«, die auch ohne Mitglied den
Raum betreten dürfen.
Eine Entscheidung darüber trifft der Vorstand. \\[1ex]

Die Versammlung wurde um 18:15 aufgelöst.

\newpage 
\appendix{}
\section{Anwesenheitsliste}
% Amtsgericht möchte das haben. 
% Ich weiß, dass ist Sackgang. Aber habe nichts gefunden, was wir dagegen 
% machen können. 
% :( frlan
\newpage 
\section{Angenommene Satzungsänderungen}
\subsection{Neuer §1 der Satzung wie auf der Mitgliederversammlung beschlossen}
Der Artikel 1 lautet nun wie folgt: 

\begin{quote}
\label{sec:new_para_1}

§1 Name, Sitz, Geschäftsjahr

	\begin{enumerate}
		\item Der Verein trägt den Namen "`Hackspace Jena e.\,V."' und ist in das
			Vereinsregister beim Amtsgericht Jena eingetragen.
		\item Der Verein hat seinen Sitz in Jena. Das Geschäftsjahr entspricht
			dem Kalenderjahr.
	\end{enumerate}
\end{quote}

\subsection{Neuer §3 der Satzung wie auf der Mitgliederversammlung beschlossen} 
\label{sec:new_para_3}
Der Artikel 3 der Satzung lautet nun:

\begin{quote}
	
	§3 Zweck des Vereines und Zweckverwirklichung
\begin{enumerate}
	\item Grundlegender Zweck des Vereins ist die Förderung
		\begin{itemize}
			\item der Erziehung, der Volks- und der Berufsbildung einschließlich
				  der Studentenhilfe
			\item der Kunst und der Kultur
		\end{itemize}
		auf den Themengebieten
		\begin{itemize}
			\item der Informationstechnologie
			\item der Computersicherheit und
			\item des Datenschutzes
		\end{itemize}

	\item Das Handeln des Vereins ist durch die Gedanken
		\begin{itemize}
			\item der Gleichberechtigung
			\item des internationalen Austauschs und
			\item der Mitwirkung an der pluralistischen, demokratischen
				Gesellschaft
		\end{itemize}
		bestimmt.

	\item Das primäre Mittel zur Verwirklichung des Vereinszwecks sind
		Aufbau und Betrieb einer Begegnungsstätte, die eine räumliche
		Grundlage für Aktivitäten im Sinne dieser Satzung bildet.

	\item Die sekundären Mittel zur Verwirklichung des Vereinszwecks sind
		schwerpunktmäßig

	\begin{itemize}
		\item Veranstaltung von öffentlichen Vorträgen, Seminaren, Tagungen und
			anderen Informationsveranstaltungen zu den Themengebieten
			des Vereins

		\item gemeinschaftliche, kritische Rezeption von Medieninhalten
			wie Dokumentationen, Vortragsmitschnitten oder Artikeln, die der
			Bildung auf den Themengebieten des Vereins dienen

		\item Ausstellung technischer Geräte von historischem oder aktuellem
			Interesse

		\item Durchführung von Projekten zur Förderung, Bildung und Erziehung
			der Jugend in Themenbereichen des Vereins wie etwa
			\begin{itemize}
				\item der angeleiteten Entwicklung von Soft- und Hardwarekomponenten,
				\item dedizierter Bildungsveranstaltungen oder
				\item Kooperationen mit Schulen.
		\end{itemize}

		\item Vernetzung mit lokalen und internationalen Organisationen und
			Gruppen im Themenspektrum des Vereins durch
			\begin{itemize}
				\item Organisation von Austauschfahrten,
				\item gemeinsame Vorträge und Tagungen,
				\item Betrieb und gemeinsame Nutzung von
					Kommunikationsinfrastruktur und
				\item Kooperation mit User-Groups und Nutzerstammtischen.
			\end{itemize}

		\item Bereitstellung der physischen und elektronischen Infrastruktur zur
			Durchführung von Projekten im Sinne des Satzungszwecks,
			insbesondere die Einrichtung eines Hardwarelabors

		\item Einbindung künstlerischer Arbeiten zum und im Bereich Computer,
			Technik, neue Medien in das Vereinsleben, insbesondere durch
			Ausstellung und Vorführung künstlerischer Arbeiten in den
			Vereinsräumen sowie die Integration kreativer Elemente in
			deren Einrichtung.
	\end{itemize}
\end{enumerate}
\newpage 
\section{Neuer Vorstand}

Der am 1.12.2012 gewählte Vorstand setzt sich wie folgt zusammen: 

\label{sec:neuer_vorstand}
\begin{table}[h!]
	\centering
	\begin{tabular}{l|l|c|l}
		%Adressdaten stehen da mal nicht drin. Reicht, wenn sie im 
		%Vereinsregister stehen.
		\textbf{Name} & \textbf{Adresse} & \textbf{Geburstag} & \textbf{Amt} \\ \hline
		Konrad Schöbel & & & Vorsitzender \\
		Felix Kästner & & & Schriftführer \\
		Frank Lanitz & & & Schatzmeister 
	\end{tabular}
\end{table}
\end{quote}
\end{document}

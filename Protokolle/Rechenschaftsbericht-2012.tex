\documentclass[10pt,DIV16]{scrartcl}

\title{Rechenschaftsbericht}
\subtitle{Hackspace Jena e.\,V.}
\author{%
	Konrad Schöbel (Vorsitzender)\\
	Jan Huwald (Stellvertreter)\\
	Frank Lanitz (Schatzmeister)
}
\date{29. Januar -- 1. Dezember 2012}

\usepackage[ngerman]{babel}
\usepackage{enumitem}
\usepackage[utf8]{inputenc}
\usepackage{marvosym}

\setitemize{itemsep=0pt}

\begin{document}

\maketitle

\tableofcontents

Neben den bürokratischen und administrativen Tätigkeiten, die die
Vereinsgründung und -führung mit sich bringen, haben sich die Aktivitäten des
Vorstandes entsprechend der Satzung vor allem auf zwei Schwerpunkte
konzentriert:  Die Suche und Einrichtung geeigneter Vereinsräume sowie die
Organisation und Durchführung von Veranstaltungen.


%v------------------------------------------------------------------------------%
\section{Vereinsgründung}

\paragraph{Vereinsgründung und Eintragung in das Vereinsregister}

Der Verein wurde am 29. Januar 2012 gegründet und am 27. Februar entsprechend
§1 (Name, Sitz, Geschäftsjahr) Punkt 1 der Satzung in das Vereinsregister
eingetragen.

\paragraph{Vereinskonto und Postfach}

Entspechend §5 (Aufgaben des Schatzmeisters) Punkt 2 der Geschäftsordnung hat
der Schatzmeister ein Vereinskonto bei der Ethikbank Eisenach eingerichtet.
Kriterien für die Auswahl des Kreditinstituts waren [Kriterien].

Anfang März hat Frank ein Postfach für den Verein eingerichtet, das von ihm
regelmäßig geleert wird. 


\paragraph{Gemeinnützigkeit}

Der Verein strebt die Gemeinnützigkeit an.  Dazu haben wir den Satzungsentwurf
dem Finanzamt Jena vorgelegt.  Diese hat ergeben, dass der Verein mit der
gegenwärtigen Satzung sie Gemeinnützigkeit nicht erlangen würde.  Als Gründe
werden hierfür angeführt, dass der Vereinszweck nicht den Anforderungen der
Abgabenordnung entspricht und dass das Konzept des Vereins nicht nachprüfbar
ist.  Hierzu heißt es in der Antwort des Finanzamts:

\begin{quote}
	"`Die Satzungszwecke und die Art ihrer Verwirklichung müssen so genau
	bestimmt sein, dass auf Grund der Satzung geprüft werden kann, ob die
	satzungsgemäßen Voraussetzungen für die Steuervergünstigung gegeben sind
	(§60 [Abgabenordnung])."'
\end{quote}
Zu streichen sind deshalb:
\begin{itemize}
	\item
		in §3 Punkt 1, zweiter Anstrich der Satzung die Förderung "`der
		Forschung"', da nur die Förderung von "`Wissenschaft \underline{und}
		Forschung"' gemeinnützig ist.
		\footnote{
			Das Finanzamt definiert \emph{Forschung} als der
			"`\underline{\smash{ernsthafte}} und
			\underline{\smash{planmäßige}} Versuch zur Ermittlung der
			Wahrheit"' sowie \emph{Wissenschaft} als "`Weitergabe von
			Erkenntnissen in nachvollziehbarer, überprüfbarer Form (Lehre)"'.
			\quad m)
		}
	\item
		in §3 Punkt 1 dritter und vierter Anstrich der Satzung die Förderung
		des Umgangs mit Technologie sowie der öffentlichen Auseinandersetzung.
\end{itemize}
In Bezug auf das Konzept des Vereins heißt es
\begin{quote}
	"`Es ist notwendig, dass sich aus der Satzung ein klar erkennbares und
	nachprüfbares Konzept ergibt.  Der Satzungszweck und die Art seiner
	Verwirklichung sollten umso präziser dargelegt werden, wenn ihnen kein
	jedermann bekanntes, begrifflich fest umrissenes gedankliches Konzept
	zugrunde liegt."'
\end{quote}
und weiter
\begin{quote}
	"`Der Verein erfüllt die o.\,g.\ Zwecke u.\,a.\ durch \glq die Förderung von
	Aktivitäten zu interdisziplinären Aspekten der Vereinsthemen\grq.  Was heißt
	das konkret?  Eine bespielhafte Aufzählung der geplanten Aktivitäten in
	der Satzung ist möglich."'
\end{quote}
Zur Beseitigung dieser Kritikpunkte unterbreitet der Vorstand einen Vorschlag
zur Satzungsänderung.

\paragraph{Mitgliederentwicklung}

Die Mitgliederzahl konnte seit Vereinsgründung von 18\footnote{Bei 
der Gründungsversammlung waren 20 Personen anwesend -- 18 sind im 
Rahmen dieser direkt Mitglied geworden} auf 30 gesteigert werden. 
Davon zahlen [Anzahl] Mitglieder einen ermäßigten Mitgliedsbeitrag, 
[Anzahl] Mitglieder den vollen Mitgliedsbeitrag sowie [Anzahl] 
Mitglieder einen erhöhten Beitrag für Fördermitglieder.  Der 
Vorstand erwartet eine weitere Steigerung der Mitgliederzahlen durch 
weitere kontinuierliche Bekanntmachung des Vereins und seiner 
Veranstaltungen.

\paragraph{Namenswahl}

Nach allgemeinem Konsens auf der Mailingliste nennen wir den Hackspace Jena in
Anlehnung an die neue Adresse von nun an auch "`Krautspace"'.  Der Name des
Vereins bleibt hiervon unberührt.

\paragraph{Vereinsversicherung}

Im Oktober hat Jan eine Vereinsversicherung abgeschlossen.  Wir haben uns für
das günstigste Angebot entschieden, welches die Vereinsräume absichert und
Veranstaltungen abdeckt.


%------------------------------------------------------------------------------%
\section{Vereinsräume}

Zur Verwirklichung des in §3 (Zweck des Vereines und Zweckverwirklichung)
Punkt 3 genannten Vereinszwecks war das Hauptziel der Vereinsaktivitäten im
Gründungsjahr zunächst die Suche nach geeigneten Räumlichkeiten für den Aufbau
und Betrieb einer Begegnungsstätte -- des eigentlichen Hackerspaces.  Dies
schien bei der angespannten Mietsituation sowie der so kurz nach der Gründung
noch recht dürftigen finanziellen Situation des Vereins zunächst ein
aussichtsloses Unterfangen.  Deshalb freuen wir uns besonders, dass unsere
Suche schließlich nicht nur von Erfolg gekrönt war, sondern sich unser neues
Domizil auch in bester Innenstadtlage befindet:  gleich hinter der Abbe-Mensa
und über dem Durchgang vom Uni-Campus zur Studentenmeile Wagnergasse.

\subsection{Raumsuche}

\paragraph{Intershop-Turm}

Frank hatte zunächst bei unseren lokalen IT-Mäzenen - der [Firma] sowie der
[Firma] - eine Verlängerung für die Nutzungsvereinbarung des Raumes im
Intershop-Turm bis [Monat] erwirkt.  Wegen des Eigenbedarf besagter Firmen
wären wir danach aber "`obdachlos"' geworden.

\paragraph{Coworking ``Kombinat Süd''}

Da Coworking vom Charakter her gut zu einem Hackspace paßt, waren Frank und
Konrad mehrfach auf Planungstreffen der Coworking-Initiative "`Kombinat Süd"'
und haben dort unser Interesse angemeldet, Räumlichkeiten im Umfeld eines
Coworking-Spaces zu mieten.  Wir sind damit auf offene Ohren gestoßen, da ein
Hackspace viele Gemeinsamkeiten mit einem Cowoking-Space hat und eine gute
nicht-kommerzielle Ergänzung dazu darstellen würde.  Am jetzigen Standort des
"`Kombinat Süd"' gab es allerdings keine Möglichkeiten für den Hackspace, dort
unterzukommen.

\paragraph{Schott AG}

Eine Zeit lang waren beim "`Kombinat Süd"' auch Werksgebäude der Schott AG im
Gespräch, die auch gut geeignete Kellerräume günstig vermietet hätte.  Eine
diesbezügliche Anfrage von Konrad bei der Schott AG ergab allerdings, dass uns
derzeit keine Räumlichkeiten abgeboten werden können.

\paragraph{Coworking SOBAEXA}

Ende Januar haben sich Jens, Frank und Konrad mit der Coworking-Initiative
SOBAEXA getroffen, um auch dort Interesse an Räumlichkeiten im Umfeld eines
Coworking-Spaces anzumelden.

\paragraph{Abbe-Stiftung Häckel-Platz}

Bei der Begehung der späteren Räume der SOBAEXA Coworking-Initiative fand
Konrad heraus, dass auch im Dachgeschoß geeignete Räume zu mieten gibt.
Allerdings ergab eine diesbezügliche Anfrage bei der Abbe-Stiftung, die die
Räume vermietet, dass sie nur etagenweise vermieten.

\paragraph{Ehemaliger Jugendclub HUGO}

Auf Anregung von Daniel haben Frank und Konrad im Februar Kontakt zum
Ortsteilrat Winzerla aufgenommen, um in Erfahrung zu bringen, ob die
ehemaligen Vereinsräume des Jugenclubs HUGO in Winzerla durch den Hackspace
angemietet werden können.  Anlass dazu war eine Zeitungsmeldung, dass der
Ortsteilrat ergebnislos über die Zukunft dieses Gebäudes beraten hatte.  Der
Ortsteilrat hat uns an die Kommunalen Immobilien Jena (KIJ) verwiesen, wo wir
die Auskunft bekamen, dass die marode Bausubstanz des Gebäudes nur einen
Abriss zuläßt.

\paragraph{Wenigenjena}

Auf Initiative von Ruben haben sich Ruben und Konrad im Februar mit der
Orts\-teil\-bür\-ger\-meis\-te\-rin von Wenigenjena getroffen, um ihr den
Verein vorzustellen und sie nach Objekten in diesem Stadtteil zu befragen, die
als Vereinsräume in Frage kämen.

\paragraph{Löbstedter Straße (FAW)}

Aus dem Gespräch mit der Ortsteilbürgermeisterin von Wenigenjena hat sich ein
Kontakt zur Forschungsakademie der Wirtschaft (FAW) ergeben, die Büroräume in
der Löbstedter Straße 50 gegenüber des Schlachthofs vermieten.  Obwohl die
Räume von der Lage und Ausstattung her nicht optimal waren, sind wir nach
einer Besichtigung mit der FAW in Verhandlungen getreten, die jedoch wegen
Eigenbedarfs plötzlich auf unbestimmte Zeit verschoben wurden.

\paragraph{Ladenlokal in der Zwätzengasse}

Ende Mai hat Ruben im Netz ein Mietangebot für ein Ladenlokal in der
Zwätzengasse entdeckt.  Leider war es zum Zeitpunkt unserer Anfrage schon
vergeben.

\paragraph{Ladenlokale in der Neugasse bzw. Grietgasse}

In der Neugasse gab es zwei, in der Grietgasse eine vakante, 
potentiell geeignete Räumlichkeit für einen Hackspace. Frank nahm 
Kontakt mit dem Vermieter auf und musste feststellen, dass eine 
Finanzierung dieser nicht in erreichbarer Nähe liegt.

\subsection{Neue Vereinsräume}

Letzendlich haben wir durch ein von Jan im Internet entdecktes Angebot
Büroräume in der Krautgasse 26 gefunden.  Zur Disposition standen zwei Räume
-- ein kleinerer zum Durchgang hin und ein größerer mit zwei Zimmern zur
Krautgasse hin.  Obwohl mit einem größeren finanziellen Risiko verbunden,
haben wir uns mehrheitlich für den größeren der beiden angebotenen Räume
entschieden, da er zwei getrennte Zimmer sowie eine Miniküche umfaßt und mit
den Fenstern zur Krautgasse hin wesentlich bessere Möglichkeiten der
Außenwirkung bietet.

\paragraph{Finanzierung}

Das Problem war, dass die Miet- und Nebenkosten ([??\EUR]) die regelmäßigen
Einnahmen des Vereins überstiegen und die einmaligen Ausgaben wie
Maklercourtage ([??\EUR]) und Kaution ([??\EUR]) in der Größenordnung des
Vereinsvermögens lagen.  Um den Raum dennoch zu mieten, hat sich ein anonymer
Spender bereit erklärt, eine eventuelle Finanzierungslücke am Ende eines
Jahres zu schließen.  Da das aber immer noch nicht die finanzielle
Einstiegshürde gesenkt hat, haben sich spontan fünf Vereinsmitglieder
gefunden, um die Kaution von [1650\EUR] gemeinschaftlich aus eigener Tasche
aufzubringen.  Im Gegenzug hat sich der Verein im Namen des Vorstandes dazu
verpflichtet, ihnen den jeweiligen Betrag spätestens bis zum 5.\ August 2015
zurückzuzahlen.  Außerdem haben sich einzelne Mitglieder dazu entschlossen,
eine Einmalzahlung zu leisten bzw. einen (wesentlich) höheren Mitgliedsbeitrag
zu zahlen.  Derzeit umfaßt das Vereinsvermögen etwa zwei Monate Mietzins.
Zusammen mit den laufenden Mitgliedsbeiträgen halten wir uns damit bis etwa
[Monat] kommenden Jahres über Wasser, falls sich die Mitgliederzahl nicht
ändert.

Wir möchten uns an dieser Stelle bei allen bedanken, die durch ihren
finanziellen Beitrag das Vereinsleben in der jetzigen Form erst ermöglicht
haben.  In gleicher Weise möchten wir dem anonymen Spender für sein Angebot
danken und hoffen, davon keinen Gebrauch machen zu müssen, im Notfall aber
dennoch darauf zurückgreifen zu können.

Jan hat die Verhandlungen mit Makler und Hausmeister übernommen.  Nach
intensiver Durchsicht des Mietvertrages mußte nachverhandelt werden, da einige
Bedingungen mit dem Makler ander abgesprochen waren.  Für die Mietkaution hat
Frank ein Mietkautionskonto bei der Commerzbank Jena eingerichtet.  Die
Mehrwertsteuer auf die Courtage wurde uns vom Makler erlassen.

\paragraph{Umzug}

18. August 2012

\paragraph{Sicherheitsbeauftragter}

Als Sicherheitsbeauftragten hat der Verein Martin89 bestellt.

\paragraph{Einrichtung}

Vorrangig durch Sachspenden bzw. Eigenarbeit konnte unser Hackspace mit den
nötigsten Einrichtungsgegenständen bestückt werden:
\begin{itemize}
	\item Regale
	\item Stühle
	\item Arbeitstisch
	\item Leuchtreklame
	\item Küchenutensilien
	\item Putzkram
	\item Whiteboard
\end{itemize}
Wir danken all denjenigen, die Einrichtungsgegenstände gespendet oder
aufgebaut haben.  Das Whiteboard wurde aus der Vereinskasse gezahlt.

\paragraph{Internet}

Leider gibt es im Hackspace immer noch kein Internet, was vornehmlich daran
liegt, dass [Gründe].

\paragraph{Türschließanlage}

[Zusammenfassen]

1. Zwischen Samstag Nacht und Sonntag Nachmittag ist der Wifi-Router des
Schliessystems ausgestiegen. Wir vermuten das Netzteil als Fehlerquelle,
koennen das aber erst mit einem neuen Netzteil pruefen.

2. Im Schliesszylinder hat sich ein Teil (die Kupplung) verkanntet,
vermutlich weil der innen steckende Schluessel nicht perfekt senkrecht
stand. Die Schluesseldienstleister hat uns gesagt, dass der verwendete
Zylinder fuer den elektrischen Tueroeffner ungeeignet ist.

3. Wegen dieses Defektes konnte die Tuer von aussen weder mittels
Schluessel geoeffnet, noch auf einfachem Alternativweg (Lockpicking,
Aufbohren, Rausziehen) geoeffnet werden.

4. Nach 2,5 Stunden Bohren und fluchverstaerkter Gewaltanwendung auf den
Schliesszylinder konnte dieser heute Mittag entfernt werden.

5. Ein neuer Schliesszylinder wurde eingebaut. Dieser funktioniert mit
den bisherigen Schluesseln. Es besitzt ausserdem eine
Notoeffnungsfunktion und wurde vom Schluesseldienstleister explizit zu
dem Zweck ausgewaehlt, um von innen den elektrischen Tueroeffner zu
betreiben.

6. Wir haben zwei neue Schluessel bekommen und einen Alten beim
Oeffnungsversuch eingebuesst. Den Zusatzschluessel habe ich an Konrad
vergeben (damit hat jetzt jedes Vorstandsmitglied einen). Der
Schluesseldienstleister hat empfohlen, den zweiten neuen Schluessel fuer
den Tueroeffner zu verwenden, sobald der wieder eingesetzt wird, da bei
diesem der Verschleiss wegen hoher Nutzungsfrequenz am groessten ist.
Die alten Schluessel sind alle sehr abgenutzt (und schliessen deswegen
schlechter).

[+neue Lösung]

\paragraph{Fotografien}

Auf Vorschlag von Thomas als gleichzeitigem Mitglied des Fotoclubs Jena wurden
im Hackspace drei Fotos aufgehängt, die durch Thomas auch regelmäßig erneuert
werden.  Auch darüber hinaus erhoffen wir uns eine fruchtbare Zusammenarbeit
mit dem Fotoclub.

Wir danken all denjenigen, die sich an der Anschaffung der Bilderrahmen
finanziell beteiligt haben.

\paragraph{Getränke}

Frank hat stets für regelmäßigen Nachschub an Getränken gesorgt.

%------------------------------------------------------------------------------%
\section{Veranstaltungen}

\subsection{Regelmäßige Vereinsaktivitäten}

\begin{itemize}
	\item Hacken mit \texttt{0xAFFE} (dienstags 20 Uhr)
	\item Chaostreffs (sonntags 16 Uhr)
	\item Plenen (jeden zweiten Freitag im Monat 19 Uhr)
	\item Ausrichtung der regelmäßigen Stammtisches der LUG Jena
	\item Gemeinsames kochen mit Frank und Felix
		\begin{itemize}
			\item Selbstgemachte Nudeln (mit Nudelmaschine) und 
				Tomatensoße (21. September)
			\item Chili-Essen (19. Oktober)
			\item Kartoffelsuppe (16. November)
		\end{itemize}
\end{itemize}

\subsection{Vorträge \& Workshops}

\begin{itemize}
	\item Hacking:  Interpreter für Heidenhein, einer Steuersprache für CNC-Fräsen (Konrad, 23. Februar)
	\item Projektvorstellung:  passwdhash (Felix, 19. Februar)
	\item Projektvorstellung:  Wikileaks-Kabel-Parser (Konrad, 16. Februar)
	\item Gastvortrag:  C64 (Thomas Findeisen aus der M18 in Weimar, 9. Februar)
	\item Vortrag:  Wanderlust, ein E-Mail-Client für Emacs (Tim, 28. Februar)
	\item Workshop:  Zabbix + Postgres (Frank, 4. Juli)
	\item Spieleabend (11. Juli)
	\item Kurzvorstellung: moderncv, eine \LaTeX-Klasse für Lebensläufe (Frank, 15. Juli)
	\item Workshop:  dn42 (Martin89, 1. August)
	\item Vortrag:  Formale Begriffsanalyse (Felix, 14. August)
	\item Vortrag:  Gentoo (Markus, 24. Oktober)
	\item Vortrag:  Fliegen (Fortbewegung) (Oliver, 14. November)
	\item GNU-Radio Workshop (Martin, ??)
\end{itemize}

\subsection{Nerdfahrschule}

Auf Initiative von Konrad wurde während der Studieneinführungstage der
Fachschaft Informtik eine Workschopreihe zu Grundfertigkeiten eines Hackers
organisiert.  Die Veranstaltungen wurden über die Fachschaft Informatik
beworben und auch bei den Fachschaften Physik sowie SciTec \& Maschinenbau
(Fachhochschule) bekannt gemacht.

\begin{itemize}
	\item Anonym im Netz (Jens, 1. Oktober)
	\item Open Source und Linux (Frank, 2. Oktober)
	\item Powereditoren vi \& emacs (Martin \& Konrad, 4. Oktober)
	\item Erste Schritte auf dem Linux-Terminal (Jan, 9. Oktober)
	\item \LaTeX (Jörg, 10. Oktober)
	\item Moderne Versionskontrollsysteme (Tim, 11. Oktober)
\end{itemize}

Drei der Veranstaltungen haben den Hackspace gefüllt, drei waren weniger
besucht, was wahrscheinlich der ungünstigen Terminwahl sowie dem kurzen
Vorlauf bei der Ankündigung geschuldet war.  Wir planen, die Veranstaltung im
Laufe des Semesters zu wiederholen.  Der Vorstand bedankt sich bei den
Referenten für ihren Einsatz.

\subsection{FSFE-Vortrag}

Am 11. November wurde in Kooperation mit der Hochschulgruppe der Piraten ein
Vortrag von Matthias Kirschner von der FSFE mit dem Titel "`Vom Aussterben
bedroht: Die Universalmaschine Computer"' organisiert.  Dazu wurden von Jan
und Frank im Vorfeld Plakate entworfen und von [Namen] an [Orten] verteilt.
Es kamen etwa 40 Hörer.  Anschließend gab es eine Nachsitzung im Hackspace.

\subsection{Cryptoparty}

Auf Initiative von Jens wurde am 23. November eine Cryptoparty organisiert.

\subsection{Sonstige Veranstaltungen und Aktivitäten}

\begin{itemize}
	\item Prüfsteine für die Oberbürgermeisterkandidaten (Anfang April, Frank)
	\item Einzugsfeier (24. August)
\end{itemize}



%------------------------------------------------------------------------------%
\section{Sonstiges}

\subsection{Werbung}

\paragraph{Beratung durch Thüringer Agentur für die Kreativwirtschaft}

Am 15.~August hat Konrad ein längeres Telefongespräch mit einer Beraterin der
Thüringer Agentur für Kreativwirtschaft geführt, um den Verein in diesem
Bereich bekannt zu machen und weitere Möglichkeiten der Förderung und
Außenwerbung ausfindig zu mchen.  Uns wurde ein Auftritt bei einem der
regelmäßig stattfindenden [Veranstaltung im Intershop-Turm] sowie ein
Förderantrag bei der Intershop-Stiftung empfohlen -- was beides bereits
geschehen war.  Außerdem haben wir ein Marketing- und Vertriebskonzept als
Leitfaden bekommen.

\paragraph{Markt der Möglichkeiten}

Ein Auftritt auf dem Markt der Möglichkeiten mußte kurzfristig abgesagt
werden, da Tim und Konrad, die sich bereit erklärt hatten, einen Stand zu
betreuen, beide verhindert waren.  Wir möchten aber im kommenden Jahr daran
teilnehmen.

\paragraph{Kurzvorstellung des Hackspace an der Fachhochschule}

Am 23.~Oktober haben Frank und Konrad den Hackspace in einer viertelstündigen
Kurzvorstellung vor Studenten der Fachhochschule präsentiert.  Der Dozent war
der Idee eines Hackspaces sehr aufgeschlossen und hat uns zugesagt, uns auch
in Zukunft zu unterstützen.  Wir planen weitere derartige Kurzvorstellung.

\subsection{Förderanträge}

\paragraph{Intershop-Stiftung}

Am 6. Oktober haben wir bei der Intershop-Stiftung einen formlosen Antrag auf
eine Förderung von 500\EUR\ für die Anschaffung eines Beamers sowie einer
Leinwand gestellt.  Frank hat daraufhin mündlich die Zusage erhalten, dass uns
die Förderung gewährt wird, sobald der Verein die Gemeinnützigkeit erlangt.

\paragraph{Chaos Computer Club}

Am [Datum] haben wir beim Chaos Computer Club am einen Antrag auf
Anschubfinanzierung für die Anmietung der Vereinsräume gestellt, um die
Finanzierungslücke zu schließen.  Eine Antwort steht noch aus, ein positiver
Bescheid ist jedoch unwahrscheinlich.  Der Vorstand überlegt, den Antrag
zurückzuziehen und einen erneuten Antrag zu stellen, sobald sich ein
Erfa-Kreis gründet, der die Vereinsräume nutzt.

\subsection{Sonstiges}

\begin{itemize}
	\item
		Am [Datum] wurde für den Hackspace ein dedizierter Server mit
		[Diensten] eingerichtet.
	\item
		Am [Datum] wurde die Domain von \texttt{hackspace-jena.de} auf
		\texttt{krautspace.de} umgezogen und eine entsprechende Umleitung
		eingerichtet.
	\item
		Am 1. August hat Martin89 den Hackspace-Server erfolgreich an das
		dn42-Netz angeschlossen.  Sobald der Hackspace einen Internetzugang
		besitzt, soll das dn42 auch von dort aus direkt zugänglich sein.
	\item
		Am [Datum] wurde der Hackspace mit einer HackerSpace Status API
		versehen:  Es kann nun Online eingesehen werden, ob und mit wieviel
		Personen der Raum besetzt ist.
\end{itemize}

\section{Ausblick}

Sollte der Vorstand in der bestehenden Bestzung in seinem Amt bestätigt
werden, so möchte er die Vereinsaktivitäten weiter kontinuierlich ausbauen.
Insbesondere sollen auch neue Zielgruppen erschlossen werden.  So wollen wir
zunächst sporadisch, langfristig aber regelmäßig Aktivitäten für Kinder und
Jugendliche anbieten.  Nach der erfolgreichen Raumsuche soll im kommenden Jahr
die Bekanntmachung und Vernetzung des Vereins im Mittelpunkt stehen.  Ziel ist
es, die Mitgliederzahl auf ein Niveau zu steigern, auf dem sich der Verein
auch ohne die zur Zeit hohen vereinzelten Förderbeiträge selbst trägt.


\end{document}


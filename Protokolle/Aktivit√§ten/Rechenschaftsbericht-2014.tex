\documentclass[ngerman]{scrartcl}
\usepackage[utf8]{inputenc}
\usepackage[T1]{fontenc}

\usepackage{color}
\usepackage{hyperref}
\definecolor{darkblue}{rgb}{0,0,.5}

\hypersetup{pdftex=true, colorlinks=true, %
	breaklinks=true, linkcolor=black, %
	urlcolor=darkblue}

\usepackage[ngerman]{babel}
\usepackage{enumitem}
\usepackage{marvosym}

% Euro
\usepackage{eurosym}

% Einheiten und Zahlen korrekt setzen
\usepackage{siunitx}
\sisetup{locale = DE}
\DeclareSIUnit{\EUR}{\text{\euro{}}}

\usepackage{booktabs}
\usepackage{tabularx}

\setitemize{itemsep=0pt}

\title{Rechenschaftsbericht}
\subtitle{Hackspace Jena e.\,V.}
\author{%
	Jens Kubieziel (Vorsitzender)\\
    Felix Kästner (Schriftführer)\\
	Martin Neß (Schatzmeister)
}
\date{25.11.2013 bis 15.11.2014}

\begin{document}

\maketitle{}

\tableofcontents{}

\newpage{}

\section{Gemeinnützigkeit}

Nach unserer Erklärung zur Körperschafts- und Gewerbesteuer für das Jahr 2013 beim Finanzamt Jena
erreichte uns ein Freistellungsbescheid.
Dieser besagt, dass der Hackspace Jena~e.\,V. für 2013 von den Körperschafts- und Gewerbesteuer befreit ist. Weiterhin ist der Verein berechtigt, für Mitgliedsbeiträge und Spenden Zuwendungsbestätigungen auszustellen.
Somit hat man uns die Gemeinnützigkeit bestätigt.

Vereine erstellen, im Gegensatz zu Privatpersonen, alle drei Jahre eine Steuererklärung.
Dadurch enthält die Erklärung über das Jahr 2013 auch Angaben und Unterlagen seit der Vereinsgründung von Anfang 2012.
Daraus ergibt sich auch, dass die nächste Erklärung spätestens 2017 für die Jahre 2014--2016 eingereicht werden muss.

\section{Mitgliederenwicklung}

Aktuell, d.\,h. zum 9.\,November~2014, hat der Verein 39~Mitglieder. Am 24.\,November~2013 waren es 33~Mitglieder.
In diesem Zeitraum von einem Jahr haben wir 9~Mitglieder begrüßt und 3~Mitglieder verabschiedet. Die ausgeschiedenen Mitglieder gaben an aus Jena wegzuziehen.
Somit ergibt sich ein Mitgliederzuwachs von 6~Mitgliedern.

\section{Finanzen}

Im Zeitraum vom 16.\,November~2013 bis 9.\,November~2014 erhielt der Verein Einnahmen von \SI{7629,82}{\EUR} im ideellen Bereich und \SI{4334,39}{\EUR} im Bereich des Zweckbetriebs.
Gesamt sind dies Einnahmen in Höhe von \SI{11964,21}{\EUR}.


In diesem Zeitraum tätigte der Verein Gesamtausgaben von \SI{10873,24}{\EUR}. Davon sind \SI{8596,92}{\EUR} dem ideellen Bereich und \SI{2276,32}{\EUR}  dem Zweckbetrieb zuzuordnen.

In diesem Zeitraum ergibt sich ein Überschuss von \SI{1090,97}{\EUR}.

\begin{table}[h!]
  \centering{}
	\begin{tabular}{l|r|r|r}
          \toprule
	\textbf{Konto} & \textbf{Kontostand 16.11.2013} & \textbf{Kontostand 09.11.2014} \\
          \midrule
	Barkasse & \SI{216,22}{\EUR} & \SI{701,03}{\EUR} \\
	Reparier-Café Barkasse & -- & \SI{171,36}{\EUR} \\
	Kautionskonto & \SI{1663,15}{\EUR} & \SI{1666,22}{\EUR} \\
	Girokonto & \SI{3319,68}{\EUR} & \SI{3472,95}{\EUR}
	\end{tabular}
\caption{Übersicht der Konten}
\end{table}

\subsection{Ideeller Bereich}
\label{sec:ideeller_bereich}

Im ideellen Bereich gab es in diesem Zeitraum folgende Einnahmen:
\begin{itemize}
\item Mitgliedsbeiträge in Höhe von \SI{6991}{\EUR}
\item \SI{487,28}{\EUR} Spenden (davon \SI{223,86}{\EUR} bei Veranstaltungen des Reparier-Cafés eingenommen)
\item \SI{148,46}{\EUR} Gutschrift aus der Betriebskostenabrechnung 
\item \SI{3,07}{\EUR} Zinsen.
\end{itemize}
Insgesamt sind das \SI{7629,82}{\EUR}.

Die Ausgaben in diesem Zeitraum für Miete, Internet sowie die Abschläge für Nebenkosten betragen \SI{7799,92}{\EUR}.
Dieser Betrag berücksichtigt nicht die Gutschrift aus der Betriebskostenabrechnung, da der Abrechnungszeitraum vor dem 16.\,November~2013 liegt.
Da die Abrechnungen für Nebenkosten und Strom noch nicht vorliegen, weicht dieser Betrag ggf. von den tatsächlichen Kosten ab.

Für Kontoführungsgebühren, Rundfunkbeträge, Versicherung und Notar wurden in diesem Zeitraum \SI{290,37}{\EUR} ausgegeben.
Für sonstige Sachen wurden \SI{506,63}{\EUR} ausgegeben. Dies sind zum Teil Ausstattungsgegenstände und Verbrauchsmaterialien wie Reinigungsmittel, Müllbeutel usw. und zum anderen Anschaffungen z.\,B. für das Elektroniklabor. Eine Aufstellung über die Anschaffungen für das Elektroniklabor befindet sich im Abschnitt~\ref{sec:anschaffungen}.

\subsection{Zweckbetrieb}
\label{sec:Zweckbetrieb}
Im Krautspace gibt es weiterhin die Möglichkeit, Getränke und Süßigkeiten zu erwerben.
Dies zählten wir früher als wirtschaftlichen Geschäftsbetrieb. Eine Auskunft des Steuerberaters ergab, dass es sich um einen Zweckbetrieb handelt.

Aus den Verkäufen  von Getränken und Süßigkeiten ergaben sich Einnahmen von \SI{4066,30}{\EUR}, wobei für \SI{1914,85}{\EUR} Waren eingekauft worden.
Damit ergibt sich ein Überschuss von \SI{2151,45}{\EUR}, der für Finanzierungen im ideellen Bereich verwendet werden kann.

Im Vergleich zum letzten Bericht sind die Einnahmen um ca.\,\SI{340}{\EUR} gesunken, aber der Überschuss um ca.\,\SI{20}{\EUR} gestiegen.
Mit Beachtung der lagernden und bereits bezahlten Waren und der Prepaid-Zettel kann man abschätzen, dass der Überschuss gleichbleibend ist.

\subsection{Zweckgebundene Spenden}
\label{sec:zweckgebundene_spenden}

Wir besitzen zweckgebundene Spenden für Projekte. Eine Aufstellung befindet sich in Tabelle~\ref{table:spenden}.
In diesem Bereich ergab sich wenig Aktivität, lediglich eine Tellerspende für das Freifunkprojekt von \SI{7,50}{\EUR} ging ein. Darüber hinaus stehen noch \SI{95}{\EUR} für das Projekt Theremin bereit.

\begin{table}[h]
	\centering
	\begin{tabular}{l|r|r|r}
          \toprule
	\textbf{Projekt} & \textbf{Eingegangen im Zeitraum} & \textbf{Verwendet} & \textbf{Verfügbar} \\ \midrule
	Theremin & \SI{0}{\EUR} & \SI{0}{\EUR} & \SI{95}{EUR} \\
	Freifunk & \SI{7,50}{\EUR} & \SI{0}{\EUR} & \SI{7,50}{\EUR} \\
	\end{tabular}
	\caption{Sachgebundene Spenden für Projekte}
	\label{table:spenden}
\end{table}

\section{Anschaffungen}
\label{sec:anschaffungen}
Um den Besuchern des Krautspaces mit ihren Gerätschaften und der steigenden Anzahl an Gerätschaften in den Vereinsräumen ausreichend Steckdosen bieten zu können, wurden Steckdosenleisten (z.\,T. mit Schalter zum praktischen Abschalten und Stromsparen) und Verlängerungskabel für \SI{40,06}{\EUR} angeschafft.

Für den Kauf von Werkzeugen wie Lötstation, Tastköpfe und Zangen haben wir \SI{233,65}{\EUR} ausgegeben.
Die Anschaffungen stehen im Elektroniklabor zur Verwendung bereit.

Für die Veranstaltungen des Reparier-Cafés wurden Nähutensilien für \SI{52,50}{\EUR} eingekauft.

\section{Leihgaben und Sachspenden}

In diesem Jahr haben viele Leihgaben und Sachspenden ihren Weg in den Krautspace gefunden. 
Zu den Sachspenden zählen ein Analog-Oszilloskop, ein Multimeter und viele Kleinteile, Schrauben und Elektronikbauteile aus einer aufgelösten Werkstatt.
Ein Proxxon mit viel Zubehör wurde dem Verein geliehen.
Die Platinenätzanlage zum Ätzen selbst erstellter Layouts ist eine Dauerleihgabe und wurde schon für einige Projekte benutzt.
Ein besonderes Highlight unter den Leihgaben stellt der 3D-Drucker dar.
Dieser kann nach Einweisung von jedem Vereinsmitglied genutzt werden.

\section{Logo}

Seit Anfang 2014 hat der Verein ein Logo, welches in unterschiedlichen Versionen benutzt werden kann. 
Es findet sich seitdem auf vielen Veröffentlichungen des Vereins wieder. 

\section{Veranstaltungen}

\subsection{Regelmäßige (Vereins-)aktivitäten}

Ein großer Teil der Vereinstätigkeiten ergibt sich aus der
Bereitstellung der Infrastruktur. So haben sich regelmäßige offene Runden
etabliert, in denen themenbezogen gearbeitet wird. Für die
einzelnen Veranstaltungen haben sich Freiwillige aus dem Verein
gefunden, die sich um die Organisation kümmern.

\begin{table}[h]
  \centering{}
	\begin{tabularx}{\textwidth}{l|X}
          \toprule
		Name & Turnus \\ \midrule
		Elektronikrunde & jeden Montag ab 19:30 Uhr\\
		Offene Runde am Dienstag & jeden Dienstag ab 20 Uhr\\
		Sprechstunde Informationssicherheit & jeden ersten Dienstag im Monat ab 20 Uhr, seit Oktober 2014\\ 
		Spieleabend & jeden ungeraden Mittwoch ab 20 Uhr\\
		Linux User Group & jeden geraden Donnerstag ab 19 Uhr\\
		Freifunktreffen & jeden ungeraden Donnerstag ab 20 Uhr\\
		Lockpicking & jeden ersten Freitag im Monat ab 19 Uhr, beendet seit Juli 2014\\
		Gaming-Stammtisch & jeden ersten Freitag im Monat ab 19 Uhr, seit September 2014\\
		Plenum & jeden zweiten Freitag im Monat ab 19 Uhr\\
		Kochen & jeden dritten Freitag im Monat, beendet seit September 2014\\
		Thuringiafurs Stammtisch & jeden dritten Samstag im Monat ab 14 Uhr\\
		Chaoscafe / Chaostreff & jeden ungeraden Sonntag ab 16 Uhr, beendet seit März 2014\\
		Reparier-Café & monatlich seit Juli 2014\\
\end{tabularx}
\end{table}

\subsubsection{Elektronikrunde}

Die Elektronikrunde trifft sich seit 2013 jeden Montag im Krautspace um
sich konzentriert in Technikprojekte vertiefen zu können. Die
Teilnehmer helfen sich gegenseitig mit Werkzeugen, Materialien und
Wissen aus, um ihre Ideen zu verwirklichen. Der Verein stellt dabei
einen großen Teil der Werkzeuge und Verbrauchsmaterialien bereit.
Bauteile für die Schaltungen wurden durch die Teilnehmer selbstständig
organisiert.

\subsubsection{Offene Runde am Dienstag}

Jeden Dienstag gibt es die (themen-)offene Runde im Raum. D.\,h. der Raum
steht zur freien Verfügung, um gemeinsam an Themen rund um
Informationstechnologie, der Computersicherheit und des
Datenschutzes zu diskutieren und zu arbeiten.

\subsubsection{Sprechstunde Informationssicherheit}

Mitte des Jahres kam die Idee zu einem Cryptofreitag auf. 
Dabei sollten abweichend von den Cryptoparties nicht hauptsächlich Vorträge gehalten werden sondern es war angedacht sich auf die Fragen der Besucher zu konzentrieren. 
Da die potentiellen Betreuer freitags nicht verfügbar sind wurde dann eine Sprechstunde für einen Dienstag im Monat konzipiert. 
Das Ziel der Veranstaltung ist es die Fragen der Besucher zu den Themen Verschlüsselung, Privatsphäre und Datensicherheit zu beantworten. 

\subsubsection{Spieleabend -- Gesellschaftsspielerei}

In der Spielerunde werden regelmäßig Brett- und Kartenspiele zu einem
bestimmten vorher festgelegten Thema gespielt. Dabei liegt der Schwerpunkt
nicht auf den üblichen Partyspielen sondern bei anspruchsvollen Spielen mit
unterschiedlichen Spielkonzepten. Dabei kommen sehr viele unterschiedliche
Spiele zum Zug. Teilweise werden auch selbst entwickelte Spiele vorgestellt
und ausprobiert oder neue Spiele von Spielemessen präsentiert.

\subsubsection{Stammtisch der LUG Jena}

Der Stammtisch der Linux-User-Group Jena beschäftigt sich alle zwei Wochen
mit Themen rund um freie Software und insbesondere GNU/Linux. Es geht dabei
um den Erfahrungsaustausch und die Diskussion aktueller Entwicklungen.

\subsubsection{Freifunktreffen}

Die wachsende Freifunkgemeinschaft in Jena trifft sich alle zwei Wochen
im Krautspace, um die aktuelle Entwicklung zu besprechen und
Interessierten die Konzepte hinter Freifunk zu erklären, sowie die
Software auf und hinter den von Freifunk betriebenen Knoten zu
verbessern.

\subsubsection{Lockpicking mit Adrian}

Lockpicking, also Aufsperren von Schlössern ohne den passenden
Schlüssel, hat sich seit einiger Zeit zu einem Sport entwickelt, den
auch einige Mitglieder des Hackspace ausprobieren wollen. Dazu hat
Adrian eine monatliche Veranstaltung etabliert, in der sich die Teilnehmer
autodidaktisch und in gegenseitiger Hilfe die benötigten Fertigkeiten
beibringen. Es geht dabei explizit um die sportliche Herausforderung
und das kritische Auseinandersetzen mit den Sicherungsmechanismen
historischer und moderner Schlösser.

Auch diese Veranstaltung wurde aus Mangel an Interesse seit Jahresmitte beendet. 

\subsubsection{Gaming-Stammtisch}

Beim Gamingstammtisch geht es um Computerspiele – egal auf welcher Plattform, ob gekauft oder selbst geschrieben.
Die Schwerpunkte sind Game Design und die Auswirkungen des Spielens auf Spieler und Gesellschaft. 

\subsubsection{Plenum}

Das Vereinsplenum fand an jedem 2. Freitag im Monat in den
Vereinsräumen statt.

Entscheidungen des Plenums haben dabei keinen bindenden Charakter und
wurden entsprechend zum vereinsinternen Austausch bzw.\ zur Klärung von
organisatorischen Fragen von Angesicht zu Angesicht genutzt. Das Plenum ist dabei offen für Gäste.
Die Protokolle der Treffen sind über das Wiki unter
\url{https://www.krautspace.de/hswiki:verein:plenum:start} verfügbar.

\subsubsection{Gemeinsames Kochen}

Essen hält Leib und Seele zusammen und Kochen verbindet:
Ziel ist aber nicht nur das Zusammenkommen, sondern der kreative Umgang
beim Prozess der Zubereitung sowie die Erlangung von kulinarischen
Fähigkeiten -- oder anders gesagt: Der Bastler lebt nicht nur von
Tiefkühlpizza allein und das muss er auch irgendwoher erlernen.

Aus diesem Grund wurde durch Felix und weitere Mitglieder jeweils am 3. Freitag
im Monat eine gemeinsame Kochrunde in den Vereinsräumen organisiert.

\begin{itemize}
    \item Käse-Lauch-Suppe (21.02.2014)
    \item Kürbissuppe (15.08.2014)
    \item Pfannkuchen (21.03.2014)
    \item Grüner Bohneneintopf (01.08.2014)
\end{itemize}

Der Verein unterstützte die Veranstaltungen durch Bereitstellung der
Infrastruktur (Raum, Herd, Töpfe etc.). Die benötigten Zutaten wurden
durch die Teilnehmer selbstständig eingekauft. 

Aufgrund des zurück gegangenen Interesses wurde die Veranstaltung seit Mitte des Jahres nicht mehr weiter geführt. 

\subsubsection{Reparier-Café}

Seit Mai 2014 hat eine kleine Gruppe angefangen ein Reparier-Café zu gründen. 
Da die Idee auch unter Mitgliedern des Vereins viel Zustimmung fand haben sich einige Mitglieder daran beteiligt. 
Das erste Café fand am 31.07.2014 in den Vereinsräumen statt und war sehr gut besucht. 
Etwas später ist das Reparier-Café ein offizieller Teil des Vereins geworden. 


\subsection{Vorträge und Workshops}

\begin{table}[h]
  \centering{}
  \begin{tabularx}{\textwidth}{l|X}
	\textbf{Datum} & \textbf{Inhalt} \\ \midrule
	29.11.2013 & Lange Nacht der Wissenschaften Jena \\
	12.01.2014 & Python-Workshop mit Markus \\
	14.01.2014 & Cryptoparty beim Fanprojekt Jena \\
	05.02.2014 & Einführung in Haskell mit Jan \\
	10.02.2014 & Vortrag Spannungsversorgung mit Hannes \\
	15.02.2014 & Festplatten-Crypto Workshop \\
	16.02.2014 & Lötworkshop für Kinder \\
	04.03.2014 & Vortrag zu DNSSEC mit Lutz Donnerhacke \\
	24.03.2014 & Einführung 3D-Drucker \\
	01.04.2014 & Vortrag "`Wieviel Astronomie steckt in einer Flasche Bier"' von Florian Freistetter \\
	08.04.2014 & Vortrag zum community Netzwerk dn42 von Martin \\
	10.03.2014 & Vortrag Spannungsversorgung Teil 2 mit Hannes \\
	11.03.2014 & Vortrag zu Planetarien von Severin \\
	18.03.2014 & Groovy-Workshop mit Oli \\
	29.03.2014 & Thementag "`Technik und Internet"' im Rahmen der Jugendweihe-Vorbereitung \\
	12.05.2014 & Einführung in die Programmiersprache julia durch mk \\
	25.09.2014 & Python-Workshop Teil 2 mit Markus \\
	\end{tabularx}
	\caption{Liste der besonderen Vorträge und Workshops}
\end{table}

\subsubsection{Cryptoparty beim Fanprojekt Jena}

Da die Teilnehmer kaum eigene Rechner mitgebracht haben wandelte sich die
Cryptoparty zu einem Vortrag in dem wir Thunderbird mit Enigmail ausgestattet
haben, ein wenig über XMPP geredet sowie über ein paar Verhaltensregel in
sozialen Netzwerken gesprochen haben. Es war eine lockere und freie
Veranstaltung. Bei der Q\&A-Session kamen noch Fragen zu Tor und verwandte
Themen.

\subsubsection{Festplatten-Crypto Workshop}

Die Teilnehmer des Workshops haben mit Hilfe von Tim, Markus und Felix
verschlüsselte Sicherungskopien ihrer Daten angelegt und danach die Fetplatten
ihrer Laptops verschlüsselt. Während des Workshops wurden zusammen mit den
Teilnehmern Probleme gelöst und Fragen zu Sicherheit und Best Practices
beantwortet. 

%TODO: mehr beschreibung zu den vorträgen?


\section{Tätigkeitsberichte des Vorstandes}

\subsection{Jens (Konrad)}

Jens hat sich mit Folgendem beschäftigt:

\begin{itemize}
	\item Einreichen von Protokoll und Satzungsänderung der letzten Mitgliederversammlung beim Notar
    \item Vorstandstreffen bzw. Abstimmung im Vorstand per E-Mail
    \item Bearbeitung der Vereinspost
    \item Außendarstellung des Vereins
	\item Entwurf des Förderantrags für Beamer und Leinwand bei der Intershopstiftung
	\item Konzeption und Materialbeschaffung LED-Matrix
	\item Ankündigung der Kryptoparty am 25. Januar 2012:  Plakat \& Bildschirm FH
	\item Koordination und Ansprechpartner Lange Nacht der Wissenschaft, Drucken der Plakate
	\item Interview mit dem Akrützel zum Krautspace
	\item Vorstellung des Hackspace bei der Stadtführung für die Erstsemester Informatik
    \item Gesprächsleitung bei Plenen
	\item Absprache der Teilnahme des Hackspaces beim Markt der Möglichkeiten
	\item mehr erfolglose als -reiche Versuche, Referenten für einen Vortrag/Workshop zu gewinnen
	\item vergebliche Fotoaktion "Mein Lieblingshack"
    \item Mitverfassen dieses Rechenschaftsberichtes
    \item inhaltliche Vorbereitung der Mitgliederversmmlung
\end{itemize}

\subsection{Felix}

Felix hat sich in seiner Funktion als Vorstandsmitglied mit Folgendem 
beschäftigt:

\begin{itemize}
    \item Gesprächsleitung bei Plenen und anderen Veranstaltungen
    \item Erstellung und Überarbeitung von Plenumsprotokollen
    \item Schreiben von Einladungen für Plenum, Mitgliederversammlung und 
        Veranstaltungshinweise
    \item Ausgabe von Schlüsseln für das Schließsystem
    \item Unregelmäßige Vorstandstreffen, viel Kontakt und Abstimmung über 
        elektronische Kommunikationswege
    \item Mitverfassen dieses Rechenschaftsberichtes
    \item Organisation der Teilnahme des Vereins am Markt der Möglichkeiten an der FSU
\end{itemize}

Als Vereinsmitglied hat er sich hiermit beschäftigt: 

\begin{itemize}
    \item Unterstützung bei einigen Veranstaltungen (z.B. Kochrunden)
    \item Überarbeitung und Dokumentation des Schließsystems
    \item Etablierung der "`Sprechstunde Informationssicherheit"'
\end{itemize}

\subsection{Martin}
Martin hat sich in seiner Funktion als Schatzmeister und Vorstandsmitglied mit Folgendem beschäftigt:
\begin{itemize}
	\item Finanzverwaltung
	\begin{itemize}
		\item Buchführung
		\item Rechnungen bezahlen
		\item Unterlagen abheften
		\item regelmäßige Kassenprüfungen
		\item vier Berichte an die Mitglieder geschrieben
		\item Zuwendungsbescheinigungen erstellt
		\item Besuch beim Steuerberater
		\item Steuererklärung erstellt
	\end{itemize}
	\item Mitgliederverwaltung
	\begin{itemize}
		\item Mitglieder durch Email begrüßt/verabschiedet
		\item Mitglieder erinnert ihre Beiträge zu Zahlen
	\end{itemize}
	\item Bar mit Getränken und Süßigkeiten:
	\begin{itemize}
		\item Planung der Warenbeschaffung
		\item Warenwälzung
		\item Getränkebestellung bei Heiko Wackernagel
		\item Einkäufe u\.a.\ bei Globus
		\item Preisabstimmung
		\item Abrechnung
	\end{itemize}
	\item Änderung des Eintrags beim Vereinsregister, Besuch beim Notar
	\item Kontakt mit der Bank, Berechtigung für Kontozugriff einholen
	\item Ankündigung von Veranstaltungen auf der Mailingliste
	\item Planung und Einladung zur Mitgliederversammlung
	\item Erstellung dieses Rechenschaftsberichts
	\item Leerung von Briefkasten und Postfach
	\item Unregelmäßige Vorstandstreffen, viel Kontakt und Abstimmung über elektronische Kommunikationswege
	\item Absprachen mit Projekten (Elektronikrunde, Reparier-Café)
	\item Besuch einiger Plenen von Verein und Reparier-Café
	\item Mitbetreuung des Servers
	\item Stand betreut auf dem Markt der Möglichkeiten
\end{itemize}

Als Vereinsmitglied hat Martin sich mit folgenden beschäftigt: 
\begin{itemize}
	\item Betreuung des Spieleabends
	\item Anbindung an das dn42 (dezentrales Community Netzwerk)
	\item Administration des (Kabel-)Netzwerks im Krautspace
	\item Mitbetreuung des Servers
	\item Entwicklung, Umsetzung und Anbindung der Status-Ampel
\end{itemize}

\section{Ausblick und Herausforderungen}

\subsection{Finanzen}
\label{sec:ausblick:finanzen}

\subsubsection{Wahrung der Gemeinnützigkeit}
Um den Status der Gemeinnützigkeit in 3~Jahren wieder zu erhalten ist eine korrekte Mittelverwendung nach Satzung und Abgabenordnung erforderlich.
Auch sollte dies in ausreichenden Maß dokumentiert werden um später die Erklärung für das Finanzamt erstellen zu können.

\subsubsection{Auslösung Kautionsdarlehen}
\label{sec:katrionsdarlehen}
Bis August 2015 muss der Verein seiner Vereinbarung mit den Darlehensgebern nachkommen und an die fünf Personen je \SI{333}{\EUR} zurückzahlen, die diese 2012 zur Hinterlegung der Kaution für die Vereinsräume in der Krautgasse~26 zinsfrei zur Verfügung gestellt haben.
Dies ist fest eingeplant und es sind bereits \SI{915,98}{\EUR} dafür bereitgestellt.
Das bedeutet aber auch das weitere \SI{749,02}{\EUR} bis August dafür verfügbar sein müssen.

\subsection{Mitgliederentwicklung}
Wir haben weiterhin ein Defizit bei der Finanzierung des Krautspaces und der Vereinskosten aus den Mitgliedsbeiträgen.
Abhilfen können mehr beitragszahlende Mitglieder schaffen.
Wir sollten ca.\,50~Mitglieder anstreben aber auch daran arbeiten den Mitgliederschwund durch neue Mitglieder zu auszugleichen.

\subsection{Außenwerbung}
Uns erreichen regelmäßig Leute die froh sind den Verein jetzt zu kennen und bedauern, dass sie von uns nicht schon zeitiger gehört haben.
Daher sollten wir an unserem Bekanntheitsgrad arbeiten, auch im Hinblick auf die Mitgliederentwicklung und steigende Besucherzahlen im Krautspace.

\end{document}

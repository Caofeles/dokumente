\documentclass[10pt,DIV16]{scrartcl}

\usepackage{color}
\usepackage{hyperref}
\definecolor{darkblue}{rgb}{0,0,.5}

\hypersetup{pdftex=true, colorlinks=true, %
	breaklinks=true, linkcolor=black, %
	urlcolor=darkblue}

\usepackage[ngerman]{babel}
\usepackage{enumitem}
\usepackage[utf8]{inputenc}
\usepackage{marvosym}

% Euro
\usepackage{eurosym}

\setitemize{itemsep=0pt}

\title{Rechenschaftsbericht}
\subtitle{Hackspace Jena e.\,V.}
\author{%
	Jens Kubiziel (Vorsitzender)\\
    Felix Kästner (Schriftführer)\\
	Martin Neß (Schatzmeister)
}
\date{25.11.2013 bis ??.??.2014}

\begin{document}

\maketitle{}

\tableofcontents{}

\newpage{}

\section{Gemeinnützigkeit}

Infolge der Satzungsänderung der letzten Mitgliederversammlung wurde 
uns durch das Finanzamt Jena mit Wirkung zum 1.1.2013 die vorläufige 
Gemeinnützigkeit bestätigt. Diese basiert auf den von uns durch die 
Satzung gesetzten Zielen und muss durch Leben dieser bestätigt werden. 

\section{Mitgliederenwicklung}

Die Anzahl der Mitglieder konnte von 30\footnote{Stand: 1.12.2012} auf 
33\footnote{Stand: 24.10.2013} erhöht werden. Leider ist eine relativ 
hohe Fluktuation zu beobachten, deren Gründe zum Beispiel im Wegzug aber 
auch in der Inaktivität einiger Mitglieder liegen. Ferner wurden vier Mitglieder 
ausgeschlossen, da sie ihre Mitgliedsbeiträge endgültig nicht entrichtet 
haben.

Im Laufe des Berichtszeitraum wurden zwei Mahnläufe durchgeführt, um 
säumige Mitglieder zu erinnern. 

\section{Finanzen}

Eine Übersicht der Konten zum Stichtag 16.11.2013:

\begin{table}[h]
\begin{tabular}{l|r|r|r}
\textbf{Konto} & \textbf{Kontostand 1.12.2010} & \textbf{Kontostand 16.11.2013} & \textbf{Differenz} \\ \hline
Barkasse & 60,51\euro & 216,22\euro & 155,21\euro \\
Kautionskonto & 1662,00\euro & 1663,15\euro & 1,15\euro \\
Girokonto & 1471,47\euro & 3319,68\euro & 1848,21\euro \\ \hline
\end{tabular}
\end{table}

Insgesamt hat der Verein im Zeitraum vom 01.12.2012 bis 16.11.2013 Einnahmen 
von 13.000,00\euro{} erziehlt. Diese verteilen sich zu 
8.592,36\euro\footnote{Siehe auch Kapitel \ref{sec:ideeller_bereich}} auf 
den ideellen Bereich sowie 4.407,64\euro{} im wirtschaftlichen 
Bereich\footnote{Siehe auch Kapitel \ref{sec:wirschaftlicher_bereich}}. 
Ausgegeben wurden insgesamt 10.994,93\euro, die sich zu 8.483,92\euro{} 
auf den ideellen Bereich sowie zu 2.511,01\euro{} auf den wirtschaftlichen 
Bereich verteilen. Es können also 2.005,07\euro{} als monetärer 
Überschuss in die nächsten Periode übernommen werden. Mit Blick auf die 
unter Kapitel \ref{sec:ausblick:finanzen} beschrieben, bereits zu 
erwartenden Herausforderungen, ist dieser Überschuss schon 
eingeplant.

\subsection{Ideeller Bereich}
\label{sec:ideeller_bereich}

Im ideellen Bereich wurden im Berichtszeitraum 7.092\euro{} an 
Mitgliedsbeiträgen eingenommen. An Spenden konnten, inkl. der unter 
\ref{sec:sachegebundene_spenden} detailliert benannten zweckgebundenen 
Spenden, 1.500,36\euro{} eingeworben werden, so dass sich der 
Gesamtbetrag auf 8.592,36\euro{} addiert.

Ausgegeben für den Unterhalt des Raumens -- Miete, Strom, Abschlag 
allgemeine Nebenkosten wie Wasser und Heizung sowie Telefon/Internet -- 
wurden 7.152,68\euro{}. Der Betrag beinhaltet bereits eine Rückzahlung 
von Naturstrom auf Grund der Jahresendabrechnung für Strom. Für 
sonstige Sachen im ideellen Bereich wurden 1.331,24\euro{} ausgegeben. 
Dies beinhaltet sowohl die unter Punkt \ref{sec:anschaffungen} beschriebenen 
Anschaffungen, wie auch Verbrauchsmaterial wie Toilettenpapier und 
Spühlmittel sowie geringwertige Ausstattungselemente. Als Beispiel 
seien dafür kurz Euro-Projekt-Kisten genannt. 


\subsection{Wirtschaftlicher Bereich}
\label{sec:wirschaftlicher_bereich}

Zur Versorgung der Hacker im Raum gibt es ein kleines Angebot an
Getränken und Knabbereien, das über das "`Kaffeekassen-Prinzip"'.

Im Rahmen dessen konnten im Berichtszeitraum Einnahmen in Höhe von 
4.406,49\euro{} erzielt werden. Es wurden 2.511,01\euro{} für den Einkauf 
ausgegeben. Inklusive 1,15\euro{} Zinsertrag aus dem Kautionskonto, 
wurden somit 1.893,63\euro{} als Überschuss aus nicht direkt ideelen 
Bereichen erzielt, die dem Verein zur Fianzierung des ideelen Bereich 
zur Verfügung stehen.

\subsection{Sachgebundene Spenden}
\label{sec:sachegebundene_spenden}
Wir haben für verschiedene Anschaffungen sachgebundene Spenden erhalten:

\begin{table}[h]
	\begin{tabular}{l|r|r|r}
	\textbf{Projekt} & \textbf{Eingegangen} & \textbf{Verwendet} & \textbf{Rest} \\ \hline
	Theremin & 95\euro & 0\euro & 95\euro \\
	Internet & 155\euro & 155\euro & 0\euro \\
	Labornetzteil & 233,44\euro & 234,61\euro & 0\euro{} \\
	Projektor und Zubehör & 500\euro & 413,00\euro & 87\euro \\
	\end{tabular}
\end{table}

Eine besonders große Zuwendung gab es dabei von der 
Intershop-Stiftung, die im Januar 2013 eine Spende in Höhe von 
500\euro{} getätigt hat. Diese Spende ist zweckgebunden für die 
Anschaffung eines Projektors sowie einer Leinwand gedacht. In einem 
ersten Schritt wurde dafür ein portabler Projektor für die Verwendung 
im Raum angeschafft. (Vgl. dazu auch \ref{sec:beamerkauf}.)

\section{Anschaffungen}
\label{sec:anschaffungen}

\subsection{Projektor}
\label{sec:beamerkauf}

Durch eine Spende der Intershop Stiftung (vgl. Abschnitt
\ref{sec:sachegebundene_spenden}) war es uns möglich, einen Projektor (ugs.
Beamer) für Vorträge, Workshops und ähnliche Veranstaltungen in den
Vereinsräumen anzuschaffen. Nach einer Marktstudie und Abstimmung im Plenum
wurde sich dafür entschieden, für den Erwerb ausschließlich auf die Spende
zurück zu greifen und einen Projektor des Typs \texttt{LG PB60G} im Wert von
413\euro{} anzuschaffen.  Dieser steht zur Benutzung in den Vereinsräumen
bereit.

\subsection{Labornetzteil}
\label{sec:labornetzteil}

Es wurde für die Elektronikwerkstatt ein Labornetzteil gekauft, welches 
zum allergrößten Teil über zweckgebundene Spenden finanziert wurde. Das 
Gerät kostete 234,61\euro{} und steht allen Nutzern der 
Eletronikwerkstatt zur Verfügung.

\section{Veranstaltungen}

\subsection{Regelmäßige (Vereins-)aktivitäten}

Ein großer Teil der Vereinstätigkeiten ergibt sich aus der
Bereitstellung der Infrastruktur. So haben sich regelmäßige offene Runden
etabliert, in denen themenbezogen gearbeitet wird. Für die
einzelnen Veranstaltungen haben sich freiwillige aus dem Verein
gefunden, die sich um die Organisation kümmern.

\begin{table}[h]
	\begin{tabular}{l|l}
		Name   &  Turnus \\ \hline
		Elektronikrunde   &  jeden Montag ab 19:30 Uhr\\
		Offene Runde am Dienstag   &  jeden Dienstag ab 20 Uhr\\
		Spieleabend   &  jeden ungeraden Mittwoch ab 20 Uhr\\
		Linux User Group   &  jeden geraden Donnerstag ab 19 Uhr\\
		Freifunktreffen   &  jeden ungeraden Donnerstag ab 20 Uhr\\
		Lockpicking   &  jeden ersten Freitag im Monat ab 19 Uhr\\
		Plenum   &  jeden zweiten Freitag im Monat ab 19 Uhr\\
		Kochen   &  jeden dritten Freitag im Monat ab 19 Uhr\\
		Chaoscafe / Chaostreff   &  jeden ungeraden Sonntag ab 16 Uhr\\
\end{tabular}
\end{table}

\subsubsection{Offene Runde am Dienstag}

Jeden Dienstag gibt es die (themen-)offene Runde im Raum. D.\,h. der Raum
steht zur freien Verfügung, um gemeinsam an Themen rund um
Informationstechnologie, der Computersicherheit und des
Datenschutzes zu diskutieren und zu arbeiten.

\subsubsection{Gemeinsames Kochen mit Frank und Felix}

Essen hält Leib und Seele zusammen und Kochen verbindet:
Ziel ist aber nicht nur das Zusammenkommen, sondern der kreative Umgang
beim Prozess der Zubereitung sowie die Erlangung von kulinarischen
Fähigkeiten -- oder anders gesagt: Der Bastler lebt nicht nur von
Tiefkühlpizza allein und das muss er auch irgendwoher erlernen.

Aus diesem Grund wurden durch Frank und Felix jeweils am 3. Freitag
im Monat eine gemeinsame Kochrunde in den Vereinsräumen organsiert,
zu der regelmäßig ca. 10 Hacker einen Einblick in die Welt der Töpfe
und Pfannen erhalten.

\begin{itemize}
	\item Weißkohleintopf (21.12.2012)
	\item Soljanka (18.01.2013)
	\item Möhrensuppe (15.02.2013)
	\item Kürbissuppe (22.03.2013)
	\item Fingerfood (19.04.2013)
	\item Eierkuchen (17.05.2013)
	\item Pellkartoffeln und Kräuterquark (21.06.2013)
	\item Nudelsalat (19.07.2013)
	\item Tomatensuppe (16.08.2013)
	\item Tomaten-Kichererbsen-Suppe (20.09.2013)
	\item Kulajda (15.11.2013)
\end{itemize}

Der Verein unterstützte die Veranstaltungen durch Bereitstellung der
Infrastruktur (Raum, Herd, Töpfe etc.). Die benötigten Zutaten wurden
durch die Teilnehmer selbstständig eingekauft. Überschüsse gingen an
die Spendenkasse im Raum. Im Durchschnitt nahmen ca. 10 Teilnehmer an
der Veranstaltung teil.

\subsubsection{Plenum}

Das Vereinsplenum fand an jedem 2. Freitag im Monat in den
Vereinsräumen statt.

Entscheidungen des Plenums haben dabei keinen bindenden Charakter und
wurden entsprechend zum vereinsinternen Austausch bzw.\ zur Klärung von
organisatorischen Fragen von Angesicht zu Angesicht genutzt. Das Plenum ist dabei offen für Gäste.
Die Protokolle der Treffen sind über das Wiki unter
\url{https://www.krautspace.de/hswiki:verein:plenum:start} verfügbar.


\subsubsection{Lockpicking mit Adrian}

Lockpicking, also Aufsperren von Schlössern ohne den passenden
Schlüssel, hat sich seit einiger Zeit zu einem Sport entwickelt, den
auch einige Mitglieder des Hackspace ausprobieren wollen. Dazu hat
Adrian eine monatliche Veranstaltung etabliert, in der sich die Teilnehmer
autodidaktisch und in gegenseitiger Hilfe die benötigten Fertigkeiten
beibringen. Es geht dabei explizit um die sportliche Herausforderung
und das kritische Auseinandersetzen mit den Sicherungsmechanismen
historischer und moderner Schlösser.

\subsubsection{Elektronikrunde mit Jochen}

Die Elektronikrunde trifft sich seit 2013 jeden Montag im Krautspace um
sich konzentriert in Technikprojekte vertiefen zu können. Die
Teilnehmer helfen sich gegenseitig mit Werkzeugen, Materialien und
Wissen aus, um ihre Ideen zu verwirklichen. Der Verein stellt dabei
einen großen Teil der Werkzeuge und Verbrauchsmaterialien bereit.
Bauteile für die Schaltungen wurden durch die Teilnehmer selbstständig
organisiert.

\subsubsection{Spieleabend -- Gesellschaftsspielerei}

In der Spielerunde werden regelmäßig Brett- und Kartenspiele zu einem
bestimmten vorher festgelegten Thema gespielt. Dabei liegt der Schwerpunkt
nicht auf den üblichen Partyspielen sondern bei anspruchsvollen Spielen mit
unterschiedlichen Spielkonzepten. Dabei kommen sehr viele unterschiedliche
Spiele zum Zug. Teilweise werden auch selbst entwickelte Spiele vorgestellt
und ausprobiert oder neue Spiele von Spielemessen präsentiert.

\subsubsection{Stammtisch der LUG Jena}

Der Stammtisch der Linux-User-Group Jena beschäftigt sich alle zwei Wochen
mit Themen rund um freie Software und insbesondere GNU/Linux. Es geht dabei
um den Erfahrungsaustausch und die Diskussion aktueller Entwicklungen.

\subsubsection{Freifunktreffen}

Die wachsende Freifunkgemeinschaft in Jena trifft sich alle zwei Wochen
im Krautspace, um die aktuelle Entwicklung zu besprechen und
Interessierten die Konzepte hinter Freifunk zu erklären, sowie die
Software auf und hinter den von Freifunk betriebenen Knoten zu
verbessern.

\subsection{Vorträge und Workshops}

\begin{table}[h]
\begin{tabular}{r|l}
	\textbf{Datum} & \textbf{Inhalt} \\ \hline{}
	14.01.2014 & Cryptoparty beim Fanprojekt Jena \\
	10.02.2014 & Vortrag Spannungsversorgung mit Hannes \\
	15.02.2014 & Festplatten-Crypto Workshop \\
	16.02.2014 & Lötworkshop für Kinder \\
	04.03.2014 & Vortrag zu DNSSEC mit Lutz Donnerhacke \\
	01.04.2014 & Vortrag "`Wieviel Astronomie steckt in einer Flasche Bier"' von Florian Freistetter \\
	08.04.2014 & Vortrag zum community Netzwerk dn42 von Martin \\
	10.03.2014 & Vortrag Spannungsversorgung Teil 2 mit Hannes \\
	11.03.2014 & Vortrag zu Planetarien von Severin \\
	18.03.2014 & Groovy-Workshop mit Oli \\
	29.03.2014 & Thementag "`Technik und Internet"' im Rahmen der Jugendweihe-Vorbereitung \\
	12.05.2014 & Einführung in die Programmiersprache julia durch mk
	\end{tabular}
	\caption{Liste der besonderen Vorträge und Workshops}
\end{table}

\subsubsection{Cryptoparty beim Fanprojekt Jena}

Da die Teilnehmer kaum eigene Rechner mitgebracht haben wandelte sich die
Cryptoparty zu einem Vortrag in dem wir Thunderbird mit Enigmail ausgestattet
haben, ein wenig über XMPP geredet sowie über ein paar Verhaltensregel in
sozialen Netzwerken gesprochen haben. Es war eine lockere und freie
Veranstaltung. Bei der Q&A-Session kamen noch Fragen zu Tor und verwandte
Themen.

\subsubsection{Festplatten-Crypto Workshop}

Die Teilnehmer des Workshops haben mit Hilfe von Tim, Markus und Felix
verschlüsselte Sicherungskopien ihrer Daten angelegt und danach die Fetplatten
ihrer Laptops verschlüsselt. Während des Workshops wurden zusammen mit den
Teilnehmern Probleme gelöst und Fragen zu Sicherheit und Best Practices
beantwortet. 

\section{Tätigkeitsberichte des Vorstandes}

\subsection{Jens (Konrad)}

Jens hat sich mit Folgendem beschäftigt:

\begin{itemize}
	\item Einreichen von Protokoll und Satzungsänderung der letzten Mitgliederversammlung beim Notar
    \item Vorstandstreffen bzw. Abstimmung im Vorstand per E-Mail
    \item Bearbeitung der Vereinspost
    \item Außendarstellung des Vereins
	\item Entwurf des Förderantrags für Beamer und Leinwand bei der Intershopstiftung
	\item Konzeption und Materialbeschaffung LED-Matrix
	\item Ankündigung der Kryptoparty am 25. Januar 2012:  Plakat \& Bildschirm FH
	\item Koordination und Ansprechpartner Lange Nacht der Wissenschaft, Drucken der Plakate
	\item Interview mit dem Akrützel zum Krautspace
	\item Vorstellung des Hackspace bei der Stadtführung für die Erstsemester Informatik
    \item Gesprächsleitung bei Plenen
	\item Absprache der Teilnahme des Hackspaces beim Markt der Möglichkeiten
	\item mehr erfolglose als -reiche Versuche, Referenten für einen Vortrag/Workshop zu gewinnen
	\item vergebliche Fotoaktion "Mein Lieblingshack"
    \item Mitverfassen dieses Rechenschaftsberichtes
    \item inhaltliche Vorbereitung der Mitgliederversmmlung
\end{itemize}

\subsection{Felix}

Felix hat sich in seiner Funktion als Vorstandsmitglied mit Folgendem 
beschäftigt:

\begin{itemize}
    \item Gesprächsleitung bei Plenen und anderen Veranstaltungen
    \item Erstellung und Überarbeitung von Plenumsprotokollen
    \item Schreiben von Einladungen für Plenum, Mitgliederversammlung und 
        Veranstaltungshinweise
    \item Ausgabe von Schlüsseln für das Schließsystem
    \item Unregelmäßige Vorstandstreffen, viel Kontakt und Abstimmung über 
        elektronische Kommunikationswege
    \item Mitverfassen dieses Rechenschaftsberichtes
\end{itemize}

Als Vereinsmitglied hat er sich hiermit beschäftigt: 

\begin{itemize}
    \item Unterstützung bei einigen Veranstaltungen (z.B. Kochrunden)
    \item Überarbeitung und Dokumentation der Hardware des Schließsystems
\end{itemize}

\subsection{Martin (Frank)}

Martin hat sich mit Folgendem beschäftigt:

\begin{itemize}
	\item Vereinsfinanzen
		\begin{itemize}
			\item (regelmäßig) Kassenführung und Buchhaltung --
				  ca. 300 Buchungen in 2013
			\item Drei Berichte über Kassenstand
			\item Abstimmung mit dem Finanzamt bzgl. Gemeinnützigkeit
			\item Ausstellung von vereinzelten Zuwendungsbescheinigungen
				  sowie Mitarbeit an einem Script zur vereinfachten
				  Erstellung der Bescheide
			\item Eine Kassenprüfung innerhalb des Jahres
		\end{itemize}
	\item Mitgliederverwaltung
		\begin{itemize}
			\item Normale Aufgaben der Mitgliederverwaltung
			\item Zwei Mahnläufe für säumige Mitglieder inkl.
				  Vereinsausschluss einzelner Mitglieder
		\end{itemize}
	\item Minibar aka Matekasse:
		\begin{itemize}
			\item Fahrt (mit u.a. Karsta) zu Globus, C+C, Selgros
			\item Organisation Getränkebestellung bei »Heiko«
			\item Abrechnung
			\item Lagerhaltung und Preisbestimmung
		\end{itemize}
	\item Kontakt/Treffen/Außenkommunikation
		\begin{itemize}
			\item Intershopstiftung
			\item TowerByte e.\,G. sowie einzelner Firmen innerhalb dieser
			\item FSFE e.\,V.
			\item verschiedene Hackerspaces (z.\,B. Hackerspace Erfurt, RZL) und CCC-Umfeld
			\item Fanprojekt e.\,V.
			\item Bo's Inn
			\item Notar Dr. Thomas Weikart, Finanzamt und Vereinsregister
            \item Treffen mit JenKultur (Lange Nacht der Wissenschaften)
			\item Versatel (Internetanschluss)
			\item Referat IT des StuRa FSU Jena
			\item Piratenpartei Jena
			\item Interview mit Akrützel zum Krautspace
			\item Pflege von Twitter und idendi.ca\footnote{Bis zur Umstellung auf neue Software}
			\item Terminvorschaumails auf Mailingliste
			\item Vorstellung des Krautspace beim brmlab\footnote{\url{http://brmlab.cz}} in Prag
		\end{itemize}
	\item Veranstaltungen
		\begin{itemize}
			\item Organisation »Free your Android« mit Erik Ahlers von der FSFE
			\item Mitwirkung bei Linux-Installationsparty (mit Referat IT des StuRa FSU)
			\item Organisation der »Neuland«-Fete
			\item Betreung verschiedener regelmäßiger Veranstaltungen:
				\begin{itemize}
					\item Freifunktreffen
					\item Kochrunde
					\item Stammtisch der LUG Jena
					\item Mitbetreuung der »Montagsrunde« im Rahmen der Möglichkeiten
					\item Chaoscafe
					\item LPI-Lerngruppe
				\end{itemize}
			\item Unterstützung bei weiteren Veranstaltungen wie z.B. Vorbereitungen zur LNdW
		\end{itemize}
	\item Vorantreibung Ausstattung Werkstatt
		\begin{itemize}
			\item Elektro(nik)kleinteile
			\item Labornetzteil
			\item Sortierboxen
			\item Sonstige Ausstattung
		\end{itemize}
	\item Projekte
		\begin{itemize}
			\item Theremin -- noch nicht sehr weit fortgeschritten
			\item Unterstützung u.a. von »Termintool«, Prosady-Anmeldeskript
		\end{itemize}
	\item Mitbetreung des Servers; Pflege der Inhalte auf \url{https://krautspace.de}
	\item Unregelmäßige Vorstandstreffen; viel Abstimmung per Mail
	\item Organisationen rund um »Neuland« aka Internetanschluss im Krautspace
	\item Regelmäßige Inventarpflege im Raum
	\item Leerung Postfach
	\item Schlüsselverwaltung
		\begin{itemize}
			\item Physikalische Schlüssel
			\item Schlüssel für WLAN-Schloss (aus Mitgliederbestand)
		\end{itemize}
\end{itemize}

\section{Ausblick und Herausforderungen}

\subsection{Finanzen}
\label{sec:ausblick:finanzen}

\subsubsection{Wahrung der Gemeinnützigkeit}

Für die Bestätigung der vorläufigen Gemeinnützigkeit wird es notwendig
werden, einen Rechenschaftsbericht über die Mittelherkunft und
Mittelverwendung für das Finanzamt zu erstellen. Dabei wird
insbesondere ein Augenmerk auf die satzungsgemäße Verwendung von
Vereinsmitteln in Übereinkunft mit der Abgabenordnung geworfen.

\subsubsection{Auslösung Kautionsdarlehen}
\label{sec:katrionsdarlehen}

Im August 2015 läuft die aktuelle Vereinbarung mit den Darlehensgebern
für die Kaution der Vereinsräume in der Krautgasse 26 aus. In 2012
haben fünf Personen je 333\euro{} zinsfrei dem Verein zur Verfügung gestellt.
Wenn möglich sollen diese Darlehen früher aufgelöst werden -- dazu
benötigt der Verein liquide Mittel in Höhe von 1.665\euro{}.

\subsubsection{Umstellung auf SEPA}

Im Jahr 2014 werden die aktuellen Kontoverbindungen in Deutschland auf SEPA 
umgestellt. Entsprechend müssen Veränderungen vorgenommen werden. Dies 
betrifft unter anderem:

\begin{itemize}
	\item Außenkommunikation der Bankverbindung mit IBAN
	\item Umstellung von vorhanden Lastschriftverfahren auf das neue
		  SEPA-Verfahren
\end{itemize}

\subsubsection{Überarbeitung Buchhaltung}

\subsubsection{Zuwendungsbescheinigung}

Für die Spenden und Mitgliedsbeitragszahlungen müssen sehr
wahrscheinlich für 2013 zum ersten Mal im größeren Umfang
Zuwendungsbescheinigungen ausgestellt werden. Je nach weiterer
Entwicklung in den restlichen Monaten des Geschäftsjahres kann es sich
hierbei um über 200 Buchungen handeln. Diesen müssen in einer gesetzlich
genau vorgeschrieben Form ausgestellt werden.

\subsection{Mitgliederentwicklung}

Um den Verein langfristig auf sichere Füße zu stellen und Projekte auch über
die Vereinskasse statt immer über Spenden zu finanzieren, ist eine
Mitgliederzahl von mindestens 40 bis 50 Personen anzustreben.  Einher geht
dies mit dem zusätzlichen Gewinnen von Fördermitgliedern.  Hinzu kommt die
Herausforderung, stärker als bei anderen Vereinen, die Mitgliederfluktuation
z.\,B.\ durch Wegzug auszugleichen.  Diese ist vor allem durch die relativ
hohe Anzahl von Studenten und universitätsnahen Mitgliedern bedingt.

Zur Bereicherung des Vereinslebens und zur besseren Verankerung des Vereins in
der Stadt ist auch eine Diversifizierung des Spektrums der Mitgliederschaft
wünschenswert, die derzeit zum großen Teil aus Studenten und universitätsnahen
Personen besteht.  Insbesondere sollte sich der Verein durch spezifische
Veranstaltungen und Angebote Kindern und Jugendlichen öffnen sowie Kontakte zu
Einrichtungen der Kinder- und Jugendhilfe etablieren.  Das ist auch in Bezug
auf Fördermittel aus diesem Bereich von Interesse. 

\subsection{Außenwerbung}

Der Verein ist immer noch ein Nischenverein, der in Jena keinen großen
Bekanntheitsgrad besitzt.  Auch im Hinblick auf die Mitgliedergewinnung ist es
deshalb notwendig, durch neue Veranstaltungen und aktive Außenwerbung das
Konzept des Hackspaces weiter bekannt zu machen.  Die aktuellen
(welt-)politischen Entwicklungen bieten dafür derzeit die beste Grundlage.


\end{document}


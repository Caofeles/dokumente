\documentclass[10pt,DIV16]{scrartcl}

\usepackage{color}
\usepackage{hyperref}
\definecolor{darkblue}{rgb}{0,0,.5}

\hypersetup{pdftex=true, colorlinks=true, %
	breaklinks=true, linkcolor=black, %
	urlcolor=darkblue}

\usepackage[ngerman]{babel}
\usepackage{enumitem}
\usepackage[utf8]{inputenc}
\usepackage{marvosym}

% Euro
\usepackage{eurosym}

\setitemize{itemsep=0pt}

\title{Rechenschaftsbericht}
\subtitle{Hackspace Jena e.\,V.}
\author{%
	Konrad Schöbel (Vorsitzender)\\
    Felix Kästner (Schriftführer)\\
	Frank Lanitz (Schatzmeister)
}
\date{1. Dezember 2012 -- ??.??.????}

\begin{document}

\maketitle{}

\tableofcontents{}

\newpage{}

\section{Mitgliederenwicklung}

Die Anzahl der Mitglieder konnte von 30\footnote{Stand: 1.12.2012} auf
33\footnote{Stand: 24.10.2013} erhöht werden. Leider ist eine relativ
hohe Fluktuation zu beobachten deren Gründe zum Beispiel im Wegzug aber
auch in der Inaktivität liegen. Ferner wurden xx Mitglieder
ausgeschlossen, da sie die Mitgliedsbeiträge nicht entrichtet haben.

\section{Finanzen}

\subsection{Sachgebundene Spenden}
\subsubsection{Intershop Stiftung}
\label{sec:iss-spende}
Im Januar 2013 haben wir eine Spende in Höhe von 500\euro{} der Intershop
Stiftung Jena erhalten. diese Spende ist zweckgebunden für die
Anschaffung eines Beamers sowie einer Leinwand gedacht. In einem
ersten Schritt wurde dafür ein portabler Beamer für die Verwendung
im Raum angeschafft. (Vgl. dazu auch \ref{sec:beamerkauf}.)

\section{Anschaffungen}

\subsection{Projektor}
\label{sec:beamerkauf}

Durch eine Spende (vgl. Abschnitt \ref{sec:iss-spende}) der
Intershop Stiftung, war es uns möglich, einen Projektor (ugs. Beamer) für den Raum
für Vorträge, Workshops u.s.w. anzuschaffen. Nach einer Marktstudie
und Abstimmung im Plenum wurde sich dafür entschieden, für den
Erwerb ausschließlich auf die Spende zurück zu greifen und einen
Projektor des Types \texttt{LG PB60G }im Wert von 413\euro{} an
zuschaffen. Dieser steht zur Benutzung in den Vereinsräumen bereit.

\section{Veranstaltungen}

\subsection{Regelmäßige (Vereins-)aktivitäten}

Ein großer Teil der Vereinstätigkeiten ergibt sich aus der
Bereitstellung der Infrastruktur. So haben sich regelmäßige Runden
etabliert, in denen themenbezogen gearbeitet wird.

\begin{table}[h]
	\begin{tabular}{l|l}
		Name   &  Tornus \\ \hline
		Elektronikrunde   &  jeden Montag ab 19:30 Uhr\\
		Offene Runde am Dienstag   &  jeden Dienstag ab 20 Uhr\\
		Spieleabend   &  jeden ungeraden Mittwoch ab 20 Uhr\\
		Linux User Group   &  jeden geraden Donnerstag ab 19 Uhr\\
		Freifunktreffen   &  jeden ungeraden Donnerstag ab 20 Uhr\\
		Lockpicking   &  jeden ersten Freitag im Monat ab 19 Uhr\\
		Plenum   &  jeden zweiten Freitag im Monat ab 19 Uhr\\
		Kochen   &  jeden dritten Freitag im Monat ab 19 Uhr\\
		Chaoscafe / Chaostreff   &  jeden ungeraden Sonntag ab 16 Uhr\\
\end{tabular}
\end{table}

\subsubsection{Offene Runde am Dienstag}

Jeden Dienstag gibt es die (themen-)offene Runde im Raum. D.\,h. der Raum
steht zur freien Verfügung, um gemeinsam an Themen rund um
Informationstechnologie, der Computersicherheit und des
Datenschutzes zu diskutieren und zu arbeiten.

\subsubsection{Gemeinsames Kochen mit Frank und Felix}

Ziel ist aber nicht nur das Zusammenkommen, sondern der kreative Umgang
beim Prozess der Zubereitung sowie die Erlangung von kulinarischen
Fähigkeiten -- oder anders gesagt: Der Bastler lebt nicht nur von
Tiefkühlpizza allein und das muss er auch irgendwoher können.

Aus diesem Grund wurden durch Frank und Felix jeweils am 3. Freitag
im Monat eine gemeinsame Kochrunde in den Vereinsräumen organsiert,
zu der regelmäßig ca. 10 Hacker einen Einblick in die Welt der Töpfe
und Pfannen erhalten.

\begin{itemize}
	\item Weißkohleintopf (21.12.2012)
	\item Soljanka (18.01.2013)
	\item Möhrensuppe (15.02.2013)
	\item Kürbissuppe (22.03.2013)
	\item Fingerfood (19.04.2013)
	\item Eierkuchen (17.05.2013)
	\item Pellkartoffeln und Kräuterquark (21.06.2013)
	\item Nudelsalat (19.07.2013)
	\item Tomatensuppe (16.08.2013)
	\item Tomaten-Kichererbsen-Suppe (20.09.2013)
\end{itemize}

Der Verein unterstützte die Veranstaltungen durch Bereitstellung der
Infrastruktur (Raum, Herd, Töpfe etc.). Die benötigten Zutaten wurden
durch die Teilnehmer selbstständig eingekauft. Überschüsse gingen an
die Spendenkasse im Raum. Im Dutchschnitt nahmen ca. 10 Teilnehmer an
der Veranstaltung teil.

\subsubsection{Plenum}

Das Vereinsplenum fand an jedem 2. Freitag im Monat in den
Vereinsräumen statt. Termine waren entsprechend:

\begin{itemize}
	\item 11.01.2013
	\item 08.02.2013
	\item 08.03.2013
	\item 12.04.2013
	\item 10.05.2013
	\item 14.06.2013
	\item 12.07.2013
	\item 09.08.2013
	\item 13.09.2013
	\item 11.10.2013
	\item 08.11.2013
\end{itemize}

Entscheidungen des Plenums haben dabei keinen bindenden Charakter
und wurden entsprechend zum vereinsinternen Austausch bzw. zur
Klärung von organisatorischen Fragen genutzt. Das Plenum war dabei
offen für Gäste.

\subsubsection{Lockpicking mit Adrian}

Lockpicking, also Aufsperren von Schlössern ohne den passenden Schlüssel, 
hat sich seit einiger Zeit zu einem Sport entwickelt, den auch einige 
Mitglieder des Hackspace ausprobieren wollen. Dazu hat Adrian eine monatliche 
Veranstaltung etabliert, in der die Teilnehmer autodidaktisch und in 
gegenseitiger Hilfe die benötigten Fertigkeiten beibringen. Es geht dabei 
explizit um die sportliche Herausforderung. 

\subsubsection{Elektronikrunde mit Jochen}

Die Elektronikrunde trifft sich seit 2013 jeden Montag im Krautspace um sich 
konzentriert in Technikprojekte vertiefen zu können. Die Teilnehmer helfen 
sich gegenseitig mit Werkzeugen, Materialien und Wissen aus um ihre Ideen zu 
verwirklichen. Der Verein stellt dabei einen großen Teil der Werkzeuge und 
Verbrauchsmaterialien bereit. 

\subsubsection{Spieleabend -- Gesellschaftsspielerei}

In der Spielerunde werden regelmäßig Brett- und Kartenspiele zu einem 
bestimmten vorher festgelegten Thema gespielt. Dabei liegt der Schwerpunkt 
nicht auf den üblichen Partyspielen sondern bei anspruchsvollen Spielen mit 
unterschiedlichen Spielkonzepten. Dabei kommen sehr viele unterschiedliche 
Spiele zum Zug. Teilweise werden auch selbstentwickelte Spiele vorgestellt 
und ausprobiert oder neue Spiele von Spielemessen präsentiert. 

\subsubsection{Stammtisch der LUG Jena}

Der Stammtisch der Linux-User-Group Jena beschäftigt sich alle zwei Wochen 
mit Themen rund um freie Software und insbesondere GNU/Linux. Es geht dabei 
um den Erfahrungsaustausch und die Diskussion aktueller Entwicklungen. 

\subsubsection{Freifunktreffen}

Die Wachsende Freifunkgemeinschaft in Jena trifft sich alle Zwei wochen im 
Krautspace um die aktuelle Entwicklung zu besprechen und Interessierten die 
Konzepte hinter Freifunk zu erklären. 

\subsection{Vorträge und Workshops}

\begin{table}[h]
\begin{tabular}{r|l}
	\textbf{Datum} & \textbf{Inhalt} \\ \hline{}
	05.12.2012 & Ikea-GRÖNÖ-Kassetten-Lampen-Hacken\\
	09.02.2013 & Kernel Hacking im Krautspace\\
	22.02.2013 & TPB AFK\\
	23.02.2013 & Free your Android\\
	05.03.2013 & Vortrag: Linux Blockdeviceveredelung -- Einfuehrung in LVM, DM, MD und deren Kombination\\
	20.06.2013 & Zeichnen für Hacker\\
	22.06.2013 & Krautspace Wandertag zur Sternwarte am Forst\\
	31.08.2013 & Workshop Compilerbau\\
	15.10.2013 & Wie baue ich mein eigenes ARM-basierendes Platinenlayout
	\end{tabular}
	\caption{Liste der besonderen Vorträge und Workshops}
\end{table}

\subsubsection{Linux-Kernel-Hacking im Krautspace}

Am 9.2.2013 hat Anne einen Workshop über das Erweitern der
Funktionalität des Linux-Kernels mit Hilfe von Modulen gegeben. Den
ca. 10 Anwesenden zweigte sie, wie man prinzipiell ein Modul für den
Linux-Kernel erstellt und wie diese eingbunden werden können. Im
weiteren Verlauf des Workshops wurde dann gemeinsam ein Plugin
erstellt, welches einen einfachen Zählen implementiert, der über ein
char-Device innerhalb von zum Beispiel /dev/ angesprochen werden
kann.

Der Verein unterstützte die Veranstaltung, in dem er die notwendige
Infrasturktur für den Worksho zur Verfügung stellt. Weitere
Aufwendungen sind für Verein nicht entstanden.

\subsubsection{TPB AFK}

Es wurde gemeinsam der Film \textit{TPB AFK}, eine Dokumentation
über die Gründe des Internetportals \textit {The Piratebay} und
anschlißeend darüber diskutiert.

\subsubsection{Free your Android (23.2.2013)}

Moderne Handys können viel. Doch leider versuchen die Hersteller,
die Geräte zu reinen Konsumprodukten verkommen zu lassen, Daten vom
Nutzer zu sammeln oder anderweitig die freie, und selbstbestimmte
Verwendung zu verhindern. Glücklicherweise basieren die
Android-Telefone mit dem Betriebssystem von Google auf dem
Linux-Kernel, so dass man darauf aufbauend, sein Smartphone
selbstständig mit alternativer Software bespielen kann. Da dies die
Hersteller und Provider ebenfalls nicht so gerne sehen, ist dies
aber mit ein oder zwei Hindernissen verbunden, die wir gemeinsam
überwinden wollen.

Dazu fand mit Untestützung von Erik Ahlers von der FSFE am 23.2.2013
ein Workshop statt, in dem gemeinsam Telefone mit dem freien
Betriebssystem bespielt wurden und darüber hinaus grundsätzliche
Fragen zur Durchführung weiterer Workshops zu diesem Thema
besprochen wurden. So richtete sich der Workshop an der nicht mehr
unerfahrene Nutzer

Der Verein ünterstütze die Veranstaltung durch Übernahme der
Fahrtkosten von Erik Ahlers in Höhe von 61\euro{} sowie durch die
Bereitstellung der Infrastruktur der Vereinsräume. Darüber hinaus
sind dem Verein keine weiteren Auslagen entstanden.

\subsubsection{Cryptoparty -- 25.1.2013}

\subsubsection{LPI-Lerngruppe}

In Vorbereitung von Prüfungen rund um das LPI Level 1 Zertifikat des
LPIC zu den Chemnitzer Linuxtagen fand im Zeitraum zwischen
17.1.2013 und 14.3.2013 in zweiwöchigen Abstand in den Vereinsräumen
eine Lerngruppe in Vorbereitung der Zertifizierung statt. Die Gruppe
war mit 2 bis 7 Mann unterschiedlich stark besucht. Alle die
letztendlich bei der Prüfung teilgenommen haben, haben das
Zertifikat bestanden.

Neben der Bereitstellung der Infrastruktur sind dem Verein keine
weiteren Aufwendungen entstanden.


\end{document}


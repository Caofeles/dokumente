\documentclass[10pt,DIV16]{scrartcl}

\usepackage{color}
\usepackage{hyperref}
\definecolor{darkblue}{rgb}{0,0,.5}

\hypersetup{pdftex=true, colorlinks=true, %
	breaklinks=true, linkcolor=black, %
	urlcolor=darkblue}

\usepackage[ngerman]{babel}
\usepackage{enumitem}
\usepackage[utf8]{inputenc}
\usepackage{marvosym}

\setitemize{itemsep=0pt}

\title{Rechenschaftsbericht}
\subtitle{Hackspace Jena e.\,V.}
\author{%
	Konrad Schöbel (Vorsitzender)\\
	Felix Kästner\\
	Frank Lanitz (Schatzmeister)
}
\date{1. Dezember 2012 -- ??.??.????}

\begin{document}

\maketitle{}

\tableofcontents{}

\newpage{}

\section{Veranstaltungen}

\subsection{Regelmäßige (Vereins-)aktivitäten}

Ein großer Teil der Vereinstätigkeiten ergibt sich aus der 
Bereitstellung der Infrastruktur. So haben sich regelmäßige Runden 
etabliert, in denen themenbezogen gearbeitet wird. 

\begin{table}[h]
	\begin{tabular}{l|l}
		Name   &  Tornus \\ \hline
		Offene Runde am Dienstag   &  jeden Dienstag ab 20 Uhr\\
		Linux User Group   &  jeden geraden Donnerstag ab 19 Uhr\\
		Chaoscafe / Chaostreff   &  jeden ungeraden Sonntag ab 16 Uhr\\
		Spieleabend   &  jeden ungeraden Mittwoch ab 20 Uhr\\
		Elektronikrunde   &  (montags)\\
		Lockpicking   &  jeden ersten Freitag im Monat ab 19 Uhr\\
		Plenum   &  jeden zweiten Freitag im Monat ab 19 Uhr\\
		Kochen   &  jeden dritten Freitag im Monat ab 19 Uhr\\
\end{tabular}
\end{table}

\subsubsection{Offene Runde mit \texttt{0xAFFE} und \texttt{fpunktk}}

\subsubsection{Gemeinsames Kochen mit Frank und Felix}

Ziel ist aber nicht nur das Zusammenkommen, sondern der kreative 
Umgang beim Prozess zu Zubereitung sowie die Erlangung von 
kulinarischen Fähigkeiten -- oder anders gesagt: Der Bastler lebt 
nicht nur von Tiefkühlpizza allein. 

Aus diesem Grund wurden durch Frank und Felix jeweils am 3. Freitag 
im Monat eine gemeinsame Kochrunde in den Vereinsräumen organsiert, 
zu der regelmäßig ca. 10 Hacker einen Einblick in die Welt der Töpfe 
und Pfannen erhalten.

\begin{itemize}
	\item Weißkrauteintopf (21.12.2012)
	\item Soljanka (18.1.2013)
\end{itemize}

\subsubsection{Plenum}

\subsubsection{Lockpicking mit Adrian}

\subsubsection{Elektronikrunde mit Jochen}

\subsubsection{Spieleabend -- Gesellschaftsspielerei} 
\subsection{Vorträge und Workshops}

\subsubsection{Cryptoparty -- 25.1.2013}
\subsubsection{LPI-Lerngruppe}

\end{document}


\documentclass[10pt,DIV16]{scrartcl}

\usepackage{color}
\usepackage{hyperref}
\definecolor{darkblue}{rgb}{0,0,.5}

\hypersetup{pdftex=true, colorlinks=true, %
	breaklinks=true, linkcolor=black, %
	urlcolor=darkblue}

\usepackage[ngerman]{babel}
\usepackage{enumitem}
\usepackage[utf8]{inputenc}
\usepackage{marvosym}

% Euro
\usepackage{eurosym}

\setitemize{itemsep=0pt}

\title{Rechenschaftsbericht}
\subtitle{Hackspace Jena e.\,V.}
\author{%
	Konrad Schöbel (Vorsitzender)\\
	Felix Kästner\\
	Frank Lanitz (Schatzmeister)
}
\date{1. Dezember 2012 -- ??.??.????}

\begin{document}

\maketitle{}

\tableofcontents{}

\newpage{}

\section{Finanzen}

\subsection{Sachgebundene Spenden}
\subsubsection{Intershop Stiftung}
\label{sec:iss-spende}
Im Januar 2013 haben wir eine Spende in Höhe von 500\euro{} der Intershop
Stiftung Jena erhalten. diese Spende ist zweckgebunden für die
Anschaffung eines Beamers sowie einer Leinwand gedacht. In einem
ersten Schritt wurde dafür ein portabler Beamer für die Verwendung
im Raum angeschafft. (Vgl. dazu auch \ref{sec:beamerkauf}.)

\section{Anschaffungen}

\subsection{Beamer}
\label{sec:beamerkauf}

Durch eine Spende (vgl. Abschnitt \ref{sec:iss-spende}) der
Intershop Stiftung, war es uns möglich, einen Beamer für den Raum
für Vorträge, Workshops u.s.w. anzuschaffen. Nach einer MArkstudie
und Abstimmung im Plenum wurde sich dafür entschieden, für den
Erwerb ausschließlich auf die Spende zurück zu griefen und einen
Beamer des Types LG PB60G im Wert von 413\euro{} an zuschaffen. Dieser
steht zur Benutzung in den Vereinsräumen bereit.


\section{Veranstaltungen}

\subsection{Regelmäßige (Vereins-)aktivitäten}

Ein großer Teil der Vereinstätigkeiten ergibt sich aus der
Bereitstellung der Infrastruktur. So haben sich regelmäßige Runden
etabliert, in denen themenbezogen gearbeitet wird.

\begin{table}[h]
	\begin{tabular}{l|l}
		Name   &  Tornus \\ \hline
		Offene Runde am Dienstag   &  jeden Dienstag ab 20 Uhr\\
		Linux User Group   &  jeden geraden Donnerstag ab 19 Uhr\\
		Chaoscafe / Chaostreff   &  jeden ungeraden Sonntag ab 16 Uhr\\
		Spieleabend   &  jeden ungeraden Mittwoch ab 20 Uhr\\
		Elektronikrunde   &  (montags)\\
		Lockpicking   &  jeden ersten Freitag im Monat ab 19 Uhr\\
		Plenum   &  jeden zweiten Freitag im Monat ab 19 Uhr\\
		Kochen   &  jeden dritten Freitag im Monat ab 19 Uhr\\
\end{tabular}
\end{table}

\subsubsection{Offene Runde mit \texttt{0xAFFE} und \texttt{fpunktk}}

\subsubsection{Gemeinsames Kochen mit Frank und Felix}

Ziel ist aber nicht nur das Zusammenkommen, sondern der kreative
Umgang beim Prozess zu Zubereitung sowie die Erlangung von
kulinarischen Fähigkeiten -- oder anders gesagt: Der Bastler lebt
nicht nur von Tiefkühlpizza allein.

Aus diesem Grund wurden durch Frank und Felix jeweils am 3. Freitag
im Monat eine gemeinsame Kochrunde in den Vereinsräumen organsiert,
zu der regelmäßig ca. 10 Hacker einen Einblick in die Welt der Töpfe
und Pfannen erhalten.

\begin{itemize}
	\item Weißkrauteintopf (21.12.2012)
	\item Soljanka (18.1.2013)
	\item Mähensuppe (15.2.2013)
	\item Kürbissuppe (22.3.2013)
\end{itemize}

Der Verein unterstützte die Veranstaltungen durch Bereitstellung der
Infrastruktur (Raum, Herd, Töpfe etc.). Die benötigten Zutaten
wurden durch die Teilnehmer selbstständig eingekauft. Überschüsse
gingen an die Spendenkasse im Raum.

\subsubsection{Plenum}

Das Vereinsplenum fand an jedem 2. Freitag im Monat in den
Vereinsräumen statt. Termine waren entsprechend:

\begin{itemize}
	\item 11.01.2013
	\item 08.02.2013
	\item 08.03.2013
\end{itemize}

Entscheidungen des Plenums haben dabei keinen bindenden Charakter
und wurden entsprechend zum vereinsinternen Austausch bzw. zur
Klärung von organisatorischen Fragen genutzt. Das Plenum war dabei
offen für Gäste.

\subsubsection{Lockpicking mit Adrian}

\subsubsection{Elektronikrunde mit Jochen}

\subsubsection{Spieleabend -- Gesellschaftsspielerei}
\subsection{Vorträge und Workshops}

\subsubsection{Cryptoparty -- 25.1.2013}

\subsubsection{LPI-Lerngruppe}

In Vorbereitung von Prüfungen rund um den LPI1 des LPIC zu den
Chemnitzer Linuxtagen fand im Zeitraum zwischen 17.1.2013 und
14.3.2013 in zweiwöchigen Abstand in den Vereinsräumen eine
Lerngruppe in Vorbereitung der Zertifizierung statt. Die Gruppe war
mit 2 bis 7 Mann unterschiedlich stark besucht.

\end{document}


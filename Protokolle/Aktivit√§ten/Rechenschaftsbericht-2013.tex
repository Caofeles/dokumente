\documentclass[10pt,DIV16]{scrartcl}

\usepackage{color}
\usepackage{hyperref}
\definecolor{darkblue}{rgb}{0,0,.5}

\hypersetup{pdftex=true, colorlinks=true, %
	breaklinks=true, linkcolor=black, %
	urlcolor=darkblue}

\usepackage[ngerman]{babel}
\usepackage{enumitem}
\usepackage[utf8]{inputenc}
\usepackage{marvosym}

% Euro
\usepackage{eurosym}

\setitemize{itemsep=0pt}

\title{Rechenschaftsbericht}
\subtitle{Hackspace Jena e.\,V.}
\author{%
	Konrad Schöbel (Vorsitzender)\\
    Felix Kästner (Schriftführer)\\
	Frank Lanitz (Schatzmeister)
}
\date{01.12.2012 -- 24.11.2013}

\begin{document}

\maketitle{}

\tableofcontents{}

\newpage{}

\section{Gemeinnützigkeit}

\section{Mitgliederenwicklung}

Die Anzahl der Mitglieder konnte von 30\footnote{Stand: 1.12.2012} auf
33\footnote{Stand: 24.10.2013} erhöht werden. Leider ist eine relativ
hohe Fluktuation zu beobachten deren Gründe zum Beispiel im Wegzug aber
auch in der Inaktivität liegen. Ferner wurden xx Mitglieder
ausgeschlossen, da sie die Mitgliedsbeiträge nicht entrichtet haben.

\section{Finanzen}

Zuerst eine Übersicht der Konten zum Stichtag 16.11.2013:

\begin{table}[h]
\begin{tabular}{l|r|r|r}
\textbf{Konto} & \textbf{Kontostand 1.12.2010} & \textbf{Kontostand 16.11.2013} & \textbf{Differenz} \\ \hline
Barkasse & 60,51\euro & 216,22\euro & 155,21\euro \\
Kautionskonto & 1662,00\euro & 1663,15\euro & 1,15\euro \\
Girokonto & 1471,47\euro & \euro & \euro \\ \hline
\end{tabular}
\end{table}

Insgesamt hat der Verein im Zwitraum 1.12.2012 bis 16.11.2013 Einnahmen 
von 13,000,00\euro{} erziehlt. Diese verteilen sich den ideelen Bereich 
mit 8.592,36\euro\footnote{Siehe auch Kapitel 
\ref{sec:ideeller_bereich}} sowie 4.407,64\euro{} im wirtschaftlichen 
Bereich\footnote{Siehe auch Kapitel \ref{sec:wirschaftlicher_bereich}}. 
Ausgegeben wurden insgesamt 10.994,93\euro, die sich zu 8.483,92\euro{} 
auf den ideellen Bereich sowie zu 2.511,01\euro{} wirtschaftlichen 
Bereich verteilen. Es können also 2.005,07\euro{} als monetärer 
Überschuss in die nächsten Periode übernommen werden. Mit Blick auf die 
unter Kapitel \ref{sec:ausblick:finanzen} beschrieben, bereites zu 
erwartenden Herausforderungen, ist dieser Überschuss bereits 
"verplant".

\subsection{Ideeller Bereich}
\label{sec:ideeller_bereich}


\subsection{Wirtschaftlicher Bereich}
\label{sec:wirschaftlicher_bereich}

Zur Versorgung der Hacker im Raum gibt es ein kleines Angebot an
Getränken und Knabbereien, das über das ,,Kaffeekassen-Prinzip''. 

Im Rahmen dessen konnten im Berichtszeitraum Einnahmen in Höhe von 
4.406,49\euro{} erzielt werden. Es wurden 2.511,01 für den Einkauf 
ausgegeben. Inklusive 1,15\euro{} Zinsertrag aus dem Kautionskonto, 
wurden somit 1.893,63\euro{} als Überschuss aus nicht direkt ideelen 
Bereichen erzielt, die dem Verein zur Fianzierung des ideelen Bereich 
zur Verfügung stehen.

\subsection{Sachgebundene Spenden}
Wir haben für verschiedene Anschaffungen sachgebundene Spenden erhalten:

\begin{table}[h]
	\begin{tabular}{l|r|r|r}
	\textbf{Projekt} & \textbf{Eingegangen} & \textbf{Verwendet} & \textbf{Rest} \\ \hline
	Thermin & 95\euro & 0\euro & 95\euro \\
	Internet & 155\euro & 155\euro & 0\euro \\
	Labornetzteil & 233,44\euro & 234,61\euro & 0\euro{} \\
	Beamer und Zubehör & 500\euro & 413,00\euro & 87\euro \\
	\end{tabular}
\end{table}


\subsubsection{Intershop Stiftung}
\label{sec:iss-spende}
Im Januar 2013 haben wir eine Spende in Höhe von 500\euro{} der Intershop
Stiftung Jena erhalten. Diese Spende ist zweckgebunden für die
Anschaffung eines Projektors sowie einer Leinwand gedacht. In einem
ersten Schritt wurde dafür ein portabler Projektor für die Verwendung
im Raum angeschafft. (Vgl. dazu auch \ref{sec:beamerkauf}.)


\section{Anschaffungen}


\subsection{Projektor}
\label{sec:beamerkauf}

Durch eine Spende (vgl. Abschnitt \ref{sec:iss-spende}) der
Intershop Stiftung, war es uns möglich, einen Projektor (ugs. Beamer) für den Raum
für Vorträge, Workshops u.s.w. anzuschaffen. Nach einer Marktstudie
und Abstimmung im Plenum wurde sich dafür entschieden, für den
Erwerb ausschließlich auf die Spende zurück zu greifen und einen
Projektor des Types \texttt{LG PB60G }im Wert von 413\euro{} an
zuschaffen. Dieser steht zur Benutzung in den Vereinsräumen bereit.

\subsection{Labornetzteil}
\label{sec:labornetzteil}

\section{Veranstaltungen}

\subsection{Regelmäßige (Vereins-)aktivitäten}

Ein großer Teil der Vereinstätigkeiten ergibt sich aus der
Bereitstellung der Infrastruktur. So haben sich regelmäßige Runden
etabliert, in denen themenbezogen gearbeitet wird. Für die
einzelnen Veranstaltungen haben sich freiwillige aus dem Verein
gefunden, die sich um die Organisation kümmern.

\begin{table}[h]
	\begin{tabular}{l|l}
		Name   &  Tornus \\ \hline
		Elektronikrunde   &  jeden Montag ab 19:30 Uhr\\
		Offene Runde am Dienstag   &  jeden Dienstag ab 20 Uhr\\
		Spieleabend   &  jeden ungeraden Mittwoch ab 20 Uhr\\
		Linux User Group   &  jeden geraden Donnerstag ab 19 Uhr\\
		Freifunktreffen   &  jeden ungeraden Donnerstag ab 20 Uhr\\
		Lockpicking   &  jeden ersten Freitag im Monat ab 19 Uhr\\
		Plenum   &  jeden zweiten Freitag im Monat ab 19 Uhr\\
		Kochen   &  jeden dritten Freitag im Monat ab 19 Uhr\\
		Chaoscafe / Chaostreff   &  jeden ungeraden Sonntag ab 16 Uhr\\
\end{tabular}
\end{table}

\subsubsection{Offene Runde am Dienstag}

Jeden Dienstag gibt es die (themen-)offene Runde im Raum. D.\,h. der Raum
steht zur freien Verfügung, um gemeinsam an Themen rund um
Informationstechnologie, der Computersicherheit und des
Datenschutzes zu diskutieren und zu arbeiten.

\subsubsection{Gemeinsames Kochen mit Frank und Felix}

Essen hält Leib und Seele zusammen und kochen verbindet:
Ziel ist aber nicht nur das Zusammenkommen, sondern der kreative Umgang
beim Prozess der Zubereitung sowie die Erlangung von kulinarischen
Fähigkeiten -- oder anders gesagt: Der Bastler lebt nicht nur von
Tiefkühlpizza allein und das muss er auch irgendwoher erlernen.

Aus diesem Grund wurden durch Frank und Felix jeweils am 3. Freitag
im Monat eine gemeinsame Kochrunde in den Vereinsräumen organsiert,
zu der regelmäßig ca. 10 Hacker einen Einblick in die Welt der Töpfe
und Pfannen erhalten.

\begin{itemize}
	\item Weißkohleintopf (21.12.2012)
	\item Soljanka (18.01.2013)
	\item Möhrensuppe (15.02.2013)
	\item Kürbissuppe (22.03.2013)
	\item Fingerfood (19.04.2013)
	\item Eierkuchen (17.05.2013)
	\item Pellkartoffeln und Kräuterquark (21.06.2013)
	\item Nudelsalat (19.07.2013)
	\item Tomatensuppe (16.08.2013)
	\item Tomaten-Kichererbsen-Suppe (20.09.2013)
\end{itemize}

Der Verein unterstützte die Veranstaltungen durch Bereitstellung der
Infrastruktur (Raum, Herd, Töpfe etc.). Die benötigten Zutaten wurden
durch die Teilnehmer selbstständig eingekauft. Überschüsse gingen an
die Spendenkasse im Raum. Im Durchschnitt nahmen ca. 10 Teilnehmer an
der Veranstaltung teil.

\subsubsection{Plenum}

Das Vereinsplenum fand an jedem 2. Freitag im Monat in den
Vereinsräumen statt. Termine waren entsprechend:

\begin{itemize}
	\item 11.01.2013
	\item 08.02.2013
	\item 08.03.2013
	\item 12.04.2013
	\item 10.05.2013
	\item 14.06.2013
	\item 12.07.2013
	\item 09.08.2013
	\item 13.09.2013
	\item 11.10.2013
	\item 08.11.2013
\end{itemize}

Entscheidungen des Plenums haben dabei keinen bindenden Charakter und
wurden entsprechend zum vereinsinternen Austausch bzw. zur Klärung von
organisatorischen Fragen genutzt. Das Plenum war dabei offen für Gäste.
Die Protokolle der Treffen sind über das Wiki unter
\url{https://www.krautspace.de/hswiki:verein:plenum:start} verfügbar


\subsubsection{Lockpicking mit Adrian}

Lockpicking, also Aufsperren von Schlössern ohne den passenden
Schlüssel, hat sich seit einiger Zeit zu einem Sport entwickelt, den
auch einige Mitglieder des Hackspace ausprobieren wollen. Dazu hat
Adrian eine monatliche Veranstaltung etabliert, in der sich die Teilnehmer
autodidaktisch und in gegenseitiger Hilfe die benötigten Fertigkeiten
beibringen. Es geht dabei explizit um die sportliche Herausforderung
und das kritische Auseinandersetzen mit den Sicherungsmechanismen
historischer und moderner Schlösser.

\subsubsection{Elektronikrunde mit Jochen}

Die Elektronikrunde trifft sich seit 2013 jeden Montag im Krautspace um
sich konzentriert in Technikprojekte vertiefen zu können. Die
Teilnehmer helfen sich gegenseitig mit Werkzeugen, Materialien und
Wissen aus, um ihre Ideen zu verwirklichen. Der Verein stellt dabei
einen großen Teil der Werkzeuge und Verbrauchsmaterialien bereit.
Bauteile für die Schaltungen wurden durch die Teilnehmer selbstständig
organisiert.

\subsubsection{Spieleabend -- Gesellschaftsspielerei}

In der Spielerunde werden regelmäßig Brett- und Kartenspiele zu einem
bestimmten vorher festgelegten Thema gespielt. Dabei liegt der Schwerpunkt
nicht auf den üblichen Partyspielen sondern bei anspruchsvollen Spielen mit
unterschiedlichen Spielkonzepten. Dabei kommen sehr viele unterschiedliche
Spiele zum Zug. Teilweise werden auch selbst entwickelte Spiele vorgestellt
und ausprobiert oder neue Spiele von Spielemessen präsentiert.

\subsubsection{Stammtisch der LUG Jena}

Der Stammtisch der Linux-User-Group Jena beschäftigt sich alle zwei Wochen
mit Themen rund um freie Software und insbesondere GNU/Linux. Es geht dabei
um den Erfahrungsaustausch und die Diskussion aktueller Entwicklungen.

\subsubsection{Freifunktreffen}

Die wachsende Freifunkgemeinschaft in Jena trifft sich alle zwei Wochen
im Krautspace, um die aktuelle Entwicklung zu besprechen und
Interessierten die Konzepte hinter Freifunk zu erklären, sowie die
Software auf und hinter den von Freifunk betriebene Knoten zu
verbessern.

\subsection{Vorträge und Workshops}

\begin{table}[h]
\begin{tabular}{r|l}
	\textbf{Datum} & \textbf{Inhalt} \\ \hline{}
	05.12.2012 & Ikea-GRÖNÖ-Kassetten-Lampen-Hacken\\
    25.01.2013 & Zweite Cryptoparty\\
	09.02.2013 & Linux-Kernel-Hacking im Krautspace\\
	22.02.2013 & TPB AFK\\
	23.02.2013 & Free your Android Workshop\\
	05.03.2013 & Vortrag: Linux Blockdeviceveredelung -- Einfuehrung in LVM, DM, MD und deren Kombination\\
	20.06.2013 & Zeichnen für Hacker\\
	22.06.2013 & Krautspace Wandertag zur Sternwarte am Forst\\
	31.08.2013 & Workshop Compilerbau\\
	15.10.2013 & Wie baue ich mein eigenes ARM-basierendes Platinenlayout
	\end{tabular}
	\caption{Liste der besonderen Vorträge und Workshops}
\end{table}

\subsubsection{Zweite Cryptoparty}

Am 25.01.2013 wurde in der zweiten Cryptoparty im Krautspace
Workshopartig gezeigt, wie bestimmte Software zu benutzen ist und wie
sie funktioniert. Durch qbi wurden OTR und die Einrichtung mit pidgin
sowie Tor vorgestellt und viele Fragen zum Betrieb eines Tor-Servers
beantwortet. fpunktk hat die Einrichtung von Festplattenverschlüsselung
mit TrueCrypt gezeigt.

Der Verein unterstütze die Veranstaltung durch Bereitstellung von
Infrastruktur wie Projektor und Räumlichkeiten. Zusätzliche Kosten sind
nicht entstanden.

\subsubsection{Linux-Kernel-Hacking im Krautspace}

Am 09.02.2013 hat Anne einen Workshop über das Erweitern der
Funktionalität des Linux-Kernels mit Hilfe von Modulen gegeben. Den ca.
10 Anwesenden zweigte sie, wie man prinzipiell ein Modul für den
Linux-Kernel erstellt und wie diese eingebunden werden können. Im
weiteren Verlauf des Workshops wurde dann gemeinsam ein Plugin
erstellt, welches einen einfachen Zähler implementiert, der über ein
char-Device innerhalb von zum Beispiel \texttt{/dev/} angesprochen
werden kann.

Der Verein unterstützte die Veranstaltung, in dem er die notwendige
Infrastruktur für den Workshop zur Verfügung stellt. Weitere
Aufwendungen sind für den Verein nicht entstanden.

\subsubsection{TPB AFK}

Am 22.02.2013 wurde gemeinsam der Film \textit{TPB AFK}, eine
Dokumentation über die Gründer des Internetportals \textit {The
Piratebay}, angesehen und anschließend darüber diskutiert.


\subsubsection{Free your Android Workshop (23.02.2013)}
\label{sec:free-your-android}

Moderne Mobiltelefone können viel. Doch leider versuchen die Hersteller,
die Geräte zu reinen Konsumprodukten verkommen zu lassen, Daten vom
Nutzer zu sammeln oder anderweitig die freie und selbstbestimmte
Verwendung zu verhindern. Glücklicherweise basieren die
Android-Telefone mit dem Betriebssystem von Google auf dem
Linux-Kernel, so dass man darauf aufbauend, sein Smartphone
selbstständig mit alternativer Software bespielen kann. Da dies die
Hersteller und Provider ebenfalls nicht so gerne sehen, ist es
aber mit ein oder zwei Hindernissen verbunden, die gemeinsam
überwunden werden sollten.

Dazu fand mit Unterstützung von Erik Ahlers von der FSFE am 23.02.2013
ein Workshop statt, in dem gemeinsam Telefone mit dem freien
Betriebssystem bespielt wurden und darüber hinaus grundsätzliche
Fragen zur Durchführung weiterer Workshops zu diesem Thema
besprochen wurden. So richtete sich der Workshop an nicht ganz
unerfahrene Nutzer.

Der Verein unterstütze die Veranstaltung durch Übernahme der
Fahrtkosten von Erik Ahlers in Höhe von 61\euro{} sowie durch die
Bereitstellung der Infrastruktur der Vereinsräume. Darüber hinaus
sind dem Verein keine weiteren Auslagen entstanden.

\subsubsection{Linux Blockdeviceveredelung -- Einführung in LVM, DM, MD und deren Kombination}

Jan hat am 05.03.2013 den versammelten Hackern Einblicke in das
Arbeiten mit Blockdevices unter Linux gegeben. Dabei legte er großen
Wert auf praktische Anwendung und die sinnvolle Kombination der
unterschiedlichen Techniken. Zahlreiche Fragen der Teilnehmer wurden
diskutiert und beantwortet.

\subsubsection{Zeichnen für Hacker}

Jojo hat am 20.06.2013 einen Einblick gegeben, wie er am Computer
zeichnet, welche Software und Hardware er dafür benutzt und wie diese
funktioniert. Zahlreiche Fragen wurden anhand von Beispielen und
Anekdoten beantwortet. Jojo fasste die Veranstaltung auf seine Art
zusammen:
\url{http://blog.beetlebum.de/2013/06/24/re-zeichnen-fur-geeks/}

\subsubsection{Krautspace Wandertag zur Sternwarte am Forst}

Der Ausflug zur Sternwarte begann mit Zubereitung und Verzehr von Grillgut in
Jochens Garten. Bei Einbruch der Dunkelheit wanderten wir dann zur Sternwarte
am Forst, wo uns Uwe eine Führung gab und viele Fragen beantwortete. Wir
hatten Glück mit dem Wetter und konnten einige interessante Sterne sehen.

\subsubsection{Workshop Compilerbau}

Oliver zeigte an einem Nachmittag, wie man sich einen Compiler für die
Programmiersprache Brainfuck schreibt. Dabei hielt er sich nicht lange mit
dem Erlären der Theorie auf, sondern legte darauf Wert, dass hinterher jeder
Teilnehmer ein lauffähiges Programm hat. Eine kurze Einführung in Brainfuck
gab er am Dienstag zuvor.

\subsubsection{Wie baue ich mein eigenes ARM-basierendes Platinenlayout}

Hannes hat von seinen Erfahrungen beim Bau eines ARM-Bords erzählt.
Angeschnitten wurden Themen wie Platinenlayout, Komponentenwahl,
Hardwaredebugging und auch die zugehörigen Lötarbeiten.

\subsubsection{LPI-Lerngruppe}

In Vorbereitung von Prüfungen rund um das LPI Level 1 Zertifikat des
LPIC zu den Chemnitzer Linuxtagen fand im Zeitraum zwischen
17.01.2013 und 14.03.2013 in zweiwöchigen Abstand in den Vereinsräumen
eine Lerngruppe in Vorbereitung der Zertifizierung statt. Die Gruppe
war mit 2 bis 7 Mann unterschiedlich stark besucht. Alle die
letztendlich bei der Prüfung teilgenommen haben, haben das
Zertifikat bestanden.

Neben der Bereitstellung der Infrastruktur sind dem Verein keine
weiteren Aufwendungen entstanden.

\section{Tätigkeitsberichte des Vorstandes}

\subsection{Konrad}

Konrad hat sich mit Folgendem beschäftigt:
\begin{itemize}
	\item ...
\end{itemize}

\subsection{Felix}

Felix hat sich mit Folgendem beschäftigt:

\begin{itemize}
    \item Gesprächsleitung bei Plenen und anderen Veranstaltungen
    \item Erstellung und Überarbeitung von Plenumsprotokollen
    \item Schreiben von Einladungen für Plenum, Mitgliederversammlung und 
        Veranstaltungshinweise
    \item Ausgabe von Schlüsseln für das Schließsystem
    \item Unterstützung bei einigen Veranstaltungen (z.B. Kochrunden)
    \item Unregelmäßige Vorstandstreffen, viel Kontakt und Abstimmung über 
        elektronische Kommunikationswege
    \item Mitverfassen dieses Rechenschaftsberichtes
\end{itemize}

\subsection{Frank}

Frank hat sich mit Folgendem beschäftigt:

\begin{itemize}
	\item Vereinsfinanzen
		\begin{itemize}
			\item (regelmäßig) Kassenführung und Buchhaltung --
				  ca. 300 Buchungen in 2013
			\item Drei Berichte über Kassenstand
			\item Abstimmung mit dem Finanzamt bzgl. Gemeinnützigkeit
			\item Ausstellung von vereinzelten Zuwendungsbescheinigungen
				  sowie Mitarbeit an einem Script zur vereinfachten
				  Erstellung der Bescheide
			\item Eine Kassenprüfung innerhalb des Jahres
		\end{itemize}
	\item Mitgliederverwaltung
		\begin{itemize}
			\item Normale Aufgaben der Mitgliederverwaltung
			\item Zwei Mahnläufe für säumige Mitglieder inkl.
				  Vereinsausschluss einzelner Mitglieder
		\end{itemize}
	\item Minibar aka Matekasse:
		\begin{itemize}
			\item Fahrt (mit u.a. Karsta) zu Globus, C+C, Selgros
			\item Organisation Getränkebestellung bei »Heiko«
			\item Abrechnung
			\item Lagerhaltung und Preisbestimmung
		\end{itemize}
	\item Kontakt/Treffen/Außenkommunikation
		\begin{itemize}
			\item Intershopstiftung
			\item TowerByte e.\,G. sowie einzelner Firmen innerhalb dieser
			\item FSFE e.\,V.
			\item verschiedene Hackerspaces (z.\,B. Hackerspace Erfurt, RZL) und CCC-Umfeld
			\item Fanprojekt e.\,V.
			\item Bo's Inn
			\item Notar Dr. Thomas Weikart, Finanzamt und Vereinsregister
            \item Treffen mit JenKultur (Lange Nacht der Wissenschaften)
			\item Versatel (Internetanschluss)
			\item Referat IT des StuRa FSU Jena
			\item Piratenpartei Jena
			\item Interview mit Akrützel zum Krautspace
			\item Pflege von Twitter und idendi.ca\footnote{Bis zur Umstellung auf neue Software}
			\item Terminvorschaumails auf Mailingliste
			\item Vorstellung des Krautspace beim brmlab\footnote{\url{http://brmlab.cz}} in Prag
		\end{itemize}
	\item Veranstaltungen
		\begin{itemize}
			\item Organisation »Free your Android« mit Erik Ahlers von der FSFE
			\item Mitwirkung bei Linux-Installationsparty (mit Referat IT des StuRa FSU)
			\item Organisation der »Neuland«-Fete
			\item Betreung verschiedener regelmäßiger Veranstaltungen:
				\begin{itemize}
					\item Freifunktreffen
					\item Kochrunde
					\item Stammtisch der LUG Jena
					\item Mitbetreuung der »Montagsrunde« im Rahmen der Möglichkeiten
					\item Chaoscafe
					\item LPI-Lerngruppe
				\end{itemize}
			\item Unterstützung bei weiteren Veranstaltungen wie z.B. Vorbereitungen zur LNdW
		\end{itemize}
	\item Vorantreibung Ausstattung Werkstatt
		\begin{itemize}
			\item Elektro(nik)kleinteile
			\item Labornetzteil
			\item Sortierboxen
			\item Sonstige Ausstattung
		\end{itemize}
	\item Projekte
		\begin{itemize}
			\item Theremin -- noch nicht sehr weit fortgeschritten
			\item Unterstützung u.a. von »Termintool«, Prosady-Anmeldeskript
		\end{itemize}
	\item Mitbetreung des Servers; Pflege der Inhalte auf \url{https://krautspace.de}
	\item Unregelmäßige Vorstandstreffen; viel Abstimmung per Mail
	\item Organisationen rund um »Neuland« aka Internetanschluss im Krautspace
	\item Regelmäßige Inventarpflege im Raum
	\item Leerung Postfach
	\item Schlüsselverwaltung
		\begin{itemize}
			\item Physikalische Schlüssel
			\item Schlüssel für WLAN-Schloss (aus Mitgliederbestand)
		\end{itemize}
\end{itemize}

\section{Ausblick und Herausforderungen}

\subsection{Finanzen}
\label{sec:ausblick:finanzen}

\subsubsection{Wahrung der Gemeinnützigkeit}

Für die Bestätigung der vorläufigen Gemeinnützigkeit wird es notwendig
werden, einen Rechenschaftsbericht über die Mittelherkunft und
Mittelverwendung für das Finanzamt zu erstellen. Dabei wird
insbesondere ein Augenmerk auf die satzungsgemäße Verwendung von
Vereinsmitteln in Übereinkunft mit der Abgabenordnung geworfen.

\subsubsection{Auslösung Kautionsdarlehen}
\label{sec:katrionsdarlehen}

Im August 2015 läuft die aktuelle Vereinbarung mit den Darlehensgebern
für die Kaution der Vereinsräume in der Krautgasse 26 aus. In 2012
haben fünf Mann je 333\euro{} zinsfrei dem Verein zur Verfügung gestellt.
Wenn möglich sollen diese Darlehen früher aufgelöst werden -- dazu
benötigt der Verein liquide Mittel in Höhe von 1.665\euro{}.

\subsubsection{Umstellung auf SEPA}

Im Jahr 2014 werden die aktuellen Kontoverbindungen in Deutschland auf SEPA 
umgestellt. Entsprechend müssen Veränderungen vorgenommen werden. Dies 
betrifft unter anderem:

\begin{itemize}
	\item Außenkommunikation der Bankverbindung mit IBAN
	\item Umstellung von vorhanden Lastschriftverfahren auf das neue
		  SEPA-Verfahren
\end{itemize}

\subsubsection{Überarbeitung Buchhaltung}
\subsubsection{Zuwendungsbescheinigung}

Für die Spenden und Mitgliedsbeitragszahlungen müssen sehr
wahrscheinlich für 2013 zum Ersten Mal im größeren Umfang
Zuwendungsbescheinigungen ausgestellt werden. Je nach weiterer
Entwicklung in den restlichen Monaten des Geschäftsjahres kann es sich
hierbei um über 200 Buchungen handeln. Diesen müssen in einer gesetzlich
genau vorgeschrieben Form ausgestellt werden.

\subsection{Struktur}

Jena ist eine Stadt der Studenten. Dies spiegelt sich auch in einer
relativ hohen Anzahl von Studenten und universitätsnahen Mitgliedern
wieder. Da sowohl die Zeit der Studenten an der Universität, wie auch
die der Dozenten und allgemein des wissenschaftlichen Mittelbaus
begrenzt ist, ergibt sich für den Verein die Herausforderung, stärker
als bei anderen Vereinen, die Fluktuation durch Wegzug oder
Änderung der Lebensschwerpuntke entgegen zu wirken. Dazu müssen
regelmäßig neue Mitglieder für den Verein gewonnen werden, um die
Anzahl der Schultern stabil zu halten. Einher geht dies mit dem
zusätzlichen Gewinnen von Fördermitgliedern.

\end{document}


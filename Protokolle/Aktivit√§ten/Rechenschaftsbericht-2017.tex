\documentclass[ngerman]{scrartcl}
\usepackage[utf8]{inputenc}
\usepackage[T1]{fontenc}
\usepackage{lmodern}

\usepackage{xcolor}
\usepackage{hyperref}
\definecolor{darkblue}{rgb}{0,0,.5}

\hypersetup{pdftex=true, colorlinks=true, %
	breaklinks=true, linkcolor=black, %
	urlcolor=darkblue}

\usepackage[ngerman]{babel}
%\usepackage{enumitem}
\usepackage{marvosym}

% Euro
\usepackage{eurosym}

% Einheiten und Zahlen korrekt setzen
\usepackage{siunitx}
\sisetup{locale = DE, detect-all}
\DeclareSIUnit{\EUR}{\text{\euro{}}}

\usepackage{booktabs}
\usepackage{tabularx}

\usepackage{fixltx2e}
\usepackage[final,babel]{microtype} % Verbesserung der Typographie
\usepackage{paralist}

\usepackage{bera}
\usepackage{berasans}
\usepackage{beraserif}
\usepackage{beramono}
%\setitemize{itemsep=0pt}

\title{Rechenschaftsbericht}
\subtitle{Hackspace Jena e.\,V.}
\author{%
	Jens Kubieziel (Vorsitzender)\\
	Johanna Schell (Schriftführer)\\
	Bernd Kampe (Schatzmeister)
}
\date{27.11.2016 bis 09.12.2017}

\begin{document}

\maketitle{}

\tableofcontents{}

\newpage{}

\section{Mitgliederenwicklung}

Zum Stichtag den 02.\,Dezember~2017, hat unser Verein 48~Mitglieder sowie ein Fördermitglied.
Seit der letzten Jahreshauptversammlung haben wir 6~Mitglieder begrüßt und 6~Mitglieder verabschiedet.


\section{Finanzen}

Im Zeitraum vom 26. November 2016 bis zum 02. Dezember 2017 erhielt unser Verein Einnahmen von \num{11613,70}~\euro{} und tätigte Ausgaben von \num{14843,18}~\euro{}.
Daraus ergibt sich ein Verlust von \num{3229,48}~\euro{}.


\subsection{Ideeller Bereich}
\label{sec:ideeller_bereich}

Im ideellen Bereich gab es in diesem Zeitraum folgende Einnahmen:
\begin{compactitem}
\item Mitgliedsbeiträge in Höhe von \num{6506}~\euro{}
\item \num{255}~\euro{} Spenden
\item \num{1,37}~\euro{} Zinsen. % Zu Ändern!
\end{compactitem}
Insgesamt sind das Einnahmen von \num{7818,96}~\euro{}. % Zu Ändern!

Die Ausgaben in diesem Zeitraum für Miete, Internet sowie die Abschläge für Nebenkosten betragen \num{8704,63}~\euro{}.
\num{200}~\euro{} hat das Reparier-Café für den Aufbau einer Fahrradwerkstatt an den ADFC Jena gespendet.
Für sonstige Sachen wurden \num{2303,15}~\euro{} ausgegeben.
Dies sind zum Teil Ausstattungsgegenstände und Verbrauchsmaterialien wie Visitenkarten, Reinigungsmittel, Müllbeutel usw.
Außerdem wurden Werkzeuge und Bauteile für die Werkstatt angeschafft.
Gesamt sind das Ausgaben in diesem Bereich von \num{11207,78}~\euro{}.

\subsection{Zweckbetrieb}
\label{sec:Zweckbetrieb}
Aus den Verkäufen von Getränken und Snacks ergaben sich Einnahmen von \num{3585,08}~\euro{}, wobei für \num{3058,50}~\euro{} Waren eingekauft haben.
Damit ergibt sich ein Überschuss von \num{526,58}~\euro{}, der für Finanzierungen im ideellen Bereich verwendet werden kann.

Das Reparier-Café nahm durch ihre Veranstaltungen \num{1243,62}~\euro{} ein und gab für Infomaterial und Ersatzteile \num{391,34}~\euro{} aus.
Damit ergibt sich ein Überschuss von \num{852,28}~\euro{}, der für Finanzierungen im ideellen Bereich verwendet werden kann.

Insgesamt hat damit unser Zweckbetrieb einen Gewinn von \num{1378,86}~\euro{} erwirtschaftet.

\subsection{Zweckgebundene Spenden}
\label{sec:zweckgebundene_spenden}

Die Zweckbindung für die Spenden des Theremin-Projektes in Höhe von \num{95}~\euro{} wurde nun aufgehoben.

\begin{table}[h]
        \centering
        \begin{tabular}{l|r|r|r}
        \toprule
        \textsc{Projekt} & \textsc{Eingang} & \textsc{Ausgang} & \textsc{Stand} \\
        \midrule
        Linux Presentation Day und/oder Junghacker & \num{120}~\euro{} & \num{125,03}~\euro{} & \num{0}~\euro{} \\
        Theremin & \num{0}~\euro{} & \num{95}~\euro{} & \num{0}~\euro{} \\
        Freifunk & \num{0}~\euro{} & \num{0}~\euro{} & \num{26,00}~\euro{} \\
        Reparier-Café & \num{5}~\euro{} & \num{878,89}~\euro{} & \num{0}~\euro{} \\
        Junghacker & \num{0}~\euro{} & \num{380,80}~\euro{} & \num{0}~\euro{} \\
        Tor-Relay & \num{0}~\euro{} & \num{20}~\euro{} & \num{240,00}~\euro{} \\
\bottomrule
        \end{tabular}
        \caption{Eingänge/Ausgänge Zweckgebundene Spenden}
        \label{table:spenden}
\end{table}

\subsection{Aktueller Kontostand}

\begin{table}[h!]
        \centering{}
        \begin{tabular}{l|r}
        \toprule
        \textsc{Konto} & \textsc{Kontostand} \\
        & \textsc{am 02.\,12.\,2017} \\
        \midrule
        Barkasse & \num{275,26}~\euro{} \\
        Reparier-Café Barkasse & \num{164,27}~\euro{} \\
        Kautionskonto & \num{1668,11}~\euro{} \\ % Zum Ändern!
        Girokonto & \num{1211,63}~\euro{}\\
        \bottomrule
        \end{tabular}
\caption{Übersicht der Konten}
\end{table}

\section{Veranstaltungen}

\subsection{Regelmäßige (Vereins-)aktivitäten}

Ein großer Teil der Vereinstätigkeiten ergibt sich aus der
Bereitstellung der Infrastruktur. So haben sich regelmäßige offene Runden
etabliert, in denen themenbezogen gearbeitet wird. Für die
einzelnen Veranstaltungen haben sich Freiwillige aus dem Verein
gefunden, die sich um die Organisation kümmern.

\begin{table}[h]
  \centering{}
	\begin{tabularx}{\textwidth}{l|X}
          \toprule
		\textsc{Name} & \textsc{Turnus} \\ \midrule
		Elektronikrunde & jeden Montag ab 19:30 Uhr\\
		Offene Runde am Dienstag & jeden Dienstag ab 20 Uhr\\
		Sprechstunde Informationssicherheit & jeden ersten Dienstag im Monat ab 20 Uhr, seit Oktober 2014\\ 
		Spieleabend & jeden ungeraden Mittwoch ab 20 Uhr\\
		Linux User Group & jeden geraden Donnerstag ab 19 Uhr\\
		Freifunktreffen & nach Bedarf\\
		Lockpicking & jeden ersten Freitag im Monat ab 19 Uhr, beendet seit Juli 2014\\
		Gaming-Stammtisch & jeden ersten Freitag im Monat ab 19 Uhr, seit September 2014\\
                Plenum & nach Bedarf\\
		Öffentliche Vorstandssitzung & nach Bedarf\\
		Kochen & jeden dritten Freitag im Monat, beendet seit September 2014\\
		Thuringiafurs Stammtisch & jeden dritten Samstag im Monat ab 18 Uhr\\
		Chaoscafe / Chaostreff & jeden ungeraden Sonntag ab 16 Uhr, beendet seit März 2014\\
		Reparier-Café & monatlich seit Juli 2014\\
\bottomrule
\end{tabularx}
\caption{Regelmäßige Aktivitäten}
\end{table}

\subsubsection{Elektronikrunde}

Die Elektronikrunde trifft sich seit 2013 jeden Montag im Krautspace, um
sich konzentriert in Technikprojekte vertiefen zu können. Die
Teilnehmer helfen sich gegenseitig mit Werkzeugen, Materialien und
Wissen aus, um ihre Ideen zu verwirklichen. Der Verein stellt dabei
einen großen Teil der Werkzeuge und Verbrauchsmaterialien bereit.
Bauteile für die Schaltungen wurden durch die Teilnehmer selbstständig
organisiert.

\subsubsection{Offene Runde am Dienstag}

Jeden Dienstag gibt es die (themen-)offene Runde im Raum. Der Raum
steht zur freien Verfügung, um gemeinsam an Themen rund um
Informationstechnologie, der Computersicherheit und des Datenschutzes
zu diskutieren und zu arbeiten.

\subsubsection{Sprechstunde Informationssicherheit}

Mitte des Jahres 2014 kam die Idee zu einem Cryptofreitag auf. Dabei
sollten abweichend von den Cryptoparties nicht hauptsächlich Vorträge
gehalten werden, sondern es war angedacht sich auf die Fragen der
Besucher zu konzentrieren.  Da die potentiellen Betreuer Freitags
nicht verfügbar sind, wurde dann eine Sprechstunde für einen Dienstag
im Monat konzipiert.  Das Ziel der Veranstaltung ist es die Fragen der
Besucher zu den Themen Verschlüsselung, Privatsphäre und
Datensicherheit zu beantworten.

\subsubsection{Spieleabend -- Gesellschaftsspielerei}

In der Spielerunde werden regelmäßig Brett- und Kartenspiele zu einem
bestimmten vorher festgelegten Thema gespielt. Dabei liegt der
Schwerpunkt nicht auf den üblichen Partyspielen, sondern bei
anspruchsvollen Spielen mit unterschiedlichen Spielkonzepten. Dabei
kommen sehr viele unterschiedliche Spiele zum Zug. Teilweise werden
auch selbst entwickelte Spiele vorgestellt und ausprobiert oder neue
Spiele von Spielemessen präsentiert.

\subsubsection{Stammtisch der LUG Jena}

Der Stammtisch der Linux-User-Group Jena beschäftigt sich alle zwei
Wochen mit Themen rund um freie Software und insbesondere
GNU/Linux. Es geht dabei um den Erfahrungsaustausch und die Diskussion
aktueller Entwicklungen.

\subsubsection{Freifunktreffen}

Die wachsende Freifunkgemeinschaft in Jena trifft sich unregelmäßig im
Krautspace, um die aktuelle Entwicklung zu besprechen und
Interessierten die Konzepte hinter Freifunk zu erklären, sowie die
Software auf und hinter den von Freifunk betriebenen Knoten zu
verbessern.

\subsubsection{Gaming-Stammtisch}

Beim Gamingstammtisch geht es um Computerspiele — egal auf welcher
Plattform, ob gekauft oder selbst geschrieben. Die Schwerpunkte sind
Game Design und die Auswirkungen des Spielens auf Spieler und
Gesellschaft.

\subsubsection{Plenum bzw. offene Vorstandssitzung}

Das Plenum hat sich im Jahr 2015 von einer regelmäßigen Veranstaltung
zu einer Bedarfsveranstaltung geändert. Im Jahr 2016 wurden, um dem
Plenum etwas Leben einzuhauchen, die Vorstandssitzungen in der
öffentlichkeit abgehalten. Leider hat dies nicht das gewünschte
Interesse nach sich gezogen und die öffentliche Vorstandssitzung ist
eingeschlafen.

\subsubsection{Reparier-Café}

Seit Mai 2014 hat eine kleine Gruppe außerhalb des Hackspace',
angefangen ein Reparier-Café zu organisieren. Dabei geht es darum,
nicht mehr funktionierende Gegenstände in Eigenregie zu reparieren.

Da die Idee auch unter Mitgliedern des Vereins viel Zustimmung fand,
haben sich einige Mitglieder daran beteiligt.

Das erste Café fand am 31.\,Juli~2014 in den Vereinsräumen statt und
war sehr gut besucht.  Später ist das Reparier-Café ein offizieller
Teil des Vereins geworden. Jeweils zum Monatsende sind alle
eingeladen, eigene Gegenstände zu reparieren oder anderen bei der
Reparatur zu unterstützen.

\subsubsection{Junghackertage}

Auf der letzten Vollversammlung kam die Idee auf, einen
Jungerhackertag zu veranstalten. Inhalt dieser Veranstaltung soll
sein, Kindern und Jungendlichen die Welt der Elektrotechnik und
Programmierung näher zu bringen.

Der erste Junghackertag fand am 23.04.2016 statt und findet seit dem in
unregelmäßig Intervallen an Samstagen statt.

\section{Tätigkeitsberichte des Vorstandes}

\subsection{Tim}

Tim hat sich mit Folgendem beschäftigt:

\begin{compactitem}
\item Kommunikation mit der Ethikbank um Zugriff auf das Bankkonto zu erlangen.  
\item Vorstandstreffen bzw. Abstimmung im Vorstand per E-Mail
\item Außendarstellung des Vereins
\item Bestrebungen nach neuen Räumlichkeiten koordiniert 
\item Diskussion zu anderen Vereinsaktivitäten angeregt.
\item Beantwortung diverser E-Mails an die Office"=Adresse
\item SSL-Zertifikate des Vereins auf Let's Encrypt umgestellt.
\item Mitbetreutung des Servers svr0.
\item Betreuung des XMPP-Servers auf dem svr0.
\end{compactitem}

\subsection{Johanna}

Johanna hat sich in ihrer Funktion als Vorstandsmitglied mit Folgendem 
beschäftigt:

\begin{compactitem}
    \item Mitorganisation und Teilnahme an Vorstandssitzungen
    \item Führen der Protokolle der Vorstandssitzungen
    \item Mitorganisation der offenen Vorstandssitzungen
    \item Führen der Protokolle der offenen Vorstandssitzungen
    \item Wahrnehmen von Terminen beim Notar
    \item Verteilung der E-Mails
    \item vorstandsinterne Absprachen und Diskussionen 
    \item Mitglieder an ihre Mitgliedsbeiträge erinnern
\end{compactitem}

\subsection{Bernd}
Bernd hat sich als Schatzmeister und Vorstandsmitglied mit Folgendem beschäftigt:
\begin{compactitem}
	\item Finanzverwaltung und Planung
	\begin{compactitem}
		\item Buchführung
		\item Rechnungen bezahlen
		\item Unterlagen abheften
		\item Kassenprüfung
		\item Zuwendungsbescheinigungen erstellt
	\end{compactitem}
	\item Mitgliederverwaltung
	\begin{compactitem}
		\item Neue Mitglieder angeworben
		\item Mitglieder persönlich begrüßt und in die Verwaltung aufgenommen
		\item Johanna über säumige Mitglieder informiert
		\item Fragen von Mitglieder bezüglich ihren Beiträgen beantwortet
	\end{compactitem}
	\item Bar mit Getränken und Süßigkeiten:
	\begin{compactitem}
		\item Planung der Warenbeschaffung
		\item Getränkebestellung bei Heiko Wackernagel
		\item Absprachen mit Verantwortlichen
	\end{compactitem}
	\item Erstellung des Rechenschaftsberichts
	\item Verwaltung von Post und Postfach
	\item Treffen und Absprachen im Vorstand
	\item Absprachen mit Projektverantwortlichen
\end{compactitem}
\end{document}

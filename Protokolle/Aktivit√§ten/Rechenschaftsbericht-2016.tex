\documentclass[ngerman]{scrartcl}
\usepackage[utf8]{inputenc}
\usepackage[T1]{fontenc}
\usepackage{lmodern}

\usepackage{xcolor}
\usepackage{hyperref}
\definecolor{darkblue}{rgb}{0,0,.5}

\hypersetup{pdftex=true, colorlinks=true, %
	breaklinks=true, linkcolor=black, %
	urlcolor=darkblue}

\usepackage[ngerman]{babel}
%\usepackage{enumitem}
\usepackage{marvosym}

% Euro
\usepackage{eurosym}

% Einheiten und Zahlen korrekt setzen
\usepackage{siunitx}
\sisetup{locale = DE, detect-all}
\DeclareSIUnit{\EUR}{\text{\euro{}}}

\usepackage{booktabs}
\usepackage{tabularx}

\usepackage{fixltx2e}
\usepackage[final,babel]{microtype} % Verbesserung der Typographie
\usepackage{paralist}

\usepackage{bera}
\usepackage{berasans}
\usepackage{beraserif}
\usepackage{beramono}
%\setitemize{itemsep=0pt}

\title{Rechenschaftsbericht}
\subtitle{Hackspace Jena e.\,V.}
\author{%
	Tim Schumacher (Vorsitzender)\\
	Johanna Schell (Schriftführer)\\
	Adrian Pauli (Schatzmeister)
}
\date{14.11.2015 bis 27.11.2016}

\begin{document}

\maketitle{}

\tableofcontents{}

\newpage{}

\section{Mitgliederenwicklung}

Aktuell, d.\,h. zum Stichtag am 25.\,November~2016, hat der Verein 48~Mitglieder. Am 13.\,November~2015 waren es 44~Mitglieder. Durch einen Software-Irrtum wurde letztes Jahr fälschlicher Weise 45 Mitglieder vermerkt. In diesem Zeitraum von einem Jahr haben wir 5~Mitglieder begrüßt und ein Mitglieder verabschiedet. Somit ergibt sich ein Mitgliederzuwachs von 4~Mitgliedern.

\section{Finanzen}

Im Jahr~2014 erhielt der Verein Einnahmen von \num{11848,00}~\euro{} und tätigte Ausgaben von \num{10483,35}~\euro{}.
Daraus ergibt sich ein Überschuss von \num{365,65}~\euro{}.

\subsection{Ideeller Bereich}
\label{sec:ideeller_bereich}

Im ideellen Bereich gab es in diesem Zeitraum folgende Einnahmen:
\begin{compactitem}
\item Mitgliedsbeiträge in Höhe von \num{6932}~\euro{}
\item \num{614,46}~\euro{} Spenden (davon \num{301,77}~\euro{} bei Veranstaltungen des Reparier-Cafés eingenommen)
\item \num{271,13}~\euro{} Gutschrift aus der Betriebskostenabrechnung
\item \num{1,37}~\euro{} Zinsen.
\end{compactitem}
Insgesamt sind das Einnahmen von \num{7818,96}~\euro{}.

Die Ausgaben in diesem Zeitraum für Miete, Internet sowie die Abschläge für Nebenkosten betragen \num{7251,92}~\euro{}.
Für Kontoführungsgebühren, Rundfunkbeträge und Versicherung wurden in diesem Zeitraum \num{241.19}~\euro{} ausgegeben.
Für sonstige Sachen wurden \num{446,79}~\euro{} ausgegeben. Dies sind zum Teil Ausstattungsgegenstände und Verbrauchsmaterialien wie Visitenkarten, Reinigungsmittel, Müllbeutel usw.
Gesamt sind das Ausgaben in diesem Bereich von \num{7939,90}~\euro{}.

\subsection{Zweckbetrieb}
\label{sec:Zweckbetrieb}
Aus den Verkäufen von Getränken und Süßigkeiten ergaben sich Einnahmen von \num{3752,73}~\euro{}, wobei für \num{2019,99}~\euro{} Waren eingekauft haben.
Damit ergibt sich ein Überschuss von \num{1732,74}~\euro{}, der für Finanzierungen im ideellen Bereich verwendet werden kann.

Auch kauften wir Elektronikbauteile für \num{429,41}~\euro{} die in unser Werkstat zum verkauf bereitstehen; dadurch nahmen wir \num{276,31}~\euro{} ein.
Das Reparier-Café kaufte Nähmaterial für \num{21,90}~\euro{} ein.

\subsection{Zweckgebundene Spenden}
\label{sec:zweckgebundene_spenden}
Wir besitzen zweckgebundene Spenden für Projekte. Eine Aufstellung befindet sich in \autoref{table:spenden}.
In diesem Zeitraum erhielten wir \num{72,50}~\euro{} für das Freifunkprojekt.

In diesem Bereich ergaben sich seit Jahresbeginn größere Entwicklungen.
So erhielt das Reparier-Café von Sunfried e.\,V. eine Spende von \num{1000}~\euro{} und gewann den Jenaer Umweltpreis und erhielt dadurch weitere \num{1000}~\euro{}.
Die Thüringer Ehrenamtsstiftung förderte mit \num{277}~\euro{} die ehrenamtlichen Tätigkeiten im Reparier-Café.

Seit Anfang des Jahres haben wir das Projekt Tor-Relay; für diese sind Spenden in der Höhe von \num{260}~\euro{} eingegangen.

\begin{table}[h]
	\centering
	\begin{tabular}{l|r|r}
	\toprule
	\textsc{Projekt} & \textsc{Eingegangen in 2014} & \textsc{Eingegangen 1.1.2115 – 13.11.2015} \\
	\midrule
	Theremin & \num{0}~\euro{} & \num{0}~\euro{} \\
	Freifunk & \num{72,50}~\euro{} & \num{0}~\euro{} \\
	Projektor und Zubehör & \num{0}~\euro{} & \num{0}~\euro{} \\
	Reparier-Café & \num{301,77}~\euro{} & \num{3214,38}~\euro{} \\
	Tor-Relay & - & \num{260,00}~\euro{} \\
\bottomrule
	\end{tabular}
	\caption{Eingang Zweckgebundene Spenden}
	\label{table:spenden:ein}
\end{table}

Von diese Spenden haben wir eine tragbare Leinwand finanziert und das Reparier-Café hat Werkzeuge angeschafft und eine Austausch fahrt zum bundesweiten Vernetzungstreffen des Netzwerks Reparatur-Initiativen unternommen.

\begin{table}[h]
	\centering
	\begin{tabular}{l|r|r}
	\toprule
	\textsc{Projekt} & \textsc{Verwendet in 2014} & \textsc{Verwendet 1.1.2015 – 13.11.2015} \\
	\midrule
	Theremin & \num{0}~\euro{} & \num{0}~\euro{} \\
	Freifunk & \num{0}~\euro{} & \num{46,50}~\euro{} \\
	Projektor und Zubehör & \num{0}~\euro{} & \num{76,99}~\euro{} \\
	Reparier-Café & \num{31,42}~\euro{} & \num{2313,45}~\euro{} \\
	Tor-Relay & - & \num{0}~\euro{} \\
\bottomrule
	\end{tabular}
	\caption{Ausgaben Zweckgebundene Spenden}
	\label{table:spenden:aus}
\end{table}

Gesamt stehen folgende Spenden zur Verwendung bereit:
\begin{table}[h]
	\centering
	\begin{tabular}{l|r}
	\toprule
	\textsc{Projekt} & \textsc{Verfügbar (13.11.2015)} \\
	\midrule
	Theremin & \num{95,00}~\euro{} \\
	Freifunk & \num{26,00}~\euro{} \\
	Projektor und Zubehör & \num{0}~\euro{} \\
	Reparier-Café & \num{1247,56}~\euro{} \\
	Tor-Relay & \num{260,00}~\euro{} \\
\bottomrule
	\end{tabular}
	\caption{Verfügbare Zweckgebundene Spenden}
	\label{table:spenden}
\end{table}

\subsection{Aktuelle Entwicklung}
An die fünf Personen, die uns 2012 ein Darlehen für die Hinterlegung der Kaution von je \num{333}~\euro{} gegeben haben, haben wir die Darlehen vollständig zurückbezahlt.

Damit haben wir aktuell kein Fremdkapital und somit auch keine Rückzahlung als Belastung.

\begin{table}[h!]
	\centering{}
	\begin{tabular}{l|r}
	\toprule
	\textsc{Konto} & \textsc{Kontostand} \\
	& \textsc{am 11.\,11.\,2015} \\
	\midrule
	Barkasse & \num{382,89}~\euro{} \\
	Reparier-Café Barkasse & \num{163,78}~\euro{} \\
	Kautionskonto & \num{1667,59}~\euro{} \\
	Girokonto & \num{3935,68}~\euro{}\\
	\bottomrule
	\end{tabular}
\caption{Übersicht der Konten}
\end{table}

\section{Veranstaltungen}

\subsection{Regelmäßige (Vereins-)aktivitäten}

Ein großer Teil der Vereinstätigkeiten ergibt sich aus der
Bereitstellung der Infrastruktur. So haben sich regelmäßige offene Runden
etabliert, in denen themenbezogen gearbeitet wird. Für die
einzelnen Veranstaltungen haben sich Freiwillige aus dem Verein
gefunden, die sich um die Organisation kümmern.

\begin{table}[h]
  \centering{}
	\begin{tabularx}{\textwidth}{l|X}
          \toprule
		\textsc{Name} & \textsc{Turnus} \\ \midrule
		Elektronikrunde & jeden Montag ab 19:30 Uhr\\
		Offene Runde am Dienstag & jeden Dienstag ab 20 Uhr\\
		Sprechstunde Informationssicherheit & jeden ersten Dienstag im Monat ab 20 Uhr, seit Oktober 2014\\ 
		Spieleabend & jeden ungeraden Mittwoch ab 20 Uhr\\
		Linux User Group & jeden geraden Donnerstag ab 19 Uhr\\
		Freifunktreffen & Nach Bedarf\\
		Lockpicking & jeden ersten Freitag im Monat ab 19 Uhr, beendet seit Juli 2014\\
		Gaming-Stammtisch & jeden ersten Freitag im Monat ab 19 Uhr, seit September 2014\\
                Plenum & Nach Bedarf\\
		Öffentliche Vorstandssitzung & Nach Bedarf\\
		Kochen & jeden dritten Freitag im Monat, beendet seit September 2014\\
		Thuringiafurs Stammtisch & jeden dritten Samstag im Monat ab 14 Uhr\\
		Chaoscafe / Chaostreff & jeden ungeraden Sonntag ab 16 Uhr, beendet seit März 2014\\
		Reparier-Café & monatlich seit Juli 2014\\
\bottomrule
\end{tabularx}
\caption{Regelmäßige Aktivitäten}
\end{table}

\subsubsection{Elektronikrunde}

Die Elektronikrunde trifft sich seit 2013 jeden Montag im Krautspace, um
sich konzentriert in Technikprojekte vertiefen zu können. Die
Teilnehmer helfen sich gegenseitig mit Werkzeugen, Materialien und
Wissen aus, um ihre Ideen zu verwirklichen. Der Verein stellt dabei
einen großen Teil der Werkzeuge und Verbrauchsmaterialien bereit.
Bauteile für die Schaltungen wurden durch die Teilnehmer selbstständig
organisiert.

\subsubsection{Offene Runde am Dienstag}

Jeden Dienstag gibt es die (themen-)offene Runde im Raum. Der Raum
steht zur freien Verfügung, um gemeinsam an Themen rund um
Informationstechnologie, der Computersicherheit und des Datenschutzes
zu diskutieren und zu arbeiten.

\subsubsection{Sprechstunde Informationssicherheit}

Mitte des Jahres 2014 kam die Idee zu einem Cryptofreitag auf. Dabei
sollten abweichend von den Cryptoparties nicht hauptsächlich Vorträge
gehalten werden, sondern es war angedacht sich auf die Fragen der
Besucher zu konzentrieren.  Da die potentiellen Betreuer Freitags
nicht verfügbar sind, wurde dann eine Sprechstunde für einen Dienstag
im Monat konzipiert.  Das Ziel der Veranstaltung ist es die Fragen der
Besucher zu den Themen Verschlüsselung, Privatsphäre und
Datensicherheit zu beantworten.

\subsubsection{Spieleabend -- Gesellschaftsspielerei}

In der Spielerunde werden regelmäßig Brett- und Kartenspiele zu einem
bestimmten vorher festgelegten Thema gespielt. Dabei liegt der
Schwerpunkt nicht auf den üblichen Partyspielen, sondern bei
anspruchsvollen Spielen mit unterschiedlichen Spielkonzepten. Dabei
kommen sehr viele unterschiedliche Spiele zum Zug. Teilweise werden
auch selbst entwickelte Spiele vorgestellt und ausprobiert oder neue
Spiele von Spielemessen präsentiert.

\subsubsection{Stammtisch der LUG Jena}

Der Stammtisch der Linux-User-Group Jena beschäftigt sich alle zwei
Wochen mit Themen rund um freie Software und insbesondere
GNU/Linux. Es geht dabei um den Erfahrungsaustausch und die Diskussion
aktueller Entwicklungen.

\subsubsection{Freifunktreffen}

Die wachsende Freifunkgemeinschaft in Jena trifft sich unregelmäßig im
Krautspace, um die aktuelle Entwicklung zu besprechen und
Interessierten die Konzepte hinter Freifunk zu erklären, sowie die
Software auf und hinter den von Freifunk betriebenen Knoten zu
verbessern.

\subsubsection{Gaming-Stammtisch}

Beim Gamingstammtisch geht es um Computerspiele — egal auf welcher
Plattform, ob gekauft oder selbst geschrieben. Die Schwerpunkte sind
Game Design und die Auswirkungen des Spielens auf Spieler und
Gesellschaft.

\subsubsection{Plenum bzw offene Vorstandssitzung}

Das Plenum hat sich im Jahr 2015 von einer regelmäßigen Veranstaltung
zu einer Bedarfsveranstaltung geändert. Im Jahr 2016 wurde um dem
Plenum etwas leben einzuhauchen, die Vorstandssitzungen in der
öffentlichkeit abgehalten. Leider hat dies nicht das gewünschte
Interesse nach sich gezogen und die öffentliche Vorstandssitzung ist
eingeschlafen.

\subsubsection{Reparier-Café}

Seit Mai 2014 hat eine kleine Gruppe außerhalb des Hackspace',
angefangen ein Reparier-Café zu organisieren. Dabei geht es darum,
nicht mehr funktionierende Gegenstände in Eigenregie zu reparieren.

Da die Idee auch unter Mitgliedern des Vereins viel Zustimmung fand,
haben sich einige Mitglieder daran beteiligt.

Das erste Café fand am 31.\,Juli~2014 in den Vereinsräumen statt und
war sehr gut besucht.  Später ist das Reparier-Café ein offizieller
Teil des Vereins geworden. Jeweils zum Monatsende sind alle
eingeladen, eigene Gegenstände zu reparieren oder anderen bei der
Reparatur zu unterstützen.

\subsubsection{Junghackertage}

Auf der letzten Vollversammlung kam die Idee auf, einen
Jungerhackertag zu veranstalten. Inhalt dieser Veranstaltung soll
sein, Kindern und Jungendlichen die Welt der Elektrotechnik und
Programmierung näher zu bringen.

Der erste Junghackertag fand am xxx statt und findet seit dem in
unregelmäßig Intervallen an Samstagen statt.


\subsection{Vorträge und Workshops}

\begin{table}[h]
  \centering{}
  \begin{tabularx}{\textwidth}{l|X}
	\textbf{Datum} & \textbf{Inhalt} \\ \midrule
	30.\,01.\,2015 & Wiki Hackathon \\
        06.\,02.\,2015 & Themenabend Bewegungssteuerung \\
	19.\,03.\,2015 & Diskussion über Anforderungen an Vereinsräumlichkeiten \\
	23.\,03.\,2015 & Projekt CNC-Maschine \\
	25.\,04.\,2015 & Debian Jessie Release Party \\
        07.\,04.\,2015 & Hands on DNSSEC \\
	09.\,04.\,2015 & Besichtigung der Räume, vom Freiraum e.V. \\
	21.\,05.\,2015 & Investitionsplenum \\
	18.\,06.\,2015 & Projektvorstellung consearch \\
	20.\,10.\,2015 & Workshop Sicherheit im Internet \\
	14.\,10.\,2015 & Erstsemester-Rally \\
        15.\,10.\,2015 & NixOS \\
	14.\,11.\,2015 & Linux Presentation Day 2016 \\
\bottomrule
	\end{tabularx}
	\caption{Liste der besonderen Vorträge und Workshops}
\end{table}

\section{Tätigkeitsberichte des Vorstandes}

\subsection{Tim}

Tim hat sich mit Folgendem beschäftigt:

\begin{compactitem}
\item Vorstandstreffen bzw. Abstimmung im Vorstand per E-Mail
\item Außendarstellung des Vereins
\item Bestrebungen nach neuen Räumlichkeiten koordiniert 
\item Anschaffung der Domain kraut.space
\item Einführung von DNSSEC für die Domains des Vereins.
\item Einführung von OTRS zur Vorstandskommunikation
\item Diskussion zu anderen Vereinsaktivitäten angeregt. Optionen
  \begin{itemize}
  \item Lötworkshop für Kinder
  \item Sicherheitsthemen
  \end{itemize} (Ergebnislos)
\item Beantwortung diverser E-Mails an die Office"=Adresse
\item Twitter (@KrautspaceRaumStatus) und Quitter
  (https://quitter.is/KrautspaceRaumStatus) Account für den Raumstatus
  implementiert.
\item Versand der Einladungen zum Lötworkshop und Beantwortung von
  Fragen
\item Neues Zertifikat für die Webseite bei StartSSL organisiert.
\item Teilnahme bei der Preisverleihung des Sunfried Preises für das Reparier Cafés.
\item Mitbetreutung des Servers svr0.
\item Betreuung des XMPP-Servers auf dem svr0.
\end{compactitem}

\subsection{Katja}

Johanna hat sich in ihrer Funktion als Vorstandsmitglied mit Folgendem 
beschäftigt:

\begin{compactitem}
    \item Mitorganisation und Teilnahme an Vorstandssitzungen
    \item Führen der Protokolle der Vorstandssitzungen
    \item Mitorganisation der offenen Vorstandssitzungen
    \item Führen der Protokolle der offenen Vorstandssitzungen
    \item Wahrnehmen von Terminen beim Notar
    \item Verteilung der E-Mails
    \item vorstandsinterne Absprachen und Diskussionen 
    \item Mitglieder an ihre Mitgliedsbeiträge erinnern
\end{compactitem}

\subsection{Martin}
Martin hat sich als Schatzmeister und Vorstandsmitglied mit Folgendem beschäftigt:
\begin{compactitem}
	\item Finanzverwaltung und Planung
	\begin{compactitem}
		\item Buchführung
		\item Rechnungen bezahlen und erstellen
		\item Unterlagen abheften
		\item regelmäßige Kassenprüfungen
		\item vier Finanz-Berichte geschrieben
		\item Zuwendungsbescheinigungen erstellt
	\end{compactitem}
	\item Mitgliederverwaltung
	\begin{compactitem}
		\item Mitglieder durch E-Mail begrüßt und verabschiedet
		\item Mitglieder erinnert, ihre Beiträge zu zahlen
		\item Fragen von Mitglieder bezüglich ihren Beiträgen beantwortet
	\end{compactitem}
	\item Bar mit Getränken und Süßigkeiten:
	\begin{compactitem}
		\item Planung der Warenbeschaffung
		\item Getränkebestellung bei Heiko Wackernagel
		\item Absprachen mit Verantwortliche
	\end{compactitem}
	\item Auf der Mailingliste diskutiert, angeregt und Situationen geschildert
	\item Planung und Einladung zu Mitgliederversammlungen
	\item Erstellung dieses Rechenschaftsberichts
	\item Leerung von Briefkasten und Postfach
	\item Regelmäßige Vorstandstreffen
	\item Absprachen mit Projekten
	\item Station auf der Erstsemester Stadtrally betreut
	\item Mitbetreuung der Server/Infrastruktur
\end{compactitem}

Als Vereinsmitglied hat Martin sich mit folgenden beschäftigt: 
\begin{compactitem}
	\item Anbindung an das dn42 (dezentrales Community Netzwerk)
	\item Administration des (Kabel-)Netzwerks im Krautspace
	\item Planung und Einrichtung des neuen Routers
	\item Inbetriebnahme des gespendeten Switches
	\item Mitbetreuung des Servers
	\item Weiterentwicklung der Status-Ampel
\end{compactitem}

\end{document}

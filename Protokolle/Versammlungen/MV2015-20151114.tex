\documentclass[ngerman]{scrartcl}
\usepackage[utf8]{inputenc}
\usepackage{lmodern}
\usepackage{courier}
\usepackage[scaled=.92]{helvet}
\usepackage[T1]{fontenc}
\usepackage[ngerman]{babel}

\usepackage{minutes}
\usepackage{eurosym}
\usepackage{tabularx}

\begin{document}
\begin{Protokoll}{Protokoll der ordentlichen Mitgliederversammlung 2015 des Hackspace Jena e.\,V.}
\protokollant{Tim Schumacher}
\teilnehmer{Zu Beginn: 15 Vereinsmitglieder}
\gaeste{ein Gast}
\moderation{Bernd Kampe}

\sitzungsdatum{14.\,November~2014}
\sitzungsbeginn{15:20 Uhr}
\sitzungsende{18:29 Uhr}
\sitzungsort{Krautspace, Krautgasse~26, 07743~Jena}

\protokollKopf{}

\topic{Begrüßung}

Vorstandsvorsitzender Tim Schumacher und Schatzmeister Martin Neß
begrüßen die Anwesenden. Es befinden sich 15 ordentliche Mitglieder
sowie ein Gast im Raum. Martin Neß stell fest, dass die Versammlung
beschlussfähig ist, da laut Satzung mindestens zur Erfüllung der
23\%-Regelung 10~Mitglieder anwesend sein müssen.


\topic{Wahl des Versammlungsleiters \& Protokollführers}

  Einstimmig wird Bernd Kampe in einer offenen Wahl durch
  Handzeichen zum Versammlungsleiter gewählt.

  Tim Schumacher wird in einer offenen Wahl durch Handzeichen
  einstimmig zum Protokollführer gewählt.

\topic{Abstimmung der Tagesordnung}

Die Tagesordnung wird einstimmig durch offene Wahl mit Handzeichen
angenommen.

\topic{Feststellung ordentliche Einladung}

Martin berichtet, dass die Einladung für die Versammlung mehr als
14~Tage im Voraus -- am 30.\,10.\,2015 gegen 17:43~Uhr -- an alle
Mitglieder per E-Mail sowie als Kopie an die öffentliche Mailingliste
versendet wurde. Die Korrektheit der Einladung wird von der
Versammlung einstimmig durch Handzeichen bestätigt.

\topic{Genehmigung des Protokolls der Mitgliederversammlung 2014}

Das Protokoll Mitgliederversammlung des Jahres 2014 lag den
Mitgliedern vor und wird in offener Wahl angenommen.

\begin{Abstimmung}
  \abstimmung{Ergebnis der Abstimmung}{15}{0}{0}[Das Protokoll der ordentlichen Mitgliederversammlung 2014 wird angenommen]
\end{Abstimmung}

\topic{Genehmigung des Protokolls der außerordentlichen Mitgliederversammlung 2015}

Das Protokoll der außerordentlichen Mitgliederversammlung des Jahres
2014 lag den Mitgliedern vor und wird in offener Wahl angenommen.

\begin{Abstimmung}
  \abstimmung{Ergebnis der Abstimmung}{15}{0}{0}[Das Protokoll der außerordentlichen Mitgliederversammlung 2015 wird angenommen]
\end{Abstimmung}

\topic{Rechenschaftsbericht des Vorstandes inkl.\,Schatzmeister}

Der Bericht des Vorstands befindet sich im Anhang zu diesem
Protokoll. Auf der Versammlung wird dieser vorgestellt und auf Fragen
dazu eingegangen.


\topic{Bericht der Kassenprüfer}

Adrian Pauli stellt das Ergebnis der Kassenprüfung vor. Er berichten
von regelmäßigen Treffen und unterjährlichen Prüfungen zussamen mit
Gerome Bochmann, wobei kein Fehler festgestellt werden konnte. Die
Kassenprüfung war erfolgreich. Es wird die Entlastung des Vorstands
empfohlen.

Nach dem Bericht des Kassenprüfers gab es eine Pause von 20 Minuten.

\topic{Abstimmung über Entlastung des Vorstandes}

Die Abstimmung zur Entlastung des Vorstands im Block erfolgt in offener Wahl.
\begin{Abstimmung}
  \abstimmung{Ergebnis der Abstimmung zur Entlastung des
    Vorstands}{15}{0}{0}[Der alte Vorstand wird entlastet.]
\end{Abstimmung}


\topic{Wahl des Vorstandes und der Kassenprüfer}

\subtopic{Vorsitzender}

Zur Wahl stellt sich Tim Schumacher. Die Abstimmung erfolgt in geheimer
Personenwahl.

\begin{Abstimmung}
  \abstimmung{Ergebnis der Abstimmung zur Wahl des Vorstandsvorsitzenden Kandidat Tim Schumacher}
  {14}{0}{1}
  [Damit wurde Tim Schumacher als Vorstandsvorsitzender gewählt]
\end{Abstimmung}

\beschluss{Vorstandsvorsitzender}{Tim Schumacher nimmt die Wahl an und ist damit als
  Vorstandsvorsitzender gewählt.}

\subtopic{Schriftführer}

Zur Wahl stellt sich Johanna Schell. Die Abstimmung erfolgt in geheimer Personenwahl.

\begin{Abstimmung}
  \abstimmung{Ergebnis der Abstimmung zur Wahl des Schriftführers Kandidat Johanna Schell}
  {14}{0}{1}
  [Damit wurde Johanna Schell als Schriftführer gewählt]
\end{Abstimmung}

\beschluss{Schriftführer}{Johanna Schell nimmt die Wahl an und ist damit als
  Schriftführerin gewählt.}

\subtopic{Schatzmeister}

Zur Wahl stellt sich Adrian Pauli. Die Abstimmung erfolgt in geheimer Personenwahl.

\begin{Abstimmung}
  \abstimmung{Ergebnis der Abstimmung zur Wahl des Schatzmeisters Kandidat Adrian Pauli}
  {14}{0}{1}
  [Adrian Pauli wird als Schatzmeister gewählt]
\end{Abstimmung}

\beschluss{Schatzmeister}{Martin Neß nimmt die Wahl an und ist damit als
  Schatzmeister gewählt.}

Der Vorstand ist damit komplett.
Die Stammdaten der Vorstandsmitglieder befinden sich im Anhang \ref{sec:neuer_vorstand}.

\topic{Bestimmung der Kassenprüfer}

Tim Hering und Johannes Melzer kandidieren als Kassenprüfer. Es wird
öffentlich und im Block durch Handzeichen abgestimmt.
\begin{Abstimmung}
  \abstimmung{Ergebnis der Abstimmung}{15}{0}{0}[Die Kandidatur war erfolgreich]
\end{Abstimmung}

\beschluss{Kassenprüfer}{Tim Hering und Johannes Melzer nehmen die Wahl an und sind damit als Kassenprüfer bestimmt.}

\topic{Diskussion und Abstimmung über Änderungsanträge an Satzung}
\subtopic{Satzung: Aufbau einer Werkstatt als Vereinzweck}

Frank Lanitz schlägt vor die Satzung so zu ändern das der Aufbau einer
Werkstatt zum Vereinszweck erkoren wird.

Es wird in offener Wahl durch Handzeichen abgestimmt.

\begin{Abstimmung}
  \abstimmung{Ergebnis der Abstimmung}{3}{12}{0}[Der Antrag wird
  abgelehnt.]
\end{Abstimmung}

\topic{Verschiedenes}

Das Plenum beschließt einstimmig, den Tagesordnungspunkt
``Verschiedenes'' zu überspringen und nach der Veranstaltung in einer
freien Diskussion zu diskutieren. Damit gib es keine weiteren Punkte.

\topic{Verabschiedung durch neuen Vorstand}
Der neue Vorstand dankt dem alten Vorstand für die geleistete Arbeit und freut
sich auf eine konstruktive Zusammenarbeit im neuen Jahr.

\newpage
\appendix

\section{Anwesenheitsliste}
% Amtsgericht möchte das haben.
\newpage
\section{Neuer Vorstand}

Der am 16.\,11.\,2014 gewählte Vorstand setzt sich wie folgt zusammen:

\label{sec:neuer_vorstand}
\begin{table}[h!]
    \centering
    \begin{tabularx}{\textwidth}{l|l}
        % Bitte Adressdaten und besonders Geburtsdaten nicht ins git pushen!
        \textsc{Name} & \textsc{Amt} \\ \hline
        Tim Daniel Schumacher & Vorsitzender \\
        Johanna Schell &  Schriftführer \\
        Adrian Pauli & Schatzmeister
    \end{tabularx}
\end{table}

\newpage
\section{Unterschriften}
\vspace{2cm}
\begin{multicols}{2}
  \noindent \makebox[5cm]{\hrulefill} \\
  Bernd Kampe \\
  Versammlungsleiter \\
  Jena, 25.\,11.\,2014

  \noindent \makebox[5cm]{\hrulefill} \\
  Tim Schumacher \\
  Protokollant \\
  Jena, 25.\,11.\,2014
\end{multicols}

\end{Protokoll}

\end{document}

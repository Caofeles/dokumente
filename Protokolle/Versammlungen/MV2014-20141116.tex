\documentclass[ngerman]{scrartcl}
\usepackage[utf8]{inputenc}
\usepackage{lmodern}
\usepackage{courier}
\usepackage[scaled=.92]{helvet}
\usepackage[T1]{fontenc}
\usepackage[ngerman]{babel}

\usepackage{minutes}
\usepackage{eurosym}
\usepackage{tabularx}

\begin{document}
\begin{Protokoll}{Protokoll der Mitgliederversammlung 2014 des Hackspace Jena e.\,V.}
\protokollant{Stephan Richter}
\teilnehmer{Zu Beginn: 16 Vereinsmitglieder}
\gaeste{ein Gast}
\moderation{Frank Lanitz}

\sitzungsdatum{16.\,November~2014}
\sitzungsbeginn{15:20 Uhr}
\sitzungsende{18:29 Uhr}
\sitzungsort{Krautspace, Krautgasse~26, 07743~Jena}

\protokollKopf{}

\topic{Begrüßung}

Schriftführer Felix Kästner und Schatzmeister Martin Neß begrüßen die
Anwesenden. Es befinden sich 16 ordentliche Mitglieder sowie ein Gast
im Raum. Martin Neß stell fest, dass die Versammlung beschlussfähig
ist, da laut Satzung mindestens zur Erfüllung der 23\%-Regelung
7~Mitglieder anwesend sein müssen.


\topic{Wahl des Versammlungsleiters \& Protokollführers}

  Einstimmig wird Frank Lanitz in einer offenen Wahl durch
  Handzeichen zum Versammlungsleiter gewählt.

  Stephan Richter wird in einer offenen Wahl durch Handzeichen
  einstimmig zum Protokollführer gewählt.

\topic{Abstimmung der Tagesordnung}

Die Tagesordnung wird einstimmig durch offene Wahl mit Handzeichen
angenommen.

\topic{Feststellung ordentliche Einladung}
Martin berichtet, dass die Einladung für die Versammlung mehr als
14~Tage im Voraus -- am 1.\,11.\,2014 gegen 16:02~Uhr -- an alle
Mitglieder per E-Mail sowie als Kopie an die öffentliche Mailingliste
versendet wurde. Die Korrektheit der Einladung wird von der
Versammlung einstimmig durch Handzeichen bestätigt.

\topic{Genehmigung des Protokolls der Mitgliederversammlung 2013}

Das Protokoll Mitgliederversammlung des Jahres 2013 lag
den Mitgliedern vor und wird in offener Wahl angenommen.

\begin{Abstimmung}
  \abstimmung{Ergebnis der Abstimmung}{12}{0}{4}[Das Protokoll der letzten Mitgliederversammlung wird angenommen]
\end{Abstimmung}

\topic{Rechenschaftsbericht des Vorstandes inkl.\,Schatzmeister}

Der Bericht des Vorstands befindet sich im Anhang zu diesem
Protokoll. Auf der Versammlung wird dieser vorgestellt und auf Fragen
dazu eingegangen.


\topic{Bericht der Kassenprüfer}

Adrian Pauli und Gerome Bochmann stellen das Ergebnis der
Kassenprüfung vor. Sie berichten von regelmäßigen Treffen und
unterjährlichen Prüfungen, wobei kein Fehler festgestellt werden
konnte. Die Kassenprüfung war erfolgreich. Es wird die Entlastung des
Vorstands empfohlen.

\topic{Abstimmung über Entlastung des Vorstandes}

Die Abstimmung zur Entlastung des Vorstands im Block erfolgt in offener Wahl.
\begin{Abstimmung}
  \abstimmung{Ergebnis der Abstimmung zur Entlastung des
    Vorstands}{14}{0}{2}[Der alte Vorstand wird entlastet.]
\end{Abstimmung}


\topic{Diskussion um zukünftige Ausrichtung des Vereins und Rolle des Vorstands}

Es wird diskutiert, wie in Zukunft die satzungsmäßigen Ziele des
Vereins umgesetzt werden sollen. Es findet keine Abstimmung oder
Festlegung fest.

Nach der Diskussion gibt es eine 10 minütige Pause.

\topic{Wahl des Vorstandes und der Kassenprüfer}

\subtopic{Beisitzer}

Es wird abgestimmt, ob der Vorstand durch 3 Beisitzer ergänzt werden soll.
\begin{Abstimmung}
  \abstimmung{Ergebnis der Abstimmung zur Bestimmung von Beisitzer}
  {2}{7}{7}
  [Es werden keine Beisitzer bestimmt.]
\end{Abstimmung}

Es werden also nur Vorsitzender, Schriftführer und Schatzmeister gewählt.

\subtopic{Vorsitzender}

Zur Wahl stellt sich Tim Schumacher. Die Abstimmung erfolgt in geheimer
Personenwahl. Tim ist zum Zeitpunkt der Wahl krankheitsbedingt
verhindert und hat Frank Lanitz eine Vollmacht über die mögliche
Annahme der Wahl erteilt.\footnote{Kopie der Vollmacht ist im Anhang
beigefügt}. Bei der Wahl gab es eine ungültige Stimme.

\begin{Abstimmung}
  \abstimmung{Ergebnis der Abstimmung zur Wahl des Vorstandsvorsitzenden Kandidat Tim Schumacher}
  {12}{1}{2}
  [Damit wurde Tim Schumacher als Vorstandsvorsitzender gewählt]
\end{Abstimmung}

Frank Lanitz erklärt, dass Tim Schumacher die Wahl annimmt.

\subtopic{Schriftführer}

Der amtierende Schriftführer Felix Kästner stellt sich wieder zur Wahl.
Als Gegenkandidaten treten Jan Huwald sowie Katja Hagenbring. Die
Abstimmung erfolgt in geheimer Personenwahl. Auf Jan Huwald entfallen 5
Stimmen, auf Katja Hagenbring 7 Stimmen sowie auf Felix Kästner 3
Stimmen. Bei der Wahl gab es eine Enthaltung. Es wird festgestellt, das
Katja Hagenbring die meisten Stimmen auf sich vereinigen konnte.

\beschluss{Schriftführer}{Katja Hagenbring nimmt die Wahl an und ist
  damit als Schriftführer gewählt.}

1 Mitglied verlässt die Veranstaltung. Damit sind noch 15
stimmberechtigte Personen anwesend.

\subtopic{Schatzmeister}

Zur Wahl stellt sich Martin Neß.
\begin{Abstimmung}
  \abstimmung{Ergebnis der Abstimmung zur Wahl des Schatzmeisters Kandidat Martin Neß}
  {13}{1}{1}
  [Martin Neß wird als Schatzmeister gewählt]
\end{Abstimmung}

\beschluss{Schatzmeister}{Martin Neß nimmt die Wahl an und ist damit als
  Schatzmeister gewählt.}

Der Vorstand ist damit komplett.
Die Stammdaten der Vorstandsmitglieder befinden sich im Anhang \ref{sec:neuer_vorstand}.
Nach dem Wahlgang gibt es eine zehnminütige Pause. Der Gast verlässt
die Veranstaltung.

\topic{Bestimmung der Kassenprüfer}

Adrian Pauli und Gerome Bochmann kandidieren erneut als Kassenprüfer. Es wird
öffentlich und im Block durch Handzeichen abgestimmt.
\begin{Abstimmung}
  \abstimmung{Ergebnis der Abstimmung}{13}{0}{2}[Die Kandidatur war erfolgreich]
\end{Abstimmung}

\beschluss{Kassenprüfer}{Adrian Pauli und Gerome Bochmann nehmen die Wahl an und sind damit als Kassenprüfer bestimmt.}

\topic{Diskussion und Abstimmung über Änderungsanträge an Satzung und Geschäftsordnung}
\subtopic{Satzung: Plenum als Vereinsorgan}

Frank Lanitz schlägt vor, ein Plenum als Vereinsorgan einzuführen. Über
den Satzungsänderungsentwurf wird diskutiert. Der Entwurf lag mit
Einladung zur Mitgliederversammlung unverändert vor.

Es wird in offener Wahl durch Handzeichen abgestimmt.
\begin{Abstimmung}
  \abstimmung{Ergebnis der Abstimmung}{1}{12}{2}[Der Antrag wird
  abgelehnt.]
\end{Abstimmung}

\subtopic{Notwendigkeit des Sicherheitsbeauftragten}

Es wird vorgeschlagen, den Sicherheitsbeauftragten laut
Geschäftsordnung nur noch fakultativ durch den Vorstand bestimmen zu
lassen. Es wird nach Diskussion offen durch Handzeichen abgestimmt.

\begin{Abstimmung}
  \abstimmung{Ergebnis der Abstimmung}{0}{11}{4}[Der Antrag wird
  abgelehnt.]
\end{Abstimmung}

Ein Mitglied verlässt die Veranstaltung. Damit sind jetzt 14
stimmberechtigte Mitglieder anwesend.

\subtopic{Verringerung des Mitgliedsbeitrages in Einzelfällen, drei
mögliche Versionen}

Es wird vorgeschlagen, die Beitragsordnung (§1 der Geschäftsordnung)
insofern zu ändern, als dass dem Vorstand eine Option obliegt, den
Beitrag in begründeten Ausnahmefällen auszusetzen bzw. unterhalb der
üblichen Sätze individuell anzupassen. Dafür liegen 3 mögliche
Vorschläge zur Diskussion. Nach einer Diskussion werden Version 1 und
Version 2 durch den Antragsteller zurückgezogen, so dass nur noch
Version 3 zur Abstimmung kommt. Es wird offen durch Handzeichen
abgestimmt.

\begin{Abstimmung}
  \abstimmung{Ergebnis der Abstimmung}{13}{1}{0}[Der Antrag wird
  angenommen.]
\end{Abstimmung}

\topic{Verschiedenes}

Das Plenum beschließt einstimmig, den Tagesordnungspunkt
``Verschiedenes'' zu überspringen und nach der Veranstaltung in einer
freien Diskussion zu diskutieren. Damit gib es keine weiteren Punkte.

\topic{Verabschiedung durch neuen Vorstand}
Der neue Vorstand dankt dem alten Vorstand für die geleistete Arbeit und freut
sich auf eine konstruktive Zusammenarbeit im neuen Jahr.

\newpage
\appendix

\section{Anwesenheitsliste}
% Amtsgericht möchte das haben.
\newpage
\section{Angenommene Geschäftsordnungsänderungsanträge}
\subsection{Neuer §\,1 der Geschäftsordnung wie auf der Mitgliederversammlung beschlossen}
Der neue Geschäftsordnungsparagraph 1 lautet nun:
\label{sec:neuer_go_artikel}
\begin{quote}
§1 Mitgliedsbeiträge

\begin{enumerate}
\item Der Verein erhebt gemäß §6 seiner Satzung Mitgliedsbeiträge wie
	folgt:
\begin{itemize}
	\item 8\euro{}/Monat für ermäßigte Mitgliedschaft
	\item 16\euro{}/Monat für normale Mitgliedschaft
	\item 32\euro{}/Monat, 42\euro{}/Monat, 64\euro{}/Monat,
		128\euro{}/Monat oder 256\euro{}/Monat für
		Fördermitglieder
\end{itemize}
\item Der Mitgliedsbeitrag wird für mindestens einen Monat im Voraus
	entrichtet. Einmal gezahlte Mitgliedsbeiträge werden nicht
	zurückerstattet.
\item Voraussetzung für eine ermäßigte Mitgliedschaft ist die Vorlage
	eines gültigen Schüler-, Studenten-, Renten- oder
	Behindertenausweises, einer Arbeitslosigkeitsbescheinigung oder
	eines vergleichbaren Nachweises gegenüber dem Vorstand. Der
	entsprechende Nachweis ist jährlich neu zu erbringen.
\item Jedem Mitglied steht es frei, den Verein durch einen höheren
	Mitgliedsbeitrag stärker finanziell zu unterstützen. Damit sind
	keinerlei Privilegien oder Stimmvorteile verbunden.
\item Der Vorstand ist berechtigt ausnahmsweise und bei Vorliegen
	besonderer Gründe den Mitgliedsbeitrag für einzelne Mitglieder
	auf deren Antrag individuell zwischen einschließlich 0\euro{}
	und dem in Ziffer 1 genannten Betrag festzulegen. Das Vorliegen
	besonderer Gründe ist dem Vorstand glaubhaft zu machen. Die
	Festlegung erfolgt nach Ermessen des Vorstandes und durch dessen
	einstimmige Entscheidung.
\item Änderungen bezüglich Mitgliedsbeitrag oder Mitgliedsart sind dem
	Vorstand schriftlich mitzuteilen und gelten mit sofortiger Wirkung.
\end{enumerate}

\end{quote}

\newpage
\section{Neuer Vorstand}

Der am 16.\,11.\,2014 gewählte Vorstand setzt sich wie folgt zusammen:

\label{sec:neuer_vorstand}
\begin{table}[h!]
    \centering
    \begin{tabularx}{\textwidth}{l|l}
        % Bitte Adressdaten und besonders Geburtsdaten nicht ins git pushen!
        \textsc{Name} & \textsc{Amt} \\ \hline
        Tim Daniel Schumacher & Vorsitzender \\
        Katja Hagenbring &  Schriftführer \\
        Martin Neß & Schatzmeister
    \end{tabularx}
\end{table}

\newpage
\section{Unterschriften}
\vspace{2cm}
\begin{multicols}{2}
  \noindent \makebox[5cm]{\hrulefill} \\
  Frank Lanitz \\
  Versammlungsleiter \\
  Dresden, 25.\,11.\,2014

  \noindent \makebox[5cm]{\hrulefill} \\
  Stepahn Richter \\
  Protokollant \\
  Jena, 28.\,11.\,2014
\end{multicols}

\end{Protokoll}

\end{document}

\documentclass{scrartcl}
\usepackage[utf8]{inputenc}
\usepackage[T1]{fontenc}
\usepackage[ngerman]{babel}
\title{Protokoll der Mitgliederversammlung 2013.1}
\author{Hackspace Jena e.\,V.}
\date{24.\,11.\,2013}
\begin{document}
\maketitle{}


\begin{itemize}
\item Anwesende: Zu Beginn 19 Vereinsmitglieder, ein Gast
\item Protokollant: Jan Huwald
\item Versammlungsleitung: Konrad Schöbel
\item Ort: Krautspace, Krautgasse 26 in Jena
\item Zeit: 24.11.2013 von 15:15 bis 18:30 % @TODO End-Zeit?
\end{itemize}

\section{Begrüßung}
Schatzmeister Frank Lanitz und Vorstandsvorsitzender Konrad Schöbel begrüßen die Anwesenden.
Frank Lanitz stellt fest, dass zu Beginn der Versammlung 19 ordentliche
Mitglieder und 1 Gast anwesend sind. Die Versammlung ist beschlussfähig,
da mindestens 7 Mitglieder anwesend sein müssen.

\section{Wahl des Versammlungsleiters}
Einstimmig wird Konrad Schöbel in einer offenen Wahl durch Handzeichen
zum Versammlungsleiter gewählt.

\section{Wahl des Protokollführers}
Jan Huwald wird in einer offenen Wahl durch Handzeichen mit 18
Stimmen und einer Enthaltungen zum Versammlungsleiter gewählt.

\section{Genehmigung der Tagesordnung}
Die Tagesordnung wird einstimmig durch offene Wahl mit Handzeichen
angenommen.

Frank berichtet, dass die Einladung für die Versammlung 14 Tage im
Voraus -- am 4.11.2013 gegen 23:35 -- an alle Mitlgieder per Email
sowie als Kopie an die öffentliche Mailingliste versendet wurde.
Die Korrektheit der Einladung wurde von der
Versammlung einstimmig durch Handzeichen bestätigt.

\section{Abstimmung Protokoll der 2. Mitgliederversammlung 2012}
Das Protokoll wird in einer offenen Wahl durch Handzeichen mit 18 Stimmen
und 1 Enthaltungen angenommen.

\section{Rechenschaftsbericht des Vorstandes}
Der Bericht des Vorstandes befindet sich im Anhang zu diesem Protokoll.
Auf der Versammlung wird dieser vorgestellt und auf Fragen dazu
eingegangen.

Rückfragen und Diskussion gibt es zur Mitgliederfluktuation.

\section{Bericht der Kassenprüfer}
Martin Neß und Adrian Pauli stellen das Ergebnis der Kassenprüfung vor.
Die Kassenprüfung war erfolgreich. Es wird die Entlastung des Vorstandes empfohlen.

%In einer offenen Abstimmung durch Handzeichen wurde die Kassenprüfung
%mit 10 Stimmen und einer Enthaltung angenommen.

\section{Entlastung des Vorstandes} % todo Name?
In einer offenen Wahl wird der Vorstand mit 17 Stimmen und 2
Enthaltung entlastet.

\section{Entlastung des Schatzmeisters}
In einer offenen Wahl wird der Schatzmeister mit 18 Stimmen und einer
Enthaltung entlastet.

\section{Wahl des Vorstandes und der Kassenprüfer}
Ein Meinungsbild ergab, dass keine Beisitzer gewählt werden sollen. Es
werden also nur Vorsitzender, Schriftführer und Schatzmeister gewählt.

\subsection{Vorsitzender}
Zur Wahl stellen sich Tim und Jens. % @todo Namen

Es wird in geheimer Personenwahl abgestimmt.

\begin{table}[h!]
    \centering
    \begin{tabular}{c|c|c|c}
        \textbf{Tim} & \textbf{Jens} & \textbf{Enthaltungen} &
\textbf{Ungültig} \\ \hline
        7 & 10 & 1 & 1 \\
    \end{tabular}
\end{table}

Jens nimmt die Wahl zum Vorsitzenden an.

\subsection{Schriftführer}
Zur Wahl stellt sich Felix Kästner (amtierender Schriftführer).
Es wird in geheimer Personenwahl abgestimmt.

\begin{table}[h!]
    \centering
    \begin{tabular}{c|c|c}
        \textbf{Felix} & \textbf{Enthaltungen} &
\textbf{Ungültig} \\ \hline
        18 & 1 & 0 \\
    \end{tabular}
\end{table}

Felix Kästner nimmt die Wahl an.

\subsection{Schatzmeister}

Zur Wahl stellt sich Martin Neß und Markus. % @todo Name

\begin{table}[h!]
    \centering
    \begin{tabular}{c|c|c|c}
        \textbf{Martin} & \textbf{Markus} & \textbf{Enthaltungen} &
\textbf{Ungültig} \\ \hline
        11 & 5 & 3 & 0 \\
    \end{tabular}
\end{table}

Martin Neß nimmt die Wahl an.

% Pause
Ein Gast verlässt die Veranstaltung.

\subsection{Kassenprüfer}

Adrian Pauli und Gerome kandidieren für den Posten der % @todo Name
Kassenprüfer. Es wird öffentlich und im Block durch Handzeichen abgestimmt.

\begin{table}[h!]
    \centering
    \begin{tabular}{c|c|c}
        \textbf{Ja}& \textbf{Nein} & \textbf{Enthaltungen} \\ \hline
        17 & 0 & 2
    \end{tabular}
\end{table}

Der Vorstand ist damit komplett. Die Stammdaten befinden sich im Anhang unter \ref{sec:neuer_vorstand}.

\section{Geschäftsordnungsänderungsanträge}
\subsection{Veränderungen des Mitgliedsbeitrages}
Der Antrag über die Erhöhung der Mitgliedsbeitrage wird von Martin Neß vorgestellt.
Anschließend wird über vermutete finanzielle Auswirkungen diskutiert.

Es wird in offener Wahl durch Handzeichen abgestimmt:

\begin{table}[h!]
    \centering
    \begin{tabular}{c|c|c}
        \textbf{Ja} & \textbf{Nein} & \textbf{Enthaltung} \\ \hline
        3 & 14 & 2
    \end{tabular}
\end{table}

Der Antrag wurde abgelehnt.

\subsection{Verringerung der Informationspflicht des Schatzmeisters}
Der Antrag wird von Martin Neß vorgestellt.

Es wird in offener Wahl durch Handzeichen abgestimmt:

\begin{table}[h!]
    \centering
    \begin{tabular}{c|c|c}
        \textbf{Ja} & \textbf{Nein} & \textbf{Enthaltung} \\ \hline
        19 & 0 & 0
    \end{tabular}
\end{table}

Der Antrag wurde angenommen.

\subsection{Höhere Flexibilität für Fördermitglieder}
Der Antrag wird von Frank Lanitz vorgestellt.
Anschließend lebhaft diskutiert und es werden 
Formulierungsprobleme bzgl. Wirkreichweite des Antrags festgestellt.

Der Antrag wird von Frank zurückgezogen.

Die Versammlung wurde um 18:30 aufgelöst.  % @TODO End-Zeit?

\newpage 
\appendix{}
\section{Anwesenheitsliste}
% Amtsgericht möchte das haben. 
\newpage 
\section{Angenommene Geschäftsordnungsänderungsanträge}
\subsection{Neuer §5 der Geschäftsordnung wie auf der Mitgliederversammlung beschlossen}
Der Artikel 3 lautet nun wie folgt: 

\begin{quote}
%\label{sec:new_para_1}

§5 Aufgaben des Schatzmeisters

\begin{enumerate}
% hier muss eine "3." eingefügt werden
    \item Der Schatzmeister informiert die Vereinsmitglieder
        mindestens vierteljährlich über den
        Kassenstand. Einnahmen und Ausgaben über 100\euro{} sind dabei
        einzeln aufzulisten.
\end{enumerate}
\end{quote}

\newpage 
\section{Neuer Vorstand}

Der am 24.11.2013 gewählte Vorstand setzt sich wie folgt zusammen: 

\label{sec:neuer_vorstand}
\begin{table}[h!]
    \centering
    \begin{tabular}{l|l|c|l}
        %Adressdaten stehen da mal nicht drin. Reicht, wenn sie im 
        %Vereinsregister stehen.
        \textbf{Name} & \textbf{Adresse} & \textbf{Geburstag} & \textbf{Amt} \\ \hline
        Jens Kubieziel & & & Vorsitzender \\
        Felix Kästner & & & Schriftführer \\
        Martin Neß & & & Schatzmeister 
    \end{tabular}
\end{table}
%\end{quote}
\end{document}

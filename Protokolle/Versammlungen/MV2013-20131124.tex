\documentclass[ngerman]{scrartcl}
\usepackage[utf8]{inputenc}
\usepackage{lmodern}
\usepackage{courier}
\usepackage[scaled=.92]{helvet}
\usepackage[T1]{fontenc}
\usepackage[ngerman]{babel}

\usepackage{minutes}
\usepackage{eurosym}
\usepackage{tabularx}

\begin{document}
\begin{Protokoll}{Protokoll der Mitgliederversammlung 2013 des Hackspace
    Jena e.\,V.}
\protokollant{Jan Huwald}
\teilnehmer{Zu Beginn: 19 Vereinsmitglieder}
\gaeste{ein Gast}
\moderation{Konrad Schöbel}

\sitzungsdatum{24.\,November~2013}
\sitzungsbeginn{15:15 Uhr}
\sitzungsende{18:25 Uhr}
\sitzungsort{Krautspace, Krautgasse~26, 07743~Jena}

\protokollKopf

\topic{Begrüßung}

Schatzmeister Frank Lanitz und Vorstandsvorsitzender Konrad Schöbel
begrüßen die Anwesenden.  Frank Lanitz stellt fest, dass zu Beginn der
Versammlung 19~ordentliche Mitglieder und ein Gast anwesend sind. Die
Versammlung ist beschlussfähig, da laut Satzung mindestens
7~Mitglieder anwesend sein müssen.


\topic{Wahl des Versammlungsleiters}

  Einstimmig wird Konrad Schöbel in einer offenen Wahl durch
  Handzeichen zum Versammlungsleiter gewählt.

\topic{Wahl des Protokollführers}

  Jan Huwald wird in einer offenen Wahl durch Handzeichen mit
  18~Stimmen und einer Enthaltungen zum Protokollführer gewählt.
  \begin{Abstimmung}
    \abstimmung{Ergebnis der Abstimmung}{18}{0}{1}
  \end{Abstimmung}

\topic{Abstimmung der Tagesordnung}

Die Tagesordnung wird einstimmig durch offene Wahl mit Handzeichen
angenommen.

Frank berichtet, dass die Einladung für die Versammlung mehr als
14~Tage im Voraus -- am 4.\,11.\,2013 gegen 23:35~Uhr -- an alle
Mitglieder per E-Mail sowie als Kopie an die öffentliche Mailingliste
versendet wurde.  Die Korrektheit der Einladung wird von der
Versammlung einstimmig durch Handzeichen bestätigt.

\topic{Genehmigung des Protokolls der Mitgliederversammlung 2012.2}

Das Protokoll der zweiten Mitgliederversammlung des Jahres 2012 lag
den Mitgliedern vor und wird in offener Wahl angenommen.
\begin{Abstimmung}
  \abstimmung{Ergebnis der Abstimmung}{18}{0}{1}
\end{Abstimmung}

\topic{Rechenschaftsbericht des Vorstandes inkl.\,Schatzmeister}

Der Bericht des Vorstands befindet sich im Anhang zu diesem
Protokoll. Auf der Versammlung wird dieser vorgestellt und auf Fragen
dazu eingegangen.

Rückfragen und Diskussion gibt es zur Mitgliederfluktuation.

\topic{Bericht der Kassenprüfer}

Martin Neß und Adrian Pauli stellen das Ergebnis der Kassenprüfung
vor.  Die Kassenprüfung war erfolgreich. Es wird die Entlastung des
Vorstands empfohlen.


\topic{Abstimmung über Entlastung des Vorstandes sowie der Kassenprüfer}
\subtopic{Entlastung des Vorstands}
Die Abstimmung zur Entlastung des Vorstands erfolgt in offener Wahl.
\begin{Abstimmung}
  \abstimmung{Ergebnis der Abstimmung zur Entlastung des
    Vorstands}{17}{0}{2}[Damit ist der Vorstand entlastet.]
\end{Abstimmung}


\subtopic{Entlastung des Schatzmeisters}

Die Abstimmung zur Entlastung des Schatzmeisters erfolgt in offener Wahl.
\begin{Abstimmung}
  \abstimmung{Ergebnis der Abstimmung zur Entlastung des
    Schatzmeisters}{18}{0}{1}[Damit ist der Schatzmeister entlastet.]
\end{Abstimmung}

Es wird einstimmig festgestellt, dass die Kassenprüfer nicht entlastet werden müssen.

\topic{Wahl des Vorstandes und der Kassenprüfer}

Ein Meinungsbild ergibt, dass keine Beisitzer gewählt werden sollen. Es
werden also nur Vorsitzender, Schriftführer und Schatzmeister gewählt.

\subtopic{Vorsitzender}

Zur Wahl stellen sich Tim Schumacher und Jens Kubieziel. Die
Abstimmung erfolgt in geheimer Personenwahl. Tim kann 7~Stimmen auf
sich vereinen, Jens erzielt 10~Stimmen. Daneben gibt es eine
Enthaltung und eine ungültige Stimmabgabe.

\beschluss{Vorsitzender}{Jens Kubieziel nimmt die Wahl an und ist
  damit als Vorsitzender gewählt.}

\subtopic{Schriftführer}

Der amtierende Schriftführer Felix Kästner stellt sich wieder zur
Wahl. Gegenkandidaten gibt es keine. Die Abstimmung erfolgt in
geheimer Personenwahl. 18~Mitglieder stimmen für Felix. Es gibt eine
Enthaltung und keine Gegenstimme.

\beschluss{Schriftführer}{Felix Kästner nimmt die Wahl an und ist
  damit als Schriftführer gewählt.}

\subtopic{Schatzmeister}

Zur Wahl stellen sich Martin Neß und Markus Oehme (in Abwesenheit). Die Abstimmung
erfolgt in geheimer Personenwahl. Auf Martin entfallen 11~Stimmen, auf
Markus 5~Stimmen und es gibt 3~Enthaltungen.
\beschluss{Schatzmeister}{Martin Neß nimmt die Wahl an und ist damit als
  Schatzmeister gewählt.}

Der Vorstand ist damit komplett.
Die Stammdaten der Vorstandsmitglieder befinden sich im Anhang \ref{sec:neuer_vorstand}.
Nach dem Wahlgang gibt es eine zehnminütige Pause. Der Gast verlässt
die Veranstaltung.

\topic{Bestimmung der Kassenprüfer}

Adrian Pauli und Gerome Bochmann kandidieren als Kassenprüfer. Es wird
öffentlich und im Block durch Handzeichen abgestimmt.
\begin{Abstimmung}
  \abstimmung{Ergebnis der Abstimmung}{17}{0}{2}
\end{Abstimmung}


\topic{Diskussion über Änderung der Geschäftsordnung und Abstimmung darüber:}
\subtopic{Veränderungen des Mitgliedsbeitrages}

Der Antrag über die Erhöhung der Mitgliedsbeitrage wird von Martin Neß
vorgestellt.  Anschließend wird über vermutete finanzielle
Auswirkungen diskutiert.

Es wird in offener Wahl durch Handzeichen abgestimmt.
\begin{Abstimmung}
  \abstimmung{Ergebnis der Abstimmung}{3}{14}{2}[Der Antrag wird
  abgelehnt.]
\end{Abstimmung}

\subtopic{Verringerung der Informationspflicht des Schatzmeisters}
Der Antrag wird von Martin Neß vorgestellt.
Es wird in offener Wahl durch Handzeichen abgestimmt.
\begin{Abstimmung}
  \abstimmung{Ergebnis der Abstimmung}{19}{0}{0}[Der Antrag wird
  angenommen.]
\end{Abstimmung}
§5 der Geschäftsordnung lautet nun wie im Anhang unter
\ref{sec:neuer_go_artikel} abgedruckt.

\subtopic{Höhere Flexibilität für Fördermitglieder}
Der Antrag wird
von Frank Lanitz vorgestellt, anschließend lebhaft diskutiert und es
werden Formulierungsprobleme bzgl. Wirkreichweite des Antrags
festgestellt.

Der Antrag wird von Frank zurückgezogen.

\topic{Verschiedenes}

Es gab keine weiteren Punkte.

\topic{Verabschiedung durch neuen Vorstand}
Der neue Vorstand dankt dem Alten für die geleistete Arbeit und freut
sich auf eine konstruktive Zusammenarbeit im neuen Jahr.

\newpage
\appendix

\section{Anwesenheitsliste}
% Amtsgericht möchte das haben.
\newpage
\section{Angenommene Geschäftsordnungsänderungsanträge}
\subsection{Neuer §\,5 der Geschäftsordnung wie auf der Mitgliederversammlung beschlossen}
Der Artikel 3 lautet nun wie folgt:

\label{sec:neuer_go_artikel}
\begin{quote}
§5 Aufgaben des Schatzmeisters

\begin{enumerate}
    \setcounter{enumi}{2}
    \item Der Schatzmeister informiert die Vereinsmitglieder
        mindestens vierteljährlich über den
        Kassenstand. Einnahmen und Ausgaben über 100~\euro{} sind dabei
        einzeln aufzulisten.
\end{enumerate}
\end{quote}

\newpage
\section{Neuer Vorstand}

Der am 24.\,11.\,2013 gewählte Vorstand setzt sich wie folgt zusammen:

\label{sec:neuer_vorstand}
\begin{table}[h!]
    \centering
    \begin{tabularx}{\textwidth}{l|X|c|l|l}
        % Bitte Adressdaten und besonders Geburtsdaten nicht ins git pushen! 
        \textsc{Name} & \textsc{Adresse} & \textsc{Geburtstag} & \textsc{Geburtsort} & \textsc{Amt} \\ \hline
        Jens Kubieziel & & & & Vorsitzender \\
        Felix Kästner & & & & Schriftführer \\
        Martin Neß & & & & Schatzmeister
    \end{tabularx}
\end{table}

\newpage
\section{Unterschriften}
\vspace{2cm}
\begin{multicols}{2}
  \noindent \makebox[5cm]{\hrulefill} \\
  Konrad Schöbel \\
  Versammlungsleiter \\
  Jena, 3.\,12.\,2013

  \noindent \makebox[5cm]{\hrulefill} \\
  Jan Huwald \\
  Protokollant \\
  Jena, 3.\,12.\,2013
\end{multicols}

\end{Protokoll}

\end{document}
